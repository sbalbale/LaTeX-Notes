\documentclass{article}
\usepackage{graphicx}
\usepackage{float}
\usepackage{booktabs}
\usepackage{siunitx}
\usepackage[fleqn]{amsmath}


\title{Lab 5: Thevenin Equivalents of Lab Equipment}
\author{Sean Balbale}
\date{October 11th, 2024}
\setlength{\parindent}{0in}

\begin{document}

\begin{titlepage}
	\begin{center}
		\vspace*{1in}

		\Huge
		\textbf{Lab 5}

		\LARGE
		Thevenin Equivalents of Lab Equipment

		\vspace{3 in}

		\textbf{Student Name:} Sean Balbale
		\\ \textbf{Instructor:} Dr. Iman Salama
		\\ \textbf{Lab Partner Name:} Krish Gupta
		\\ \textbf{Date:} October 11, 2024

		\vfill


	\end{center}
\end{titlepage}

\newpage


\section{Introduction}
In this lab, the concept of Thevenin equivalents was explored, a powerful tool
in electrical engineering used to simplify complex circuits. By representing a
complex circuit as a combination of a voltage source and an impedance, circuit
analysis became more manageable. This approach was particularly useful in
systems such as ECG amplifiers or RF amplifiers in cell phones, where
understanding the interaction between sub-circuits was essential for effective
design and analysis. Through practical experiments, the Thevenin equivalents of
lab equipment, such as oscilloscopes and signal generators, were determined,
using measurement techniques like voltage division to calculate key parameters,
including Thevenin voltage and resistance. By the end of the lab, insights were
gained into how simplified models enabled the design and analysis of complex
systems with greater ease.

\section{Results}


\section{Discussions and Conclusions}

\section{References}
 [1] Dr. Iman Salama. “Lab 5 – Thevenin Equivalents of Lab Equipment” Northeastern University. 11 October 2024.

\end{document}
