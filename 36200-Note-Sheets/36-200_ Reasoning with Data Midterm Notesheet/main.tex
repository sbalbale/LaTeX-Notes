\documentclass{article}
\usepackage[utf8]{inputenc}
\usepackage[margin =0.5in]{geometry}
\usepackage[inline]{enumitem}  
\usepackage{tabularx}

\author{Sean Balbale}
\date{July 2022}

\begin{document}
\section*{36-200: Reasoning with Data Midterm Note Sheet:}
\begin{enumerate*}[label={}]
    \item Name: Sean Balbale
    \item Andrew Id: sbalbale
    \item July, 2022
\end{enumerate*}
\subsection*{Formulas:}
\textbf{Standard Deviation:}
\\For a sample containing n data values, the \textbf{Sample Standard Deviation:}
$S = \sqrt{\frac{\sum_{i=1}^{n} (X_{i} - \overline{X})^2}{n-1}}$
\\$X_{i}$ is the variable, $\overline{X}$ is the sample mean.
\\For a population containing N data values, the \textbf{Population Standard Deviation:}
$\sigma = \sqrt{\frac{\sum_{i=1}^{n} (x_{i} - {\mu})^2}{N}}$
\\$X_{i}$ is the variable, $\mu$ is the population mean.
\\S is a statistic, $\sigma$ is a parameter.
\\\textbf{Binomial Formula:} 
$P(X = x) = {n \choose x} p^x \cdot (1-p)^{n-x}$
\\P is the probability that the outcome occurs,n is the total number of trials,
\\X is the variable, x is the instance of the variable
\\\textbf{Binomial Coefficient:} $nCx = {n \choose x} = \frac{n!}{x!(n-x)!}$
\\\textbf{Probability:}
\begin{itemize}
    \item P(A): What is the probability of A happening, Marginal Probability
    \item $P(A \bigcap B)$: P(A and B) = 
    \\\textbf{If statistically independent:}$P(A) \cdot P(B)$, Joint Probability
    \\\textbf{If statistically dependant:} $P(A) \cdot P(B|A)$
    \item $P(A \bigcup B)$: P(A or B) = 
    \\\textbf{If mutually exclusive:} P(A) + P(B), Disjoint Probability
    \\\textbf{If not mutually exclusive:} P(A) + P(B) - $P(A \bigcap B)$
    \item $P(A|B)$: P(A given B) = $\frac{P(A \bigcap B)}{P(B)}$, Conditional Probability
\end{itemize}
\subsection*{Definitions:}
\textbf{Population:} The complete set of people or objects of interest or the infinite set of all possible values if the same person or item were repeatedly measured in the same way.
\\\textbf{Sample:} The subset of the population for which data can actually be obtained in a study.
\\\textbf{Parameter:} A fixed (but usually unknown) number summarizing some feature of a population.
\\\textbf{Statistic:} A computed number that summarizes sample data in some appropriate way, and which estimates a parameter.
\\\textbf{Inference:} Specifying the estimate of an unknown parameter.
\\\textbf{Mean:} The arithmetic average.
\\\textbf{Median:} The middle value after ordering.
\\\textbf{Standard Deviation:} The typical variation from the mean.
\\\textbf{Variance:} The square of the standard deviation.
\\\textbf{Quartiles:} Q1 is the 25th percentile, Q3 is the 75th percentile
\\\textbf{Five-Number Summary:} Minimum, Q1, Median, Q3, Maximum
\\\textbf{Inter Quartile Range (IQR):} The difference between quartiles.
\\\textbf{The Pearson Correlation Coefficient:} A unitless number that measures the direction and strength between two quantitative variables.
\begin{itemize}
    \item Between -1 and 1
    \item The sign (+ or -) matches the direction of the relationship.
    \item Closer to Zero: \textit{less} linear.
    \item Closer to 1 or -1: \textit{more} linear.
    \item Correlation coefficient \textit{exactly} 1 or \textit{exactly} -1 only if all the points fall on a \textit{perfect} line.
\end{itemize}
\textbf{Skewness and Modality:}
\begin{itemize}
    \item Skewed Right: Tail to the Right.
    \item Skewed Left: Tail to the Left.
    \item Unimodal: One peak.
    \item Multimodal: Multiple peaks.
\end{itemize}
\textbf{Statistical Independence:} 
A and B are statistically independent if and only if $P(A|B) = P(A|B^C) = P(A)$, $B^C$ is the opposite of B.
\subsection*{Notation:}
\begin{itemize}
    \item \textbf{Uppercase vs. lowercase:} Uppercase denotes a random variable, Lowercase denotes a number.
    \item \textbf{P's:}
    \begin{itemize}
        \item $\hat{P}$: Denotes sample proportion, estimates \textit{p}.
        \item \textit{p}: Denotes the 'true' population proportion, is a population parameter.
        \item P( ): Denotes the "probability of...".
    \end{itemize}
    \item \textbf{X vs. $\overline{X}$:}
    \begin{itemize}
        \item X: Denotes sample \textit{count}, in the case of a \textit{categorical} measurement.
        \item $\overline{X}$: Denotes sample \textit{mean}, in the case of a \textit{quantitative} measurement.
    \end{itemize}
\end{itemize}
\subsection*{Examples:}
\textbf{Data:}
\\
\begin{tabularx}{0.8\textwidth} { 
  | >{\centering\arraybackslash}X | >{\centering\arraybackslash}X | }
 \hline
 Gray & Children\\
 \hline
 Yes  & Yes\\
 \hline
 No & Yes\\
 \hline
 Yes  & No\\
 \hline
 Yes & Yes\\
  \hline
 No  & No\\
  \hline
 No  & No\\
  \hline
 No  & No\\
 \hline
 Yes & Yes\\
  \hline
 No  & No\\
  \hline
 No  & Yes\\
\hline
\end{tabularx}
\\\textbf{Contingency Table of Counts:}
\\
\begin{tabularx}{0.8\textwidth} { 
  | >{\centering\arraybackslash}X | >{\centering\arraybackslash}X | >{\centering\arraybackslash}X | >{\centering\arraybackslash}X | }
 \hline
   & Children & No Children & total\\
 \hline
 Gray & 3 & 1 & 4 \\
 \hline
 No Gray  & 2  & 4 & 6 \\
 \hline
 Total  & 5  & 5 & 10\\
\hline
\end{tabularx}
\\\textbf{Probability Table:}
\\
\begin{tabularx}{0.8\textwidth} { 
  | >{\centering\arraybackslash}X | >{\centering\arraybackslash}X | >{\centering\arraybackslash}X | >{\centering\arraybackslash}X | }
 \hline
   & Children & No Children & total\\
 \hline
 Gray & 0.3 & 0.1 & 0.4 \\
 \hline
 No Gray  & 0.2  & 0.4 & 0.6 \\
 \hline
 Total  & 0.5  & 0.5 & 1.0\\
\hline
\end{tabularx}
\\
\\\textbf{Analysis:}
This data isn't independent because the actual $P(Children \bigcap Gray) = 0.3$ not $P(Children) \times P(Gray) = P(Children \bigcap Gray) = 0.2$
\end{document}

