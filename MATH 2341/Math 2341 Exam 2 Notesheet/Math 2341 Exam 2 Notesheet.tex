\documentclass[10pt]{article}
\usepackage[margin=0.25in]{geometry}
\usepackage{amsmath, amssymb}
\usepackage{multicol}
\usepackage{nicematrix}
\setlength{\columnsep}{0.7cm} 
\setlength{\parindent}{0in}
\title{\raggedright \large Math 2341 Exam 1 Notesheet \hfill Sean Balbale \hfill October 2024 \vspace{-3em}}
\date{}

\begin{document}
\maketitle
\begin{multicols}{2}

\section*{Laplace Transform}
\subsection*{3.1 Laplace Transform and Inverse}
\[
\mathcal{L}\{f(t)\} = \int_0^\infty e^{-st}f(t) \, dt, \quad \mathcal{L}^{-1}\{F(s)\} = f(t)
\]

\subsection*{3.2 Transforms of Derivatives and IVPs}
\[
\mathcal{L}\{f'(t)\} = sF(s) - f(0), \quad \mathcal{L}\{f''(t)\} = s^2F(s) - sf(0) - f'(0)
\]

\subsection*{3.3 Shifting Theorems}
\[
\mathcal{L}\{e^{at}f(t)\} = F(s-a), \quad \mathcal{L}\{u_c(t)f(t-c)\} = e^{-cs}F(s)
\]

\subsection*{3.4 Discontinuous Inputs}
\textbf{Heaviside function:} \( u_c(t) \), \\
\[
\mathcal{L}\{u_c(t)f(t-c)\} = e^{-cs}\mathcal{L}\{f(t)\}
\]

\subsection*{3.5 Convolution}
\textbf{Convolution Theorem:}
\[
(f * g)(t) = \int_0^t f(\tau)g(t-\tau) \, d\tau, \quad \mathcal{L}\{f * g\} = F(s)G(s)
\]

\section*{Systems of Equations and Matrices (Chapter 4)}
\subsection*{4.1 Systems and Matrices}
\textbf{Representation:} A system of equations can be written as \( A x = b \), where:
\[
A = \begin{bNiceMatrix}
a_{11} & a_{12} & \cdots & a_{1n} \\
a_{21} & a_{22} & \cdots & a_{2n} \\
\vdots & \vdots & \ddots & \vdots \\
a_{m1} & a_{m2} & \cdots & a_{mn}
\end{bNiceMatrix}, \quad x = \begin{bNiceMatrix}
x_1 \\ x_2 \\ \vdots \\ x_n
\end{bNiceMatrix}, \quad b = \begin{bNiceMatrix}
b_1 \\ b_2 \\ \vdots \\ b_m
\end{bNiceMatrix}
\]

\subsection*{4.2 Gaussian Elimination}
\textbf{Steps:}
\begin{itemize}
    \item \textbf{Forward elimination:} Reduce the system to an upper triangular matrix.
    \item \textbf{Back substitution:} Solve for variables starting from the last row.
\end{itemize}
Example:
\[
\begin{bNiceMatrix}
2 & 1 & -1 & | & 8 \\
-3 & -1 & 2 & | & -11 \\
-2 & 1 & 2 & | & -3
\end{bNiceMatrix}
\rightarrow
\begin{bNiceMatrix}
2 & 1 & -1 & | & 8 \\
0 & -0.5 & 0.5 & | & -1 \\
0 & 0 & 1 & | & -2
\end{bNiceMatrix}
\]

\subsection*{4.3 Row-Echelon Form and Rank}
\textbf{Definitions:}
\begin{itemize}
    \item \textbf{Row-echelon form:} Leading coefficients (pivots) are to the right of the pivots in the rows above.
    \item \textbf{Rank:} The number of pivot positions in the matrix.
\end{itemize}

\textbf{Example (Row-Echelon Form):}
\[
\begin{bNiceMatrix}
1 & 2 & 3 & | & 4 \\
0 & 1 & 4 & | & 5 \\
0 & 0 & 1 & | & 6
\end{bNiceMatrix}
\]
This is in row-echelon form because each leading entry is to the right of the leading entry in the row above, and all rows of zeros are at the bottom.

\subsection*{4.6 Cofactor Expansions}
\textbf{Determinant:}
\[
\text{det}(A) = \sum (-1)^{i+j}a_{ij}\text{det}(A_{ij})
\]
\textbf{Example:}
\[
\text{For } A = \begin{bNiceMatrix}
1 & 2 & 3 \\
0 & 4 & 5 \\
1 & 0 & 6
\end{bNiceMatrix}, \quad \text{det}(A) = 1(24 - 0) - 2(0 - 5) + 3(0 - 4) = 22
\]

\section*{Eigenvalues and Eigenvectors (Chapter 6)}
\subsection*{6.1 Eigenvalues and Eigenvectors}
\textbf{Definition:} \( Ax = \lambda x \), where \( \lambda \) is an eigenvalue and \( x \) is an eigenvector. \\
\textbf{Characteristic Polynomial:}
\[
\text{det}(A - \lambda I) = 0
\]
\textbf{Example:}
\[
A = \begin{bNiceMatrix}
4 & 2 \\
1 & 3
\end{bNiceMatrix}, \quad \text{det}(A - \lambda I) = \text{det}\begin{bNiceMatrix}
4-\lambda & 2 \\
1 & 3-\lambda
\end{bNiceMatrix} = (\lambda - 5)(\lambda - 2)
\]
Eigenvalues: \(\lambda = 5, 2\).

\subsection*{Diagonalization}
\textbf{Matrix Diagonalization:}
\begin{align*}
    A = PDP^{-1}&, \quad P: \text{Matrix of eigenvectors} \\
    &, \quad D: \text{Diagonal matrix of eigenvalues}.
\end{align*}
\textbf{Steps:}
\begin{itemize}
    \item Find eigenvalues \(\lambda\) from \(\text{det}(A - \lambda I) = 0\).
    \item Solve \((A - \lambda I)x = 0\) for eigenvectors.
    \item Construct \(P\) and \(D\).
\end{itemize}

\end{multicols}
\end{document}
