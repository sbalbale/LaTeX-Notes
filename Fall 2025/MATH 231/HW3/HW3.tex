\documentclass[reqno, 12pt]{amsart}

\usepackage{amssymb, amsmath, amsthm, enumerate, mathtools, graphicx, enumitem, tikz, color, soul, bbm, verbatim, parskip, multicol}
\usepackage[margin=1 in]{geometry}
\usepackage[mathscr]{euscript}
\usepackage[urlcolor=blue,colorlinks=true]{hyperref}
\usepackage{mdframed}
\setlist[itemize]{noitemsep, topsep=0pt, parsep=0pt, partopsep=0pt}
\allowdisplaybreaks

\newcommand{\R}{\mathbb R}
\newcommand{\proj}{\operatorname{proj}}

\newmdenv[
  linewidth=1pt,
  linecolor=black,
  topline=true,
  bottomline=true,
  leftline=true,
  rightline=true,
  innertopmargin=10pt,
  innerbottommargin=10pt,
  innerleftmargin=10pt,
  innerrightmargin=10pt
]{answerbox}

\pagestyle{plain}


\begin{document}

\begin{center}
  {\bf MATH 231-01: Homework Assignment 3}\\~\\
  22 September 2025\\~\\~\\~\\~\\~\\
\end{center}

{\bf Due:} 29 September 2025 by 10:00pm Eastern time, submitted on Moodle as a single PDF.~\\


{\bf Instructions:} Write your solutions on the following pages. If you need more space, you may add pages, but make sure they are in order and label the problem number(s) clearly. You should attempt each problem on scrap paper first, before writing your solution here. Excessively messy or illegible work will not be graded. You must show your work/reasoning to receive credit. You do not need to include every minute detail; however the process by which you reached your answer should be evident. You may work with other students, but please write your solutions in your own words.

~\\~\\~\\~\\~\\
{\bf Name:} Sean Balbale

~\\
{\bf Score:}

\newpage
\begin{itemize}
  \item[1.] The angle between two curves at a point of intersection is defined to be the smallest angle between their tangent lines at that point. Let $C_1$ and $C_2$ be the curves parametrized by ${\bf r_1}(t) = \langle \cos(t),\sin(2t),t\rangle$ and ${\bf r_2}(t) = \langle \sin(2t), \cos(t), t\rangle$ for $t \in [0,\pi]$.
    \newline

    \begin{itemize}
      \item[(a)] Find all points of intersection between $C_1$ and $C_2$. (The double-angle formula $\sin(2t) = 2\sin(t)\cos(t)$ may be helpful.)
        \newline

        \begin{answerbox}
          The double angle formula is as follows:
          \[
            \sin(2t) = 2\sin(t)\cos(t)
          \]
          To find the points of intersection between \( C_1 \) and \( C_2
          \), we set their parametric equations equal to each other:
          \[
            \begin{aligned}
              \cos(t) &= \sin(2t) \\
              \sin(2t) &= \cos(t) \\
              t &= t
            \end{aligned}
          \]
          From the third equation, we see that \( t \) is the same for both curves
          at the points of intersection. Now, we can use the double-angle formula to rewrite the first equation:
          \[
            \cos(t) = 2\sin(t)\cos(t)
          \]
          Rearranging this gives:
          \[
            \cos(t) - 2\sin(t)\cos(t) = 0 \
            \cos(t)(1 - 2\sin(t)) = 0
          \]
          This equation holds true if either \( \cos(t) = 0 \) or \( 1 - 2\sin(t) = 0 \).
          \begin{itemize}
            \item If \( \cos(t) = 0 \), then \( t = \frac{\pi}{2} \) (within the interval \( [0, \pi] \)).
            \item If \( 1 - 2\sin(t) = 0 \),
            \item then \( \sin(t) = \frac{1}{2} \), which gives \( t = \frac{\pi}{6} \) or \( t = \frac{5\pi}{6} \) (within the interval \( [0, \pi] \)).
            \item Thus, the points of intersection occur at \( t = \frac{\pi}{6}, \frac{\pi}{2}, \frac{5\pi}{6} \).
          \end{itemize}
          Now, we can find the corresponding points on the curves:
          \begin{itemize}
            \item For \( t = \frac{\pi}{6} \):
              \[
                C_1\left(\frac{\pi}{6}\right) = \left\langle\cos\left(\frac{\pi}{6}\right), \sin\left(\frac{\pi}{3}\right), \frac{\pi}{6}\right\rangle = \left\langle\frac{\sqrt{3}}{2}, \frac{\sqrt{3}}{2}, \frac{\pi}{6}\right\rangle
              \]
            \item For \( t = \frac{\pi}{2} \):
              \[
                C_1\left(\frac{\pi}{2}\right) = \left\langle\cos\left(\frac{\pi}{2}\right), \sin(\pi), \frac{\pi}{2}\right\rangle = \left\langle 0, 0, \frac{\pi}{2}\right\rangle
              \]
            \item For \( t = \frac{5\pi}{6} \):
              \[
                C_1\left(\frac{5\pi}{6}\right) = \left\langle\cos\left(\frac{5\pi}{6}\right), \sin\left(\frac{5\pi}{3}\right), \frac{5\pi}{6}\right\rangle = \left\langle-\frac{\sqrt{3}}{2}, -\frac{\sqrt{3}}{2}, \frac{5\pi}{6}\right\rangle
              \]
          \end{itemize}
          Therefore, the points of intersection between \( C_1 \) and \( C_2 \) are:
          \[
            \left\langle\frac{\sqrt{3}}{2}, \frac{\sqrt{3}}{2}, \frac{\pi}{6}\right\rangle, \quad \left\langle 0, 0, \frac{\pi}{2}\right\rangle, \quad \left\langle-\frac{\sqrt{3}}{2}, -\frac{\sqrt{3}}{2}, \frac{5\pi}{6}\right\rangle
          \]


        \end{answerbox}
        \vspace{0.5 in}
      \item[(b)] Find the angle between $C_1$ and $C_2$ at each point of intersection. If you can't write down the exact value, round your answer to the nearest hundredth of a radian.
        \newline

        \begin{answerbox}
          To find the angle between the curves \( C_1 \) and \( C_2 \) at their points of intersection, we first need to compute the tangent vectors of each curve at those points. The tangent vector is given by the derivative of the position vector with respect to \( t \).

          The derivatives of the position vectors are:
          \[
            C_1'(t) = \langle -\sin(t), 2\cos(2t), 1 \rangle
          \]
          \[
            C_2'(t) = \langle 2\cos(2t), -\sin(t), 1 \rangle
          \]

          We will evaluate these derivatives at each point of intersection found in part (a).
          \begin{enumerate}
            \item At \( t = \frac{\pi}{6} \)
              \[
                C_1'\left(\frac{\pi}{6}\right) = \left\langle -\sin\left(\frac{\pi}{6}\right), 2\cos\left(\frac{\pi}{3}\right), 1 \right\rangle = \left\langle -\frac{1}{2}, 1, 1 \right\rangle
              \]
              \[
                C_2'\left(\frac{\pi}{6}\right) = \left\langle 2\cos\left(\frac{\pi}{3}\right), -\sin\left(\frac{\pi}{6}\right), 1 \right\rangle = \left\langle 1, -\frac{1}{2}, 1 \right\rangle
              \]
              The angle \( \theta \) between the two tangent vectors can be found using the dot product formula:
              \[
                \cos(\theta) = \frac{C_1'\left(\frac{\pi}{6}\right) \cdot C_2'\left(\frac{\pi}{6}\right)}{\|C_1'\left(\frac{\pi}{6}\right)\| \|C_2'\left(\frac{\pi}{6}\right)\|}
              \]
              Calculating the dot product:
              \[
                C_1'\left(\frac{\pi}{6}\right) \cdot C_2'\left(\frac{\pi}{6}\right) = \left(-\frac{1}{2}\right)(1) + (1)\left(-\frac{1}{2}\right) + (1)(1) = -\frac{1}{2} - \frac{1}{2} + 1 = 0
              \]
              The magnitudes of the tangent vectors are:
              \[
                \|C_1'\left(\frac{\pi}{6}\right)\| = \sqrt{\left(-\frac{1}{2}\right)^2 + 1^2 + 1^2} = \sqrt{\frac{1}{4} + 1 +   1} = \sqrt{\frac{9}{4}} = \frac{3}{2}
              \]
              \[
                \|C_2'\left(\frac{\pi}{6}\right)\| = \sqrt{1^2 + \left(-\frac{1}{2}\right)^2 + 1^2} = \sqrt{1 + \frac{1}{4} + 1} = \sqrt{\frac{9}{4}} = \frac{3}{2}
              \]
              Therefore,
              \[
                \cos(\theta) = \frac{0}{\left(\frac{3}{2}\right)\left(\frac{3}{2}\right)} = 0
              \]
              Thus, \( \theta = \frac{\pi}{2} \) radians.
            \item At \( t = \frac{\pi}{2} \)
              \[
                C_1'\left(\frac{\pi}{2}\right) = \left\langle -\sin\left(\frac{\pi}{2}\right), 2\cos\left(\pi\right), 1 \right\rangle = \left\langle -1, -2, 1 \right\rangle
              \]
              \[
                C_2'\left(\frac{\pi}{2}\right) = \left\langle 2\cos\left(\pi\right), -\sin\left(\frac{\pi}{2}\right), 1 \right\rangle = \left\langle -2, -1, 1 \right\rangle
              \]
              The dot product is:
              \[
                C_1'\left(\frac{\pi}{2}\right) \cdot C_2'\left(\frac{\pi}{2}\right) = (-1)(-2) + (-2)(-1) + (1)(1) = 2 + 2 + 1 = 5
              \]
              The magnitudes are:
              \[
                \|C_1'\left(\frac{\pi}{2}\right)\| = \sqrt{(-1)^2 + (-2)^2 + 1^2} = \sqrt{1 + 4 + 1} = \sqrt{6}
              \]
              \[
                \|C_2'\left(\frac{\pi}{2}\right)\| = \sqrt{(-2)^2 + (-1)^2 + 1^2} = \sqrt{4 + 1 + 1} = \sqrt{6}
              \]
              Therefore,
              \[
                \cos(\theta) = \frac{5}{\sqrt{6} \cdot \sqrt{6}} = \frac{5}{6}
              \]
              Thus, \( \theta = \cos^{-1}\left(\frac{5}{6}\right) \approx 0.59 \) radians.

            \item At \( t = \frac{5\pi}{6} \)
              \[
                C_1'\left(\frac{5\pi}{6}\right) = \left\langle -\sin\left(\frac{5\pi}{6}\right), 2\cos\left(\frac{5\pi}{3}\right), 1 \right\rangle = \left\langle -\frac{1}{2}, 1, 1 \right\rangle
              \]
              \[
                C_2'\left(\frac{5\pi}{6}\right) = \left\langle 2\cos\left(\frac{5\pi}{3}\right), -\sin\left(\frac{5\pi}{6}\right), 1 \right\rangle = \left\langle 1, -\frac{1}{2}, 1 \right\rangle
              \]
              Since this is the same as the case for \( t = \frac{\pi}{6} \), we have:
              \[
                \cos(\theta) = 0
              \]
              Thus, \( \theta = \frac{\pi}{2} \) radians.
          \end{enumerate}
        \end{answerbox}
        \vspace{0.5 in}
    \end{itemize}

  \item[2.] Prove the derivative rule for the dot product. In other words, given arbitrary vector functions ${\bf r_1}(t) = \langle x_1(t), y_1(t), z_1(t)\rangle$ and ${\bf r_2}(t) = \langle x_2(t), y_2(t), z_2(t)\rangle$, show that
    \begin{align*}
      ({\bf r_1}(t)\cdot{\bf r_2}(t))' = {\bf r_1}'(t)\cdot{\bf r_2}(t) + {\bf r_1}(t)\cdot{\bf r_2}'(t).
    \end{align*}
    \newline

    \begin{answerbox}
      \begin{enumerate}
        \item Lets begin with the definition of the dot product:
          \[
            {\bf r_1}(t) \cdot {\bf r_2}(t) = x_1(t)x_2(t) + y_1(t)y_2(t) + z_1(t)z_2(t)
          \]
        \item Now, we will differentiate both sides with respect to \( t \):
          \[
            \frac{d}{dt}({\bf r_1}(t) \cdot {\bf r_2}(t)) = \frac{d}{dt}(x_1(t)x_2(t) + y_1(t)y_2(t) + z_1(t)z_2(t))
          \]
        \item Applying the product rule to each term on the right-hand side, we get:
          \begin{align*}
            \frac{d}{dt}({\bf r_1}(t) \cdot {\bf r_2}(t)) &= x_1'(t)x_2(t) + x_1(t)x_2'(t) + y_1'(t)y_2(t) \\
            &\quad + y_1(t)y_2'(t) + z_1'(t)z_2(t) + z_1(t)z_2'(t)
          \end{align*}
        \item We can group the terms to see that this is equivalent to:
          \begin{align*}
            \frac{d}{dt}({\bf r_1}(t) \cdot {\bf r_2}(t)) &= (x_1'(t), y_1'(t), z_1'(t)) \cdot (x_2(t), y_2(t), z_2(t)) \\
            &\quad + (x_1(t), y_1(t), z_1(t)) \cdot (x_2'(t), y_2'(t), z_2'(t))
          \end{align*}
        \item Recognizing the derivatives as the vectors \( {\bf r_1}'(t) \) and \( {\bf r_2}'(t) \), we can rewrite this as:
          \begin{align*}
            \frac{d}{dt}({\bf r_1}(t) \cdot {\bf r_2}(t)) &= {\bf r_1}'(t) \cdot {\bf r_2}(t) + {\bf r_1}(t) \cdot {\bf r_2}'(t)
          \end{align*}
        \item Thus, we have shown that:
          \[
            ({\bf r_1}(t) \cdot {\bf r_2}(t))' = {\bf r_1}'(t) \cdot {\bf r_2}(t) + {\bf r_1}(t) \cdot {\bf r_2}'(t)
          \]
      \end{enumerate}
    \end{answerbox}
    \vspace{0.5 in}

  \item[3.] Let ${\bf r_1}(t) = \langle e^{2t}, t^2, \sin(3t)\rangle$ and ${\bf r_2}(t) = \langle \ln(1+t^2), \cos(2t), t^3\rangle$. Let $C$ be the curve parametrized by ${\bf r}(t) = {\bf r_1}(t) \times {\bf r_2}(t)$. Find an equation for the line tangent to $C$ at the point ${\bf r}(0)$.
    \newline

    \begin{answerbox}
      A tangent line needs a point and a direction vector. Calculating the point \( {\bf r}(0) \):
      \[
        {\bf r_1}(0) = \langle e^{0}, 0^2, \sin(0) \rangle = \langle 1, 0, 0 \rangle
      \]
      \[
        {\bf r_2}(0) = \langle \ln(1+0^2), \cos(0), 0^3 \rangle = \langle 0, 1, 0 \rangle
      \]
      Now, we compute the cross product \( {\bf r}(0) = {\bf r_1}(0) \times {\bf r_2}(0) \):
      \[
        {\bf r}(0) =
        \begin{vmatrix}
          \mathbf{i} & \mathbf{j} & \mathbf{k} \\
          1 & 0 & 0 \\
          0 & 1 & 0
        \end{vmatrix} = \mathbf{i}(0 - 0) - \mathbf{j}(0 - 0) + \mathbf{k}(1 - 0) = \langle 0, 0, 1 \rangle
      \]
      Next, we need to find the derivative \( {\bf r}'(t) \) to get the direction vector of the tangent line. Using the product rule for cross products:
      \[
        {\bf r}'(t) = {\bf r_1}'(t) \times {\bf r_2}(t) + {\bf r_1}(t) \times {\bf r_2}'(t)
      \]
      First, we compute the derivatives:
      \[
        {\bf r_1}'(t) = \langle 2e^{2t}, 2t, 3\cos(3t) \rangle
      \]
      \[
        {\bf r_2}'(t) = \langle \frac{2t}{1+t^2}, -2\sin(2t), 3t^2 \rangle
      \]
      Now, we evaluate these at \( t = 0 \):
      \[
        {\bf r_1}'(0) = \langle 2, 0, 3 \rangle
      \]
      \[
        {\bf r_2}'(0) = \langle 0, 0, 0 \rangle
      \]
      Now, we compute the cross products:
      \[
        {\bf r_1}'(0) \times {\bf r_2}(0) =
        \begin{vmatrix}
          \mathbf{i} & \mathbf{j} & \mathbf{k} \\
          2 & 0 & 3 \\
          0 & 1 & 0
        \end{vmatrix} = \mathbf{i}(0 - 3) - \mathbf{j}(0 - 0) + \mathbf{k}(2 - 0) = \langle -3, 0, 2 \rangle
      \]
      \[
        {\bf r_1}(0) \times {\bf r_2}'(0) =
        \begin{vmatrix}
          \mathbf{i} & \mathbf{j} & \mathbf{k} \\
          1 & 0 & 0 \\
          0 & 0 & 0
        \end{vmatrix} = \mathbf{i}(0 - 0) - \mathbf{j}(0 - 0) + \mathbf{k}(0 - 0) = \langle 0, 0, 0 \rangle
      \]
      Therefore,
      \[
        {\bf r}'(0) = \langle -3, 0, 2 \rangle + \langle 0, 0, 0 \rangle = \langle -3, 0, 2 \rangle
      \]
      The tangent line at the point \( {\bf r}(0) = \langle 0, 0, 1 \rangle \) with direction vector \( {\bf r}'(0) = \langle -3, 0, 2 \rangle \) can be parametrized as:
      \[
        {\bf L}(t) = \langle 0, 0, 1 \rangle + t \langle -3, 0, 2 \rangle = \langle -3t, 0, 1 + 2t \rangle
      \]

    \end{answerbox}
    \vspace{0.5 in}

  \item[4.] The acceleration, initial velocity, and initial position of an object traveling through space are given by
    \begin{align*}
      {\bf r}''(t) = \langle -6,2,4\rangle, \quad\quad {\bf r}'(0) = \langle -5,1,3\rangle, \quad\quad\text{and}\quad\quad {\bf r}(0) = \langle 6,-2,1\rangle,
    \end{align*}
    respectively. Find the object's position function ${\bf r}(t)$.
    \newline

    \begin{answerbox}
      To find the position function \( {\bf r}(t) \), we need to integrate the acceleration function \( {\bf r}''(t) \) twice and use the initial conditions.

      First, we integrate the acceleration to find the velocity:
      \[
        {\bf r}'(t) = \int {\bf r}''(t) \, dt = \int \langle -6, 2, 4 \rangle \, dt = \langle -6t + C_1, 2t + C_2, 4t + C_3 \rangle
      \]
      Using the initial condition \( {\bf r}'(0) = \langle -5, 1, 3 \rangle \), we find:
      \[
        {\bf r}'(0) = \langle C_1, C_2, C_3 \rangle = \langle -5, 1, 3 \rangle \implies C_1 = -5, C_2 = 1, C_3 = 3
      \]
      Thus,
      \[
        {\bf r}'(t) = \langle -6t - 5, 2t + 1, 4t + 3 \rangle
      \]

      Next, we integrate the velocity to find the position:
      \begin{align*}
        {\bf r}'(t) &= \langle -6t - 5, 2t + 1, 4t + 3 \rangle \\
        {\bf r}(t) &= \int {\bf r}'(t) \, dt = \int \langle -6t - 5, 2t + 1, 4t + 3 \rangle \, dt \\
        &= \langle -3t^2 - 5t + D_1, t^2 + t + D_2, 2t^2 + 3t + D_3 \rangle
      \end{align*}
      Using the initial condition \( {\bf r}(0) = \langle 6, -2, 1 \rangle \), we find:
      \[
        {\bf r}(0) = \langle D_1, D_2, D_3 \rangle = \langle 6, -2, 1 \rangle \implies D_1 = 6, D_2 = -2, D_3 = 1
      \]
      Thus,
      \[
        {\bf r}(t) = \langle -3t^2 - 5t + 6, t^2 + t - 2, 2t^2 + 3t + 1 \rangle
      \]
    \end{answerbox}
    \vspace{0.5 in}

  \item[5.] Find the arc length of the curve parametrized by ${\bf r}(t) = \langle t-\sin(t), \cos(t)\rangle$ for $t \in [0,2\pi]$. (The half-angle formula $\sqrt{2}\sin(t/2) = \sqrt{1-\cos(t)}$ for $t \in [0,2\pi]$ may be helpful.)
    \newline

    \begin{answerbox}
      To find the arc length, we need to compute the integral of the magnitude of the derivative of the position vector \( {\bf r}(t) \) over the interval \( [0, 2\pi] \).
      First, we compute the derivative \( {\bf r}'(t) \):
      \[
        {\bf r}'(t) = \langle 1 - \cos(t), -\sin(t) \rangle
      \]
      Next, we find the magnitude of \( {\bf r}'(t) \):
      \[
        \|{\bf r}'(t)\| = \sqrt{(1 - \cos(t))^2 + (-\sin(t))^2} = \sqrt{1 - 2\cos(t) + \cos^2(t) + \sin^2(t)}
      \]
      Since \( \cos^2(t) + \sin^2(t) = 1 \), we have:
      \[
        \|{\bf r}'(t)\| = \sqrt{2 - 2\cos(t)} = \sqrt{2(1 - \cos(t))} = \sqrt{2} \sqrt{1 - \cos(t)}
      \]
      Using the half-angle formula \( \sqrt{2}\sin(t/2) = \sqrt{1 - \cos(t)} \), we get:
      \[
        \|{\bf r}'(t)\| = \sqrt{2} \cdot \sqrt{2} \sin(t/2) = 2\sin(t/2)
      \]
      Now, we can set up the arc length integral:
      \[
        L = \int_0^{2\pi} \|{\bf r}'(t)\| \, dt = \int_0^{2\pi} 2\sin(t/2) \, dt
      \]
      Using the substitution \( u = t/2 \), which gives \( dt = 2du \). The limits of integration change accordingly: when \( t = 0 \), \( u = 0 \); and when \( t = 2\pi \), \( u = \pi \). Thus, the integral becomes:
      \[
        L = \int_0^{\pi} 2\sin(u) \cdot 2 \, du = 4 \int_0^{\pi} \sin(u) \, du
      \]
      Evaluating the integral:
      \[
        4 \left[-\cos(u)\right]_0^{\pi} = 4 \left[-\cos(\pi) + \cos(0)\right] = 4 \left[-(-1) + 1\right] = 4 \cdot 2 = 8
      \]
      Therefore, the arc length of the curve is:
      \[
        L = 8
      \]
    \end{answerbox}
    \vspace{0.5 in}

  \item[6.] Let $C$ be the \emph{moment curve}, parametrized by ${\bf r}(t) = \langle t, t^2, t^3 \rangle$.
    \newline

    \begin{itemize}
      \item[(a)] Find the curvature of $C$.
        \newline
        \begin{answerbox}
          To find the curvature \( \kappa(t) \) of the curve \( C \) parametrized by \( {\bf r}(t) = \langle t, t^2, t^3 \rangle \), we use the formula:
          \[
            \kappa(t) = \frac{\|{\bf r}'(t) \times {\bf r}''(t)\|}{\|{\bf r}'(t)\|^3}
          \]
          First, we compute the first and second derivatives of \( {\bf r}(t) \):
          \[
            {\bf r}'(t) = \langle 1, 2t, 3t^2 \rangle
          \]
          \[
            {\bf r}''(t) = \langle 0, 2, 6t \rangle
          \]
          Next, we compute the cross product \( {\bf r}'(t) \times {\bf r}''(t) \):
          \[
            {\bf r}'(t) \times {\bf r}''(t) =
            \begin{vmatrix}
              \mathbf{i} & \mathbf{j} & \mathbf{k} \\
              1 & 2t & 3t^2 \\
              0 & 2 & 6t
            \end{vmatrix} = \mathbf{i}(2t \cdot 6t - 3t^2 \cdot 2) - \mathbf{j}(1 \cdot 6t - 3t^2 \cdot 0) + \mathbf{k}(1 \cdot 2 - 2t \cdot 0)
          \]
          Simplifying this, we get:
          \[
            {\bf r}'(t) \times {\bf r}''(t) = \langle 12t^2 - 6t^2, -6t, 2 \rangle = \langle 6t^2, -6t, 2 \rangle
          \]
          Now, we find the magnitude of this cross product:
          \[
            \|{\bf r}'(t) \times {\bf r}''(t)\| = \sqrt{(6t^2)^2 + (-6t)^2 + 2^2} = \sqrt{36t^4 + 36t^2 + 4}
          \]
          Next, we compute the magnitude of \( {\bf r}'(t) \):
          \[
            \|{\bf r}'(t)\| = \sqrt{1^2 + (2t)^2 + (3t^2)^2} = \sqrt{1 + 4t^2 + 9t^4} = \sqrt{9t^4 + 4t^2 + 1}
          \]
          Now, we can substitute these into the curvature formula:
          \[
            \kappa(t) = \frac{\sqrt{36t^4 + 36t^2 + 4}}{(9t^4 + 4t^2 + 1)^{3/2}}
          \]
          Therefore, the curvature of the moment curve \( C \) is:
          \[
            \kappa(t) = \frac{\sqrt{36t^4 + 36t^2 + 4}}{(9t^4 + 4t^2 + 1)^{3/2}}
          \]

        \end{answerbox}

        \vspace{0.5 in}
      \item[(b)] Does the curvature increase or decrease as $t \rightarrow \infty$? (No need to be rigorous here; an educated guess is fine!)
        \newline

        \begin{answerbox}
          As \( t \rightarrow \infty \), the terms \( 36t^4 \) in the numerator and \( 9t^4 \) in the denominator dominate the behavior of the curvature formula. Simplifying the expression for large \( t \):
          \[
            \kappa(t) \approx \frac{\sqrt{36t^4}}{(9t^4)^{3/2}} = \frac{6t^2}{(9t^6)^{1/2}} = \frac{6t^2}{3t^3} = \frac{2}{t}
          \]
          As \( t \) increases, \( \frac{2}{t} \) decreases. Therefore, the curvature \( \kappa(t) \) decreases to zero as \( t \rightarrow \infty \).
        \end{answerbox}
        \vspace{0.5 in}
    \end{itemize}
    \vspace{0.5 in}
    \newpage
  \item[7.] Complete Problem 10 on p.~889 of the textbook (Section 11.4)\footnote{The names in this problem are purely coincidental and do not refer to individuals in our class!}
    Sam and Ben are arguing about the curvature of a function $y=f(x)$. Sam claims that the maximum curvature of every such function occurs at a relative extremum. Ben disagrees. Who is right? If Sam is right, prove the result. If Ben is right, provide a counterexample that disproves Sam's claim.
    \newline

    \begin{answerbox}
      Ben is right. A counterexample can be provided by considering the function \( f(x) = x^3 \). The curvature \( \kappa \) of a function \( y = f(x) \) is given by:
      \[
        \kappa = \frac{f''(x)}{(1 + (f'(x))^2)^{3/2}}
      \]
      For \( f(x) = x^3 \), we have:
      \[
        f'(x) = 3x^2, \quad f''(x) = 6x
      \]
      The curvature becomes:
      \[
        \kappa = \frac{6x}{(1 + (3x^2)^2)^{3/2}} = \frac{6x}{(1 + 9x^4)^{3/2}}
      \]
      The maximum curvature occurs at \( x = 0 \), which is a relative extremum. However, if we consider the function \( f(x) = \sin(x) \), the curvature can be shown to have maximum values at points that are not relative extrema (e.g., at \( x = \frac{\pi}{2} \)).
      Therefore, Sam's claim is disproven by this counterexample.


    \end{answerbox}
\end{itemize}
\end{document}
