\documentclass[reqno, 12pt]{amsart}

\usepackage{amssymb, amsmath, amsthm, enumerate, mathtools, graphicx, enumitem, tikz, color, soul, bbm, verbatim, parskip, multicol}
\usepackage[margin=1 in]{geometry}
\usepackage[mathscr]{euscript}
\usepackage[urlcolor=blue,colorlinks=true]{hyperref}
\setlist[itemize]{noitemsep, topsep=0pt, parsep=0pt, partopsep=0pt}
\allowdisplaybreaks

\newcommand{\R}{\mathbb R}
\newcommand{\proj}{\operatorname{proj}}

\pagestyle{plain}  


\begin{document}

\begin{center}
    {\bf MATH 231-01: Homework Assignment 2}\\~\\
    15 September 2025\\~\\~\\~\\~\\~\\
\end{center}

{\bf Due:} 22 September 2025 by 10:00pm Eastern time, submitted on Moodle as a single PDF.~\\


{\bf Instructions:} Write your solutions on the following pages. If you need more space, you may add pages, but make sure they are in order and label the problem number(s) clearly. You should attempt each problem on scrap paper first, before writing your solution here. Excessively messy or illegible work will not be graded. You must show your work/reasoning to receive credit. You do not need to include every minute detail; however the process by which you reached your answer should be evident. You may work with other students, but please write your solutions in your own words.

~\\~\\~\\~\\~\\
{\bf Name:} Sean Balbale

~\\
{\bf Score:}

\newpage
\begin{itemize}
    \item[1.] Use the cross product to find the area of the triangle with vertices $A = (3,0,0)$, $B = (3,3,0)$, and $C = (0,1,3)$.
          \newline


          \noindent\fbox{%
              \parbox{\dimexpr\linewidth-2\fboxsep-2\fboxrule}{
                  Let $\vec{AB} = B - A = <0,3,0>$ and $\vec{AC} = C - A = <-3,1,3>$. Then the area of the triangle is given by
                  \[
                      \text{Area} = \frac{1}{2} \| \vec{AB} \times \vec{AC} \|.
                  \]
                  Calculating the cross product, we have
                  \begin{align*}
                      \vec{AB} \times \vec{AC} = \begin{vmatrix}
                                                     \vec{i} & \vec{j} & \vec{k} \\
                                                     0       & 3       & 0       \\
                                                     -3      & 1       & 3
                                                 \end{vmatrix} & = (3 \cdot 3 - 0 \cdot 1)\vec{i} - (0 \cdot 3 - 0 \cdot -3)\vec{j} + (0 \cdot 1 - 3 \cdot -3)\vec{k} \\
                                                                 & = (9\vec{i} + 0\vec{j} + 9\vec{k})                                                                 \\
                                                                 & = <9,0,9>.
                  \end{align*}

                  Thus,
                  \[
                      \text{Area} = \frac{1}{2} \sqrt{9^2 + 0^2 + 9^2} = \frac{1}{2} \sqrt{162} = \frac{9\sqrt{2}}{2}.
                  \]
              }%
          }

          \vspace{0.5 in}

    \item[2.] Suppose ${\bf u}$, ${\bf v}$, and ${\bf w}$ are unit vectors in $\R^3$ such that ${\bf v}$ and ${\bf w}$ make an angle of $\pi/6$, and ${\bf u}$ and ${\bf v}\times{\bf w}$ make an angle of $2\pi/3$ (both in radians). Find the volume of the parallelepiped generated by ${\bf u}$, ${\bf v}$, and ${\bf w}$.
          \newline

          \noindent\fbox{%
              \parbox{\dimexpr\linewidth-2\fboxsep-2\fboxrule}{

                  The volume of the parallelepiped generated by vectors $\vec{u}$, $\vec{v}$, and $\vec{w}$ is given by the scalar triple product:
                  \[
                      \text{Volume} = |\vec{u} \cdot (\vec{v} \times \vec{w})|.
                  \]
                  We know that $\|\vec{v}\| = 1$ and $\|\vec{w}\| = 1$, and the angle between them is $\pi/6$. Thus, we can find the magnitude of their cross product:
                  \[
                      \|\vec{v} \times \vec{w}\| = \|\vec{v}\| \|\vec{w}\| \sin\left(\frac{\pi}{6}\right) = 1 \cdot 1 \cdot \frac{1}{2} = \frac{1}{2}.
                  \]
                  Next, we need to find the angle between $\vec{u}$ and $\vec{v} \times \vec{w}$. We know that $\|\vec{u}\| = 1$ and the angle between them is $2\pi/3$. Thus,
                  \[
                      |\vec{u} \cdot (\vec{v} \times \vec{w})| = \|\vec{u}\| \|\vec{v} \times \vec{w}\| \cos\left(\frac{2\pi}{3}\right) = 1 \cdot \frac{1}{2} \cdot \left(-\frac{1}{2}\right) = -\frac{1}{4}.
                  \]
                  Therefore, the volume of the parallelepiped is
                  \[
                      \text{Volume} = \left| -\frac{1}{4} \right| = \frac{1}{4}.
                  \]
              }%
          }
          \vspace{0.5 in}

          \newpage
    \item[3.] Solve Problem 60 on p.~825 of the textbook (Section 10.4).

          \vspace{0.5 in}
    \item[4.] Prove that the cross product is anticommutative. In other words, given arbitrary vectors ${\bf u} = \langle u_1,u_2,u_3\rangle$ and ${\bf v} = \langle v_1,v_2,v_3\rangle$, use the definition of the cross product to show that ${\bf v} \times {\bf u} = -({\bf u} \times {\bf v})$.
          \newline

          \noindent\fbox{%
              \parbox{\dimexpr\linewidth-2\fboxsep-2\fboxrule}{
                  Starting with the definition of the cross product:
                  \[
                      \vec{u} \times \vec{v} = \begin{vmatrix}
                          \vec{i} & \vec{j} & \vec{k} \\
                          u_1     & u_2     & u_3     \\
                          v_1     & v_2     & v_3
                      \end{vmatrix}.
                  \]
                  Calculating this determinant,
                  \begin{align*}
                      \vec{u} \times \vec{v} & = (u_2 v_3 - u_3 v_2)\vec{i} - (u_1 v_3 - u_3 v_1)\vec{j} + (u_1 v_2 - u_2 v_1)\vec{k} \\
                                             & = (u_2 v_3 - u_3 v_2, -(u_1 v_3 - u_3 v_1), u_1 v_2 - u_2 v_1).
                  \end{align*}
                  Now, computing $\vec{v} \times \vec{u}$:
                  \[
                      \vec{v} \times \vec{u} = \begin{vmatrix}
                          \vec{i} & \vec{j} & \vec{k} \\
                          v_1     & v_2     & v_3     \\
                          u_1     & u_2     & u_3
                      \end{vmatrix}.
                  \]
                  Calculating this determinant,
                  \begin{align*}
                      \vec{v} \times \vec{u} & = (v_2 u_3 - v_3 u_2)\vec{i} - (v_1 u_3 - v_3 u_1)\vec{j} + (v_1 u_2 - v_2 u_1)\vec{k} \\
                                             & = (v_2 u_3 - v_3 u_2, -(v_1 u_3 - v_3 u_1), v_1 u_2 - v_2 u_1).
                  \end{align*}
                  Notice that each component of $\vec{v} \times \vec{u}$ is the negative of the corresponding component of $\vec{u} \times \vec{v}$. Specifically,
                  \[
                      v_2 u_3 - v_3 u_2 = -(u_2 v_3 - u_3 v_2),
                  \]
                  \[
                      -(v_1 u_3 - v_3 u_1) = -(-(u_1 v_3 - u_3 v_1)) = u_1 v_3 - u_3 v_1,
                  \]
                  \[            v_1 u_2 - v_2 u_1 = -(u_1 v_2 - u_2 v_1).
                  \]
                  Therefore,
                  \[
                      \vec{v} \times \vec{u} = -(\vec{u} \times \vec{v}).
                  \]

              }%
          }
          \vspace{0.5 in}
          \newpage
    \item[5.] For many years, I've been trying to prove that $0 = 1$. Here's my latest attempt:\\

          Let ${\bf i}$, ${\bf j}$, and ${\bf k}$ denote the standard basis vectors in $\R^3$. Then
          \begin{align*}
              {\bf 0} = {\bf i} \times {\bf i} = ({\bf i} \times {\bf i}) \times {\bf j} = {\bf i} \times ({\bf i} \times {\bf j}) = {\bf i} \times {\bf k} = -{\bf j}.
          \end{align*}
          Since $\|{\bf 0}\| = 0$ and $\|-{\bf j}\| = 1$, it follows that $0 = 1$. \qed\\

          Where did I go wrong? Explain.
          \newline

          \noindent\fbox{%
              \parbox{\dimexpr\linewidth-2\fboxsep-2\fboxrule}{
                The mistake was made during step 3, where the vector triple product identity was not applied correctly. The correct identity states that for any vectors $\vec{a}$, $\vec{b}$, and $\vec{c}$,
                \[
                    \vec{a} \times (\vec{b} \times \vec{c}) = (\vec{a} \cdot \vec{c})\vec{b} - (\vec{a} \cdot \vec{b})\vec{c}.
                \]
                In this case, applying the identity correctly gives:
                \[
                    \vec{i} \times (\vec{i} \times \vec{j}) = (\vec{i} \cdot \vec{j})\vec{i} - (\vec{i} \cdot \vec{i})\vec{j} = 0 \cdot \vec{i} - 1 \cdot \vec{j} = -\vec{j}.
                \]
                Thus, the step is valid, but the conclusion that $\vec{0} = -\vec{j}$ is incorrect because it contradicts the properties of vectors. The zero vector cannot equal a non-zero vector. Therefore, the error lies in the logical conclusion drawn from the calculations, not in the calculations themselves.


              }%
          }
          \vspace{0.5 in}

    \item[6.] Let $L_1$ be the line with vector equation
          \begin{align*}
              \langle x,y,z\rangle = \langle 1,0,1\rangle + t\langle 2,-1,0\rangle.
          \end{align*}
          Let $L_2$ be the line with symmetric equations
          \begin{align*}
              \frac{x+1}{3} = \frac{1-y}{2} = z-1.
          \end{align*}
          \begin{enumerate}
              \item[(a)]{Show that $L_1$ and $L_2$ intersect.}

                    \newpage
              \item[(b)]{Find an equation for the plane that contains $L_1$ and $L_2$.}
          \end{enumerate}

          \vspace{4 in}
    \item[7.] Find an equation for the line that contains the point $P = (2,0,1)$ and is parallel to to the planes given by the equations $x+2y+3z = 6$ and $2x-y+z = 4$.

          \newpage
    \item[8.] Let $A = (-1,0,2)$ and $B = (0,3,0)$. The collection of all points $P = (x,y,z)$ with equal distance to $A$ and $B$ forms a plane. Find an equation of the form
          \begin{align*}
              a(x-x_0)+b(y-y_0) + c(z-z_0) = 0
          \end{align*}
          that represents this plane.

\end{itemize}

\end{document}







