\documentclass[12pt]{article}
\usepackage{amsmath, amssymb}
\usepackage{enumitem}
\usepackage{tikz}
\usepackage{geometry}
\usepackage{pgfplots}
\pgfplotsset{compat=1.18}
\geometry{margin=1in}
\setlength{\parindent}{0pt}

\title{Economics 101: Basic Economic Principles\\Problem Set \#4 Solutions}
\author{Sean Balbale}
\date{\today}

\begin{document}
\maketitle

\section*{Part I: Multiple Choice}

\subsection*{Problem 1}
\textbf{Problem:} Price is constant or “given” to the individual firm selling in a purely competitive market because:
\begin{enumerate}[label=(\Alph*)]
  \item The firm's demand curve is downward-sloping.
  \item There are no good substitutes for the firm's product.
  \item Product differentiation is reinforced by extensive advertising.
  \item Each seller supplies a negligible fraction of total supply.
\end{enumerate}

\textbf{Solution:} In a purely competitive market each firm is a price taker because its output is too small relative to the entire market to affect the market price. Thus, the correct answer is \textbf{(D)}.

\subsection*{Problem 2}
\textbf{Problem:} Consider the table below. At what point would a purely competitive firm cover all of its costs and earn only normal profits (i.e., zero economic profits)?
\[
\begin{array}{cccc}
\text{Price} & \text{Quantity} & \text{TFC} & \text{TVC} \\
\$5 & 5 & \$25 & \$10 \\
\$5 & 10 & \$25 & \$20 \\
\$5 & 15 & \$25 & \$50 \\
\$5 & 20 & \$25 & \$60 \\
\end{array}
\]
At output level:
\begin{enumerate}[label=(\Alph*)]
  \item 15.
  \item 10.
  \item 20.
  \item 5.
\end{enumerate}

\textbf{Solution:} At an output of 15, total revenue is \$5 $\times$ 15 = \$75 and total cost is TFC + TVC = \$25 + \$50 = \$75. This means the firm earns zero economic profit. Therefore, the answer is \textbf{(A)}.

\bigskip
\section*{Part II: Perfect Competition in the Short Run (Problem \#4)}
This problem (with parts a--f) is taken from the textbook. The firm’s cost data are given in the table below. In the table the columns represent:
\begin{itemize}[noitemsep]
  \item \textbf{AFC:} Average Fixed Cost,
  \item \textbf{AVC:} Average Variable Cost,
  \item \textbf{ATC:} Average Total Cost, and
  \item \textbf{MC:} Marginal Cost.
\end{itemize}
The fixed cost is \$60 for all output levels.
\[
\begin{array}{c|cccc}
Q & \text{AFC} & \text{AVC} & \text{ATC} & \text{MC} \\
\hline
1  & 60.00 & 45.00  & 105.00 & 45 \\
2  & 30.00 & 42.50  & 72.50  & 40 \\
3  & 20.00 & 40.00  & 60.00  & 35 \\
4  & 15.00 & 37.50  & 52.50  & 30 \\
5  & 12.00 & 37.00  & 49.00  & 35 \\
6  & 10.00 & 37.50  & 47.50  & 40 \\
7  & 8.57  & 38.57  & 47.14  & 45 \\
8  & 7.50  & 40.63  & 48.13  & 55 \\
9  & 6.67  & 43.33  & 50.00  & 65 \\
10 & 6.00  & 46.50  & 52.50  & 75 \\
\end{array}
\]

\subsection*{(a) Analysis at Product Price \$56}
\textbf{Question:} At a product price of \$56, will the firm produce in the short run? If so, what is the profit-maximizing (or loss-minimizing) output? What is the economic profit per unit?

\textbf{Solution:}
\begin{itemize}[noitemsep]
  \item In perfect competition, the firm produces where \(P = MC\) as long as \(P \geq AVC\).
  \item Here, \(P = \$56\). Checking the MC schedule, we note that:
    \begin{itemize}[noitemsep]
      \item \(MC(8) = 55\) (just below 56) and
      \item \(MC(9) = 65\) (above 56).
    \end{itemize}
  \item Thus, the firm produces \(Q=8\) units.
  \item At \(Q=8\), \(ATC = 48.13\); therefore, profit per unit \(= 56 - 48.13 \approx 7.87\).
  \item Total economic profit \( \approx 8 \times 7.87 \approx 63\) dollars.
\end{itemize}

\subsection*{(b) Analysis at Product Price \$41}
\textbf{Question:} Answer part (a) assuming product price is \$41.

\textbf{Solution:}
\begin{itemize}[noitemsep]
  \item With \(P = \$41\), we compare with the MC schedule. Notice:
    \begin{itemize}[noitemsep]
      \item \(MC(6) = 40\) (below 41) and
      \item \(MC(7) = 45\) (above 41).
    \end{itemize}
  \item Thus, the profit-maximizing output is \(Q=6\).
  \item At \(Q=6\), \(ATC = 47.50\); the per-unit loss is \(41 - 47.50 = -6.50\).
  \item Total loss \(= 6 \times (-6.50) = -39\) dollars.
  \item Since \(P\) exceeds \(AVC\) (with \(AVC(6) = 37.50\)), the firm produces despite the loss.
\end{itemize}

\subsection*{(c) Analysis at Product Price \$32}
\textbf{Question:} Answer part (a) assuming product price is \$32.

\textbf{Solution:}
\begin{itemize}[noitemsep]
  \item At \(P = \$32\), observe that the lowest \(AVC\) in the table is approximately \$37 (at \(Q=5\)).
  \item Since \(32 < 37\), \(P < AVC\) at every output.
  \item Therefore, the firm should shut down in the short run and produce \(Q=0\).
  \item The loss incurred is the fixed cost, which is \$60.
\end{itemize}

\subsection*{(d) Short-Run Supply Schedule for the Single Firm}
We now complete the supply schedule for the firm at various product prices and compute the profit or loss.

\textbf{Method:} Total cost (\(TC\)) at any \(Q\) is computed by
\[
TC = \text{ATC} \times Q,
\]
knowing that fixed cost (TFC) is \$60. (Alternatively, \(TC = TFC + (\text{AVC} \times Q)\).)

Using the results from parts (a)–(c) and the following optimal outputs:
\[
\begin{array}{c|c|c}
\textbf{Price} & Q_{\text{supply}} & \textbf{Profit per Firm} \\
\hline
\$26 & 0 & -\$60 \quad (\text{shut down}) \\
\$32 & 0 & -\$60 \quad (\text{shut down}) \\
\$38 & 5 & 5(38) - 245 = 190 - 245 = -\$55 \\
\$41 & 6 & 6(41) - 285 = 246 - 285 = -\$39 \\
\$46 & 7 & 7(46) - 330 \approx 322 - 330 = -\$8 \\
\$56 & 8 & 8(56) - 385 \approx 448 - 385 = +\$63 \\
\$66 & 9 & 9(66) - 450 = 594 - 450 = +\$144 \\
\end{array}
\]
Thus, the firm’s short-run supply schedule is as shown above.

\subsection*{(e) Industry Supply Schedule (1,500 Firms)}
If there are 1,500 identical firms, the industry supply is simply 1,500 times the output of the individual firm:
\[
\begin{array}{c|c}
\textbf{Price} & \text{Industry Quantity Supplied} \\
\hline
\$26 & 1,500 \times 0 = 0 \\
\$32 & 0 \\
\$38 & 1,500 \times 5 = 7,500 \\
\$41 & 1,500 \times 6 = 9,000 \\
\$46 & 1,500 \times 7 = 10,500 \\
\$56 & 1,500 \times 8 = 12,000 \\
\$66 & 1,500 \times 9 = 13,500 \\
\end{array}
\]

\subsection*{(f) Determination of Market Equilibrium}
The market demand schedule is given by:
\[
\begin{array}{c|c}
\textbf{Price} & \text{Total Quantity Demanded} \\
\hline
\$26 & 17,000 \\
\$32 & 15,000 \\
\$38 & 13,500 \\
\$41 & 12,000 \\
\$46 & 10,500 \\
\$56 & 9,500 \\
\$66 & 8,000 \\
\end{array}
\]
Comparing with the industry supply schedule:
\begin{itemize}[noitemsep]
  \item At \$38: Supply = 7,500; Demand = 13,500 (excess demand).
  \item At \$41: Supply = 9,000; Demand = 12,000 (excess demand).
  \item At \$46: Supply = 10,500; Demand = 10,500 (equilibrium).
  \item At higher prices, supply exceeds demand.
\end{itemize}
Thus, the equilibrium price is \$46, with total industry output of 10,500 units. Each firm produces 7 units.

At \(Q=7\) for a single firm, the average total cost is approximately \$47.14. Hence, the per-unit profit is:
\[
46 - 47.14 \approx -\$1.14,
\]
implying a small loss of about \$8 per firm. In the long run, if firms make losses, some will exit, causing the industry supply to contract and the equilibrium price to rise until only normal profits remain.

\end{document}
