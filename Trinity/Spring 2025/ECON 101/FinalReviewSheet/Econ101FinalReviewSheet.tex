\documentclass{article}
\usepackage{amsmath, amssymb}
\usepackage[margin=1in]{geometry} % Optional: for setting margins
\usepackage{enumitem} % Optional: for customizing lists

\begin{document}

\title{Final Exam Review Sheet (Chapters 1-12, 26-32, 34-36)}
\date{} % No date
\maketitle
\thispagestyle{empty} % No page number on title page

\section*{Chapter 1-2: Limits, Alternatives, and Choices; The Market System and the Circular Flow}

\subsection*{Efficiency}
\begin{itemize}
    \item \textbf{Technical efficiency:} Producing the maximum output from the minimum input.
    \item \textbf{Productive efficiency:} Producing a given output level with the minimum value of input, in the least costly way ($P = \min ATC$).
    \item \textbf{Allocative efficiency:} Producing the output level at which marginal benefit equals marginal cost ($P = MC$).
    \item \textbf{Optimal Allocation:} $MB=MC$ for the last unit produced.
\end{itemize}

\subsection*{Budget Lines/Constraints}
A schedule showing combinations of two products a consumer can purchase given income and prices.
\begin{itemize}
    \item Equation: $I = P_x X + P_y Y$
    \item Slope: $-P_x / P_y$
    \item Intercepts: Y-intercept = $I/P_y$, X-intercept = $I/P_x$
    \item Points ON: Attainable, use all income.
    \item Points INSIDE: Attainable, don't use all income.
    \item Points OUTSIDE: Unattainable.
    \item Shifts with income changes; rotates with price changes.
\end{itemize}

\subsection*{Choice}
Limited income forces choices based on marginal analysis ($MB$ vs. $MC$).

\subsection*{Opportunity Cost}
The value of the next best alternative forgone when a choice is made.

\subsection*{Law of Increasing Opportunity Costs}
As production of a good increases, the opportunity cost of producing an additional unit rises (bowed-out PPC).

\subsection*{Rational Behavior}
Individuals make purposeful choices to maximize utility/satisfaction ("Rational self-interest").

\subsection*{Marginal Analysis}
Comparing marginal benefits ($MB$) and marginal costs ($MC$). Activity should continue as long as $MB \geq MC$.

\subsection*{Factors of Production}
Land, Labor, Capital, Entrepreneurial Ability.

\subsection*{Production Possibilities Model (PPC/PPF)}
\begin{itemize}
    \item Illustrates production choices under full employment, fixed resources, fixed technology, and two goods.
    \item Points ON the curve: Attainable and efficient.
    \item Points INSIDE: Attainable but inefficient (unemployment or underemployment of resources).
    \item Points OUTSIDE: Unattainable with current resources/technology.
    \item Economic Growth: Outward shift of the PPC.
\end{itemize}

\subsection*{The Market System}
Private property, freedom of enterprise/choice, self-interest, competition, markets and prices, technology/capital goods, specialization, use of money, active but limited government.

\subsection*{Division of Labor/Human Specialization}
Increases productivity.

\subsection*{Circular Flow Model}
Illustrates flow of resources, goods/services, and money between households and businesses via the resource and product markets. Government can be added.
\begin{itemize}
    \item \textbf{Product Market:} Households buy goods/services, businesses sell.
    \item \textbf{Resource Market:} Households sell resources (labor, land, capital, entrepreneurial ability), businesses buy.
\end{itemize}

\section*{Chapter 3: Demand, Supply, and Market Equilibrium}

\subsection*{Demand}
Schedule/curve showing quantities consumers are willing and able to buy at various prices.
\begin{itemize}
    \item \textbf{Law of Demand:} Price and quantity demanded are inversely related (ceteris paribus). Downward sloping curve due to income/substitution effects, diminishing marginal utility.
    \item \textbf{Change in Quantity Demanded:} Movement \textit{along} the curve due to a change in the product's own price.
    \item \textbf{Change in Demand:} Shift \textit{of} the entire curve due to changes in determinants (Shift factors).
    \item \textbf{Determinants (Shift factors):} Tastes/Preferences, Number of Buyers, Income (Normal vs. Inferior Goods), Prices of Related Goods (Substitutes vs. Complements), Consumer Expectations.
\end{itemize}

\subsection*{Supply}
Schedule/curve showing quantities producers are willing and able to sell at various prices.
\begin{itemize}
    \item \textbf{Law of Supply:} Price and quantity supplied are directly related (ceteris paribus). Upward sloping curve.
    \item \textbf{Change in Quantity Supplied:} Movement \textit{along} the curve due to a change in the product's own price.
    \item \textbf{Change in Supply:} Shift \textit{of} the entire curve due to changes in determinants (Shift factors).
    \item \textbf{Determinants (Shift factors):} Resource (Input) Prices, Technology, Taxes and Subsidies, Prices of Other Goods (Substitution in Production), Producer Expectations, Number of Sellers.
\end{itemize}

\subsection*{Market Equilibrium}
Where Demand and Supply curves intersect ($Q_d = Q_s$). Establishes equilibrium price ($P_e$) and quantity ($Q_e$).

\subsection*{Shortage}
$Q_d > Q_s$ (occurs below $P_e$). Puts upward pressure on price.

\subsection*{Surplus}
$Q_s > Q_d$ (occurs above $P_e$). Puts downward pressure on price.

\subsection*{Changes in Equilibrium (Four Laws)}
\begin{itemize}
    \item D$\uparrow \rightarrow P_e\uparrow, Q_e\uparrow$
    \item D$\downarrow \rightarrow P_e\downarrow, Q_e\downarrow$
    \item S$\uparrow \rightarrow P_e\downarrow, Q_e\uparrow$
    \item S$\downarrow \rightarrow P_e\uparrow, Q_e\downarrow$
    \item (Simultaneous shifts: Effect on one variable is certain, effect on the other is indeterminate).
\end{itemize}

\subsection*{Price Controls}
Government-mandated prices.
\begin{itemize}
    \item \textbf{Price Ceiling:} Maximum legal price. Binding if set \textit{below} $P_e$. Causes persistent shortage ($Q_d > Q_s$).
    \item \textbf{Price Floor:} Minimum legal price. Binding if set \textit{above} $P_e$. Causes persistent surplus ($Q_s > Q_d$).
\end{itemize}

\section*{Chapter 4-5: Market Failures: Public Goods and Externalities}

\subsection*{Market Failure}
Market system fails to produce the "right" amount of product (allocative inefficiency). Resources are over- or under-allocated.

\subsection*{Demand-Side Failures}
Demand curves don't reflect consumers' full willingness to pay (e.g., public goods).

\subsection*{Supply-Side Failures}
Supply curves don't reflect the full cost of production (e.g., negative externalities).

\subsection*{Consumer Surplus (CS)}
Difference between maximum price consumers are willing to pay and the market price actually paid. Area below demand curve, above market price, out to quantity purchased.

\subsection*{Producer Surplus (PS)}
Difference between the market price producers receive and the minimum price they are willing to accept. Area above supply curve, below market price, out to quantity sold.

\subsection*{Total Surplus (TS)}
$CS + PS$. Maximized at the allocatively efficient equilibrium ($Q_e$).

\subsection*{Efficiency (Deadweight) Losses}
Reductions in TS resulting from underproduction or overproduction.

\subsection*{Public Goods}
Characterized by:
\begin{itemize}
    \item \textbf{Nonrivalry:} One person's consumption doesn't preclude others from consuming it.
    \item \textbf{Nonexcludability:} Cannot prevent non-payers from benefiting.
    \item Creates the \textbf{Free-Rider Problem:} People benefit without paying, leading to under-provision by the private market. Government often provides these (e.g., national defense). Financed by taxes.
    \item Demand Curve: Vertical summation of individual willingness-to-pay curves. Optimal quantity where society's $MB = MC$.
\end{itemize}

\subsection*{Externalities}
Costs or benefits accruing to a third party external to the market transaction.
\begin{itemize}
    \item \textbf{Negative Externalities (Spillover Costs):} Production/consumption imposes costs on third parties (e.g., pollution). Supply curve understates true costs. Results in \textit{over-allocation} of resources. Correct with direct controls, taxes (Pigovian tax). Graphically: $S_{\text{social}}$ is above $S_{\text{private}}$. Deadweight loss triangle points towards optimal Q.
    \item \textbf{Positive Externalities (Spillover Benefits):} Production/consumption creates benefits for third parties (e.g., vaccination). Demand curve understates true benefits. Results in \textit{under-allocation} of resources. Correct with subsidies to buyers/producers, government provision. Graphically: $D_{\text{social}}$ is above $D_{\text{private}}$. Deadweight loss triangle points towards optimal Q.
\end{itemize}

\subsection*{Coase Theorem}
Private bargaining can potentially solve externality problems without government intervention if property rights are clearly defined and transaction costs are low.

\section*{Chapter 6: Elasticity}

\subsection*{Price Elasticity of Demand ($E_d$)}
Measures responsiveness of $Q_d$ to a change in Price.
\begin{itemize}
    \item Formula (Midpoint): $E_d = \frac{\text{\%}\Delta Q_d}{\text{\%}\Delta P} = \frac{(Q_2-Q_1)/[(Q_2+Q_1)/2]}{(P_2-P_1)/[(P_2+P_1)/2]}$ (Use absolute value).
    \item \textbf{Elastic:} $E_d > 1$ (\% change in Q $>$ change in P). Sensitive to price changes.
    \item \textbf{Inelastic:} $E_d < 1$ (\% change in Q $<$ change in P). Insensitive to price changes.
    \item \textbf{Unit Elastic:} $E_d = 1$ (\% change in Q = change in P).
    \item \textbf{Perfectly Inelastic:} $E_d = 0$ (Vertical D curve).
    \item \textbf{Perfectly Elastic:} $E_d = \infty$ (Horizontal D curve).
    \item \textbf{Total Revenue (TR) Test:}
        \begin{itemize}
            \item Elastic Demand: P$\uparrow \rightarrow$ TR$\downarrow$; P$\downarrow \rightarrow$ TR$\uparrow$ (P and TR move opposite).
            \item Inelastic Demand: P$\uparrow \rightarrow$ TR$\uparrow$; P$\downarrow \rightarrow$ TR$\downarrow$ (P and TR move together).
            \item Unit Elastic Demand: P change $\rightarrow$ TR unchanged (TR maximized).
        \end{itemize}
    \item \textbf{Determinants:} Substitutability (more subs $\rightarrow$ more elastic), Proportion of Income (higher proportion $\rightarrow$ more elastic), Luxuries vs. Necessities (luxuries $\rightarrow$ more elastic), Time (longer time $\rightarrow$ more elastic).
\end{itemize}

\subsection*{Price Elasticity of Supply ($E_s$)}
Measures responsiveness of $Q_s$ to a change in Price.
\begin{itemize}
    \item Formula (Midpoint): $E_s = \frac{\%\Delta Q_s}{\%\Delta P}$
    \item Depends on ease/speed of shifting resources.
    \item \textbf{Time Periods:} Market Period (perfectly inelastic), Short Run (somewhat elastic), Long Run (most elastic).
\end{itemize}

\subsection*{Income Elasticity of Demand ($E_i$)}
Measures responsiveness of $Q_d$ to a change in Income.
\begin{itemize}
    \item Formula: $E_i = \frac{\%\Delta Q_d}{\%\Delta \text{Income}}$
    \item \textbf{Normal Goods:} $E_i > 0$ (Income$\uparrow \rightarrow Q_d\uparrow$).
    \item \textbf{Inferior Goods:} $E_i < 0$ (Income$\uparrow \rightarrow Q_d\downarrow$).
\end{itemize}

\subsection*{Cross-Price Elasticity of Demand ($E_{xy}$)}
Measures responsiveness of $Q_d$ of Good X to a change in Price of Good Y.
\begin{itemize}
    \item Formula: $E_{xy} = \frac{\%\Delta Q_d \text{ of X}}{\%\Delta P \text{ of Y}}$
    \item \textbf{Substitute Goods:} $E_{xy} > 0$ ($P_y\uparrow \rightarrow Q_{dx}\uparrow$).
    \item \textbf{Complementary Goods:} $E_{xy} < 0$ ($P_y\uparrow \rightarrow Q_{dx}\downarrow$).
    \item \textbf{Independent Goods:} $E_{xy} = 0$.
\end{itemize}

\section*{Chapter 7: Utility Maximization}

\subsection*{Law of Diminishing Marginal Utility}
As consumption of a good increases, the extra satisfaction (marginal utility) obtained from each additional unit decreases.

\subsection*{Total Utility (TU)}
Total satisfaction from consuming a specific quantity.

\subsection*{Marginal Utility (MU)}
Extra satisfaction from consuming one more unit. $MU = \Delta TU / \Delta Q$. MU curve slopes downward.

\subsection*{Theory of Consumer Behavior}
Explains how consumers allocate income. Assumes rational behavior, preferences, budget constraint, prices.

\subsection*{Utility Maximization Rule}
Allocate income such that the marginal utility per dollar spent is equal for all goods purchased.
\begin{itemize}
    \item $\frac{MU_A}{P_A} = \frac{MU_B}{P_B} = \dots$ for the last unit of each good bought.
    \item Alternative form: $\frac{MU_A}{MU_B} = \frac{P_A}{P_B}$
    \item The entire budget must be spent.
\end{itemize}

\subsection*{Deriving the Demand Curve}
Utility maximization explains the inverse relationship between P and $Q_d$. If $P_A$ falls, $\frac{MU_A}{P_A} > \frac{MU_B}{P_B}$. To restore equilibrium, consumer buys more A (decreasing $MU_A$) and possibly less B.

\subsection*{Income Effect}
Impact of a price change on consumer's real income (purchasing power) and thus on $Q_d$.

\subsection*{Substitution Effect}
Impact of a price change on a product's relative expensiveness and thus on its $Q_d$.

For normal goods, income and substitution effects work in the same direction, reinforcing the law of demand.

\section*{Chapter 8: Behavioral Economics}

Combines insights from economics and psychology. Challenges assumption of perfect rationality.

\subsection*{Prospect Theory}
\begin{itemize}
    \item People judge outcomes relative to a status quo (as gains or losses).
    \item Diminishing marginal utility for gains, diminishing marginal \textit{disutility} for losses.
    \item \textbf{Loss Aversion:} Losses are felt more intensely than equivalent gains (e.g., losing \$10 feels worse than finding \$10 feels good).
\end{itemize}

\subsection*{Framing Effects}
How choices are presented (framed) influences decisions, even if underlying options are identical.

\subsection*{Anchoring}
Initial information (even if irrelevant) unduly influences subsequent judgments/decisions.

\subsection*{Mental Accounting}
Separating decisions into "mental boxes" rather than looking at the overall picture (e.g., saving for one goal while borrowing for another).

\subsection*{Endowment Effect}
Tendency to overvalue something simply because one owns it.

\subsection*{Status Quo Bias}
Tendency to stick with the default option. Exploited in "opt-out" vs. "opt-in" programs (e.g., retirement savings).

\section*{Chapter 9: Businesses and the Costs of Production}

\subsection*{Economic Costs}
Include all opportunity costs (explicit + implicit).

\subsection*{Explicit Costs}
Monetary payments made to outsiders for resources.

\subsection*{Implicit Costs}
Opportunity costs of using self-owned resources (e.g., forgone wages, forgone interest, normal profit).

\subsection*{Accounting Profit}
Total Revenue (TR) - Explicit Costs.

\subsection*{Economic Profit}
Total Revenue (TR) - Economic Costs (Explicit + Implicit).
\begin{itemize}
    \item Economic Profit $>$ 0: Doing better than alternatives (abnormal profit).
    \item Economic Profit = 0: Earning a normal profit (doing as well as alternatives).
    \item Economic Profit $<$ 0: Doing worse than alternatives (should consider exiting).
\end{itemize}

\subsection*{Short Run (SR)}
Period with at least one fixed input (e.g., plant size).

\subsection*{Long Run (LR)}
Period where all inputs are variable (firm can change plant size).

\subsection*{SR Production Relationships}
\begin{itemize}
    \item \textbf{Total Product (TP):} Total output produced.
    \item \textbf{Marginal Product (MP):} Additional output from adding one more unit of a variable input (e.g., labor). $MP = \Delta TP / \Delta \text{Labor}$.
    \item \textbf{Average Product (AP):} Output per unit of variable input. $AP = TP / \text{Labor}$.
    \item \textbf{Law of Diminishing Returns:} As successive units of a variable resource are added to a fixed resource, the MP eventually declines. MP intersects AP at AP's maximum.
\end{itemize}

\subsection*{SR Cost Curves}
\begin{itemize}
    \item \textbf{Total Fixed Cost (TFC):} Costs that don't vary with output.
    \item \textbf{Total Variable Cost (TVC):} Costs that vary with output.
    \item \textbf{Total Cost (TC):} TFC + TVC.
    \item \textbf{Average Fixed Cost (AFC):} TFC / Q (declines as Q increases).
    \item \textbf{Average Variable Cost (AVC):} TVC / Q (U-shaped).
    \item \textbf{Average Total Cost (ATC):} TC / Q = AFC + AVC (U-shaped).
    \item \textbf{Marginal Cost (MC):} Additional cost of producing one more unit. $MC = \Delta TC / \Delta Q = \Delta TVC / \Delta Q$. (U-shaped, intersects AVC and ATC at their minimum points).
    \item Relationship between MP/AP and MC/AVC: When MP$\uparrow$, MC$\downarrow$; When MP$\downarrow$, MC$\uparrow$. When AP$\uparrow$, AVC$\downarrow$; When AP$\downarrow$, AVC$\uparrow$. MC intersects AVC at AVC's minimum (where MP=AP).
\end{itemize}

\subsection*{LR Cost Curve (LRATC)}
Shows the lowest ATC at which any output level can be produced when the firm can vary all inputs (plant size). Formed by the minimum points of SRATC curves for different plant sizes (envelope curve).
\begin{itemize}
    \item \textbf{Economies of Scale:} LRATC falls as output increases (due to labor/managerial specialization, efficient capital use, etc.).
    \item \textbf{Diseconomies of Scale:} LRATC rises as output increases (due to control/coordination problems).
    \item \textbf{Constant Returns to Scale:} LRATC is constant.
    \item \textbf{Minimum Efficient Scale (MES):} Lowest output level where LRATC is minimized. Affects industry structure.
\end{itemize}

\section*{Chapters 10-11: Pure Competition in the Short Run and Long Run}

\subsection*{Pure Competition Characteristics}
\begin{itemize}
    \item Very Large Numbers: Many independent sellers and buyers.
    \item Standardized Product: Identical/homogenous goods.
    \item "Price Taker": Firms cannot influence market price; must accept it. Firm's demand is perfectly elastic (horizontal line at market price P). $P = MR = AR$.
    \item Free Entry and Exit: No significant barriers.
\end{itemize}

\subsection*{Profit Maximization (SR)}
Firms aim to maximize economic profit or minimize loss.
\begin{itemize}
    \item \textbf{Approach 1: Total Revenue - Total Cost (TR-TC):} Maximize the positive difference or minimize the negative difference.
    \item \textbf{Approach 2: Marginal Revenue - Marginal Cost (MR=MC Rule):} Produce the last unit where $MR \geq MC$. For pure competition, this is $\mathbf{P=MC}$.
    \item \textbf{Profit Calculation:} Profit = $(P - ATC) \times Q$.
    \item \textbf{Three Cases (Illustrate Graphically):}
        \begin{enumerate}
            \item \textbf{Profit Maximization:} $P > ATC$ (at Q where P=MC). Economic Profit $>$ 0.
            \item \textbf{Loss Minimization:} $ATC > P > AVC$ (at Q where P=MC). Economic Profit $<$ 0, but loss is less than TFC. Firm \textit{should} produce.
            \item \textbf{Shutdown Case:} $P < AVC$ (at Q where P=MC). Loss if producing $>$ TFC. Firm \textit{should not} produce (Q=0), loss = TFC.
        \end{enumerate}
    \item \textbf{Short-Run Supply Curve:} The firm's MC curve \textit{above} the minimum AVC point. Market supply is the horizontal sum of individual firms' supply curves.
\end{itemize}

\subsection*{Profit Maximization (LR)}
Free entry/exit ensures firms earn only a normal profit (Economic Profit = 0).
\begin{itemize}
    \item If SR economic profits exist ($P > ATC$): New firms enter $\rightarrow$ Market S$\uparrow \rightarrow$ P$\downarrow \rightarrow$ Profits fall until $P = \min ATC$.
    \item If SR economic losses exist ($P < ATC$): Firms exit $\rightarrow$ Market S$\downarrow \rightarrow$ P$\uparrow \rightarrow$ Losses shrink until $P = \min ATC$.
    \item \textbf{LR Equilibrium:} $P = MR = MC = \min \text{SRATC} = \min \text{LRATC}$.
    \item Productive Efficiency Achieved: $P = \min ATC$.
    \item Allocative Efficiency Achieved: $P = MC$.
\end{itemize}

\subsection*{LR Industry Supply Curve}
\begin{itemize}
    \item \textbf{Constant-Cost Industry:} Entry/exit doesn't affect input prices/costs. LR supply curve is horizontal.
    \item \textbf{Increasing-Cost Industry:} Entry raises input prices/costs. LR supply curve is upsloping.
    \item \textbf{Decreasing-Cost Industry:} Entry lowers input prices/costs. LR supply curve is downsloping.
\end{itemize}

\section*{Chapter 12: Pure Monopoly}

\subsection*{Characteristics}
Single seller, unique product (no close substitutes), price maker (controls price via Q), blocked entry (barriers: economies of scale, legal, ownership, pricing).

\subsection*{Demand Curve}
Monopoly IS the industry; faces downsloping market demand curve.

\subsection*{Marginal Revenue (MR)}
$MR < P$ for all $Q > 1$ because monopolist must lower price on all units to sell more. MR curve lies below the demand curve.

\subsection*{Profit Maximization}
Produces Q where $MR=MC$. Finds price on the demand curve above that Q.

\subsection*{Profit Calculation}
Profit = $(P - ATC) \times Q$. Can earn LR economic profits due to barriers.

\subsection*{Inefficiency}
\begin{itemize}
    \item Allocative Inefficiency: $P > MC$ (underproduces relative to social optimum). Creates deadweight loss.
    \item Productive Inefficiency: May not produce at min ATC (unless lucky).
\end{itemize}

\subsection*{Misconceptions}
Not highest price, aims for max total profit (not unit profit), can incur losses.

\subsection*{Price Discrimination}
Charging different prices to different buyers for the same product, not based on cost differences. Conditions: Monopoly power, market segregation, no resale. Increases profit. Perfect price discrimination eliminates CS.

\section*{Chapter 26: Introduction to Macroeconomics}

\subsection*{Performance Assessment}
Real GDP, Unemployment, Inflation.

\subsection*{Real GDP}
Measures value of final goods/services produced within a country during a specific period, adjusted for price changes.

\subsection*{Nominal GDP}
GDP measured in current prices (unadjusted for inflation).

\subsection*{Unemployment}
Percentage of the labor force unable to find work. Cost is forgone output (GDP gap).

\subsection*{Inflation}
Increase in the overall level of prices. Reduces purchasing power.

\subsection*{Modern Economic Growth}
Sustained increases in real GDP per capita. Started with Industrial Revolution. Vast differences in living standards globally.

\subsection*{Saving, Investment, Expectations, Shocks}
Key factors influencing macro outcomes. Saving funds investment (creation of capital). Expectations affect consumption/investment. Shocks (unexpected events) drive business cycles. Sticky prices often worsen effects of demand shocks.

\section*{Chapter 27: Measuring Domestic Output and National Income}

\subsection*{Gross Domestic Product (GDP)}
Market value of all \textit{final} goods and services produced \textit{within} a nation's borders in a given year. Avoids multiple counting by ignoring intermediate goods. Value added approach sums value added at each production stage.

\subsection*{Expenditures Approach}
$GDP = C + I_g + G + X_n$
\begin{itemize}
    \item \textbf{C (Personal Consumption Expenditures):} Spending by households on durables, nondurables, services.
    \item \textbf{$I_g$ (Gross Private Domestic Investment):} Spending on machinery, equipment, tools; all construction; \textit{changes} in inventories. $I_g = \text{Net Investment} + \text{Depreciation}$. Net investment expands capital stock.
    \item \textbf{G (Government Purchases):} Spending by all levels of government on goods/services and publicly owned capital. Excludes transfer payments.
    \item \textbf{$X_n$ (Net Exports):} Exports (X) - Imports (M).
\end{itemize}

\subsection*{Income Approach}
GDP = Compensation of Employees + Rents + Interest + Proprietors' Income + Corporate Profits + Taxes on Production \& Imports - Net Foreign Factor Income + Statistical Discrepancy + Consumption of Fixed Capital (Depreciation).

\subsection*{Other Measures}
Net Domestic Product (NDP = GDP - Depreciation), National Income (NI), Personal Income (PI), Disposable Income (DI = PI - Personal Taxes; DI = C + S).

\subsection*{Nominal vs. Real GDP}
\begin{itemize}
    \item Nominal GDP uses current prices.
    \item Real GDP uses prices from a base year; adjusts for inflation. Real GDP = Nominal GDP / Price Index (in hundredths).
\end{itemize}

\subsection*{Price Index}
Measure of the price of a market basket in a specific year compared to base year price. PI = (Price of basket in specific year / Price of basket in base year) $\times$ 100. GDP Deflator is a common index used.

\subsection*{Shortcomings of GDP}
Doesn't measure nonmarket activities, leisure, improved product quality (fully), underground economy, environmental impact, income distribution, noneconomic sources of well-being.

\section*{Chapter 28: Unemployment, Inflation, and Economic Fluctuations}

\subsection*{Business Cycle}
Alternating rises and declines in economic activity. Phases: Peak, Recession (contraction), Trough, Expansion (recovery). Driven by shocks, often demand shocks with sticky prices. Durables/capital goods most affected.

\subsection*{Unemployment}
\begin{itemize}
    \item \textbf{Measurement:} Based on household survey. Labor force = employed + unemployed (actively seeking work). Unemployment Rate = (Unemployed / Labor Force) $\times$ 100.
    \item \textbf{Criticisms:} Understates by excluding discouraged workers; counts part-time as fully employed.
    \item \textbf{Types:}
        \begin{enumerate}
            \item \textbf{Frictional:} "Search" or "wait" unemployment. Between jobs or entering labor force. Inevitable/desirable.
            \item \textbf{Structural:} Mismatch between worker skills/location and job requirements/location. Requires retraining/relocation. More long-term.
            \item \textbf{Cyclical:} Caused by insufficient total spending (recession). Real GDP $<$ Potential GDP.
        \end{enumerate}
    \item \textbf{Full Employment (Natural Rate of Unemployment - NRU):} Frictional + Structural unemployment. Occurs when cyclical unemployment is zero. Economy producing at Potential Output (Real GDP). Varies over time (currently estimated around 4-5\%).
    \item \textbf{GDP Gap:} Forgone output when actual GDP < potential GDP. GDP Gap = Actual GDP - Potential GDP (usually negative in recession).
    \item \textbf{Okun's Law:} For every 1 percentage point actual unemployment exceeds NRU, a negative GDP gap of about 2\% occurs.
\end{itemize}

\subsection*{Inflation}
Rise in the general level of prices. Purchasing power of money falls.
\begin{itemize}
    \item \textbf{Measurement:} Consumer Price Index (CPI) is main measure. CPI = (Price of market basket in specific year / Price of basket in base year) $\times$ 100. Inflation Rate = [(CPI year 2 - CPI year 1) / CPI year 1] $\times$ 100.
    \item \textbf{Types:}
        \begin{enumerate}
            \item \textbf{Demand-Pull Inflation:} Excess total spending pulls prices up ("too much spending chasing too few goods"). Shifts AD right.
            \item \textbf{Cost-Push Inflation:} Rising per-unit production costs push prices up (e.g., supply shocks like oil price hikes). Shifts AS left. Often causes recession simultaneously (stagflation).
        \end{enumerate}
    \item \textbf{Nominal vs. Real Income:} Real Income = Nominal Income / Price Index (in hundredths). \% Change Real Income $\approx$ \% Change Nominal Income - \% Change Price Level (Inflation Rate).
    \item \textbf{Who is Hurt/Helped?}
        \begin{itemize}
            \item Hurt by \textit{unanticipated} inflation: Fixed-income receivers, Savers, Creditors (lenders).
            \item Helped/Unaffected by \textit{unanticipated} inflation: Flexible-income receivers (if income rises faster than inflation), Debtors (borrowers).
        \end{itemize}
    \item \textbf{Anticipated Inflation:} Effects are lessened as people plan for it (e.g., inflation premiums in loans - nominal interest rate $\approx$ real interest rate + inflation premium (Fisher Effect)).
    \item \textbf{Hyperinflation:} Extraordinarily rapid inflation. Devastating economic effects.
\end{itemize}

\section*{Chapter 30: Economic Growth}

\subsection*{Definition}
Increase in Real GDP over time OR Increase in Real GDP \textit{per capita} over time (better for living standards). Growth rate calculation: [(GDP Yr 2 - GDP Yr 1) / GDP Yr 1] $\times$ 100. Rule of 70: Approx. years to double = 70 / Annual growth rate \%.

\subsection*{Modern Economic Growth}
Characterized by sustained, ongoing increases in living standards. Started relatively recently (Industrial Revolution). Uneven distribution globally.

\subsection*{Institutional Structures Promoting Growth}
Strong property rights, patents/copyrights, efficient financial institutions, literacy/education, free trade, competitive market system.

\subsection*{Determinants of Growth (Supply Factors)}
\begin{enumerate}
    \item Increases in quantity/quality of natural resources.
    \item Increases in quantity/quality of human resources (labor, education).
    \item Increases in supply (stock) of capital goods.
    \item Improvements in technology.
\end{enumerate}
(Shift PPC outward)

\subsection*{Demand Factor}
Households, businesses, government must purchase the expanding output.

\subsection*{Efficiency Factor}
Economy must achieve efficiency (use resources fully and in least costly way) to reach potential.

\subsection*{Production Possibilities Analysis}
Growth shown as outward shift of PPC or movement from inside curve to the curve. Real GDP = Hours of Work $\times$ Labor Productivity. Growth comes from more hours or higher productivity.

\subsection*{Accounting for Growth (US)}
Primarily driven by increases in labor productivity due to:
\begin{itemize}
    \item Technological advance (largest contributor).
    \item Quantity of capital per worker.
    \item Education and training (human capital).
    \item Economies of scale, improved resource allocation.
\end{itemize}

\subsection*{Productivity Acceleration/Slowdown}
Recent trends debated. Role of IT revolution, startups, global competition.

\subsection*{Desirability}
Growth improves living standards, reduces scarcity burden. Concerns: Environmental impact, inequality, burnout.

\section*{Chapter 31: The Aggregate Expenditures Model (Keynesian Cross)}

Focuses on relationship between total spending and real GDP/income in the \textit{short run}, assuming a \textit{fixed price level}. Basis for AD curve.

\subsection*{Assumptions}
Closed, private economy initially ($C + I_g$). Prices fixed. GDP = DI.

\subsection*{Consumption (C) and Saving (S)}
$DI = C + S$.
\begin{itemize}
    \item \textbf{Consumption Schedule:} Planned household spending at different DI levels.
    \item \textbf{Saving Schedule:} Planned household saving at different DI levels.
    \item \textbf{Break-Even Income:} DI level where C=DI (S=0).
    \item \textbf{Average Propensity to Consume (APC):} $C / DI$.
    \item \textbf{Average Propensity to Save (APS):} $S / DI$. $APC + APS = 1$.
    \item \textbf{Marginal Propensity to Consume (MPC):} Change in C / Change in DI ($\Delta C / \Delta DI$). Slope of consumption schedule.
    \item \textbf{Marginal Propensity to Save (MPS):} Change in S / Change in DI ($\Delta S / \Delta DI$). Slope of saving schedule. $MPC + MPS = 1$.
    \item \textbf{Non-Income Determinants:} Wealth, Borrowing, Expectations, Real Interest Rates (shift C \& S schedules).
\end{itemize}

\subsection*{Investment ($I_g$)}
Spending on capital goods, construction, inventory changes. Assumed independent of current income/GDP (autonomous) initially.
\begin{itemize}
    \item \textbf{Expected Rate of Return (r):} Profit / Cost. Firms invest if $r \geq \text{real interest rate (i)}$.
    \item \textbf{Investment Demand Curve:} Shows amount of investment at various real interest rates (i). Downsloping. Shifts with changes in acquisition/maintenance costs, business taxes, technology, stock of capital, expectations.
    \item \textbf{Instability:} Investment is the most volatile component of spending.
\end{itemize}

\subsection*{Equilibrium GDP (Private, Closed Economy)}
Level where total output = aggregate expenditures (AE).
\begin{itemize}
    \item $\mathbf{AE = C + I_g.}$ Graphically, where AE line intersects 45$^\circ$ line (where AE=GDP).
    \item \textbf{Leakages = Injections:} Saving (leakage) = Planned Investment (injection). $S = I_g$.
    \item Disequilibrium: If AE > GDP $\rightarrow$ Inventories fall $\rightarrow$ Firms increase production $\rightarrow$ GDP rises toward equilibrium. If AE < GDP $\rightarrow$ Inventories rise $\rightarrow$ Firms decrease production $\rightarrow$ GDP falls toward equilibrium.
\end{itemize}

\subsection*{Multiplier Effect}
A change in autonomous spending (C, $I_g$, G, $X_n$) leads to a larger change in equilibrium GDP.
\begin{itemize}
    \item Multiplier = Change in Real GDP / Initial Change in Spending.
    \item Multiplier = $1 / (1 - MPC) = 1 / MPS$.
    \item Rationale: Initial spending becomes income, which leads to induced consumption spending, which becomes more income, etc. Size depends on MPC (larger MPC $\rightarrow$ larger multiplier).
\end{itemize}

\subsection*{Adding International Trade (Open Economy)}
$AE = C + I_g + X_n$.
\begin{itemize}
    \item \textbf{Net Exports ($X_n$):} Exports (X) - Imports (M). Exports create production/income in US; Imports are spending leaked abroad. $X_n$ can be positive or negative. Determinants: Prosperity abroad, tariffs, exchange rates.
\end{itemize}

\subsection*{Adding Government Sector (Mixed Economy)}
$AE = C + I_g + G + X_n$.
\begin{itemize}
    \item \textbf{Government Purchases (G):} Assumed autonomous.
    \item \textbf{Taxes (T):} Lump-sum tax assumed initially. Reduces DI (DI = GDP - T). Affects C and S via MPC/MPS.
    \item Equilibrium (Mixed, Open): AE = GDP OR Leakages ($S + M + T$) = Injections ($I_g + X + G$).
\end{itemize}

\subsection*{Recessionary Expenditure Gap}
Amount by which AE at full-employment GDP falls short of full-employment GDP. Causes cyclical unemployment. Need to increase AE (shift AE up) via multiplier to close gap.

\subsection*{Inflationary Expenditure Gap}
Amount by which AE at full-employment GDP exceeds full-employment GDP. Causes demand-pull inflation. Need to decrease AE (shift AE down) via multiplier to close gap.

(Note: AE model assumes fixed prices; AD-AS model allows price changes).

\section*{Chapter 32: Aggregate Demand (AD) and Aggregate Supply (AS)}

Model explains changes in real output and price level simultaneously.

\subsection*{Aggregate Demand (AD)}
Curve showing amount of real output (GDP) buyers collectively desire to purchase at each possible price level. Downsloping due to:
\begin{enumerate}
    \item \textbf{Real-Balances Effect:} Higher P $\rightarrow$ Lower real value of assets $\rightarrow$ Lower C.
    \item \textbf{Interest-Rate Effect:} Higher P $\rightarrow$ Higher demand for money $\rightarrow$ Higher interest rates $\rightarrow$ Lower $I_g$ (and C).
    \item \textbf{Foreign Purchases Effect:} Higher US P $\rightarrow$ Lower US exports (X), Higher US imports (M) $\rightarrow$ Lower $X_n$.
\end{enumerate}

\subsection*{Determinants of AD (Shift Factors)}
Shift AD curve right (increase) or left (decrease). Changes in:
\begin{itemize}
    \item \textbf{Consumer Spending (C):} Wealth, expectations, borrowing, taxes.
    \item \textbf{Investment Spending ($I_g$):} Interest rates (monetary policy), expected returns (expectations, technology, business taxes, capital stock).
    \item \textbf{Government Spending (G):} Fiscal policy.
    \item \textbf{Net Export Spending ($X_n$):} National income abroad, exchange rates.
\end{itemize}
(Essentially, factors affecting components of AE, \textit{other than price level changes}).

\subsection*{Aggregate Supply (AS)}
Curve showing level of real output producers are willing and able to produce at each possible price level. Shape depends on time horizon.
\begin{itemize}
    \item \textbf{Immediate Short Run:} Horizontal AS. Input/output prices fixed. Output determined solely by AD.
    \item \textbf{Short Run (SRAS):} Upsloping AS. Input prices fixed (or sticky), output prices flexible. Higher P $\rightarrow$ Higher profits $\rightarrow$ Higher $Q_s$. Flatter at low output, steeper near full employment.
    \item \textbf{Long Run (LRAS):} Vertical AS at the full-employment level of output (Potential GDP). Input prices fully adjust to output prices. Output determined by supply factors, independent of price level.
\end{itemize}

\subsection*{Determinants of AS (Shift Factors)}
Shift AS curve right (increase) or left (decrease). Changes in:
\begin{itemize}
    \item \textbf{Input Prices:} Domestic (wages, land, capital), Imported resources.
    \item \textbf{Productivity:} Real output per unit of input (Productivity = Total Output / Total Inputs). Main source of LR growth. Affects per-unit production cost.
    \item \textbf{Legal-Institutional Environment:} Business taxes/subsidies, government regulations.
\end{itemize}

\subsection*{Equilibrium}
Intersection of AD and SRAS determines equilibrium real output and price level.

\subsection*{Changes in Equilibrium}
\begin{itemize}
    \item \textbf{AD Increases:} Demand-Pull Inflation. P$\uparrow$, Real Output $\uparrow$ (Short Run). If starting at full employment, creates inflationary gap. In LR, nominal wages rise, SRAS shifts left, returning output to potential but at higher P.
    \item \textbf{AD Decreases:} Recession and Cyclical Unemployment. P$\downarrow$ (or disinflation), Real Output $\downarrow$ (Short Run). Creates recessionary gap. If prices/wages are downwardly flexible (Keynesian debate!), SRAS eventually shifts right returning to full employment at lower P. If sticky, requires policy intervention or long adjustment. Ratchet effect: Prices sticky downwards.
    \item \textbf{AS Decreases:} Cost-Push Inflation. P$\uparrow$, Real Output $\downarrow$ (Stagflation). Shifts SRAS left. Negative GDP gap. Policy dilemma: Fighting inflation worsens unemployment, fighting unemployment worsens inflation.
    \item \textbf{AS Increases:} Full Employment with Price Stability. P$\downarrow$, Real Output $\uparrow$. Shifts SRAS right. Desirable outcome (e.g., from productivity growth).
\end{itemize}

\section*{Chapter 34-35: Money, Banking, and Money Creation}

\subsection*{Functions of Money}
Medium of Exchange, Unit of Account, Store of Value.

\subsection*{Money Supply Definitions}
\begin{itemize}
    \item \textbf{M1:} Currency + Checkable Deposits. Most liquid. Directly usable as medium of exchange.
    \item \textbf{M2:} M1 + Near-Monies (Savings deposits including MMDAs, Small (<\$100k) time deposits, MMMFs held by individuals). Broader measure.
\end{itemize}

\subsection*{What "Backs" Money?}
Generally accepted, legal tender, relatively scarce. Value derived from goods/services it commands (purchasing power). Value of Dollar = $1 / \text{Price Level}$. Inflation erodes value. Managed by Federal Reserve to maintain stability.

\subsection*{Federal Reserve System (The Fed)}
Central bank of US. Structure: Board of Governors (7 members, appointed by Pres.), 12 Federal Reserve Banks (quasi-public, bankers' banks), Federal Open Market Committee (FOMC - 12 members incl. Board + 5 FRB Presidents; sets monetary policy). Functions: Issue currency, set reserve requirements, lend to banks (discount window), check collection, fiscal agent for US gov't, supervise banks, CONTROL THE MONEY SUPPLY. Fed independence is important.

\subsection*{Financial Crisis of 2007-2008}
Causes (subprime mortgages, leverage, securitization, failures), Policy Response (TARP, Fed lender of last resort actions). Led to Dodd-Frank Act reforms.

\subsection*{Fractional Reserve Banking System}
Banks required to hold only a fraction of checkable deposits as reserves. Characteristics: Banks create money through lending; vulnerable to "panics" or "runs" (deposit insurance helps prevent).

\subsection*{Bank Balance Sheet}
Assets = Liabilities + Net Worth. Key Assets: Reserves (Required + Excess), Loans, Securities. Key Liabilities: Checkable Deposits.

\subsection*{Required Reserves}
Amount banks \textit{must} hold. Required Reserves = Checkable Deposits $\times$ Reserve Ratio (R).

\subsection*{Excess Reserves (ER)}
Reserves held above requirement. ER = Actual Reserves - Required Reserves. Banks can lend out ER.

\subsection*{Money Creation Process}
\begin{itemize}
    \item Single Bank: Can lend only its ER. Creates checkable deposits (money) equal to the loan amount. Loan repayment destroys money.
    \item Banking System: Multiple deposit expansion. Initial deposit creates ER; bank lends ER; loan check deposited in another bank, creating new reserves/ER; process repeats.
    \item \textbf{Monetary Multiplier (m):} $m = 1 / \text{Required Reserve Ratio (R)}$. Maximum amount of \textit{new} checkable deposit money created by banking system for each \$1 of ER. Max Checkable Deposit Creation = ER $\times$ m.
    \item Reversibility: Money destruction when loans repaid or checks cashed against reserves.
    \item Leakages (currency drains, banks holding excess reserves) reduce the actual multiplier effect.
\end{itemize}

\section*{Chapter 36: Interest Rates and Monetary Policy}

\subsection*{Interest Rates}
Price paid for the use of money. Determined by money supply and money demand. Focus on Federal Funds Rate (FFR) - rate banks charge each other for overnight loans of reserves. Fed targets the FFR through monetary policy.

\subsection*{Demand for Money ($D_m$)}
Why hold money (which earns little/no interest)?
\begin{itemize}
    \item \textbf{Transactions Demand ($D_t$):} To buy goods/services. Varies directly with nominal GDP. Vertical line graph.
    \item \textbf{Asset Demand ($D_a$):} To hold as an asset (store of value). Varies inversely with opportunity cost (interest rate). Downsloping curve graph.
    \item \textbf{Total Money Demand ($D_m$):} $D_t + D_a$. Downsloping curve. Shifts with changes in nominal GDP or price level.
\end{itemize}

\subsection*{Money Market}
Supply of Money ($S_m$ - vertical line set by Fed) and Demand for Money ($D_m$) determine the equilibrium interest rate ($i_e$).

\subsection*{Monetary Policy Tools}
How Fed influences money supply and interest rates.
\begin{enumerate}
    \item \textbf{Open Market Operations (OMOs):} \textit{Most important tool}. Buying/selling government securities (bonds) from/to commercial banks and the public.
        \begin{itemize}
            \item \textit{Buying} securities $\rightarrow$ Increases bank reserves $\rightarrow$ Increases lending (via multiplier) $\rightarrow$ Increases money supply $\rightarrow$ \textit{Lowers} FFR/other interest rates (Expansionary).
            \item \textit{Selling} securities $\rightarrow$ Decreases bank reserves $\rightarrow$ Decreases lending $\rightarrow$ Decreases money supply $\rightarrow$ \textit{Raises} FFR/other interest rates (Contractionary).
        \end{itemize}
    \item \textbf{Reserve Ratio (R):} Fraction of deposits banks must hold as reserves.
        \begin{itemize}
            \item \textit{Lowering} R $\rightarrow$ Increases ER $\rightarrow$ Increases lending (multiplier increases) $\rightarrow$ Increases money supply $\rightarrow$ Lowers interest rates (Expansionary).
            \item \textit{Raising} R $\rightarrow$ Decreases ER $\rightarrow$ Decreases lending (multiplier decreases) $\rightarrow$ Decreases money supply $\rightarrow$ Raises interest rates (Contractionary). (Powerful, rarely changed).
        \end{itemize}
    \item \textbf{Discount Rate:} Interest rate Fed charges banks for loans.
        \begin{itemize}
            \item \textit{Lowering} discount rate $\rightarrow$ Encourages borrowing reserves $\rightarrow$ Increases reserves/lending $\rightarrow$ Increases money supply $\rightarrow$ Lowers FFR (Expansionary). (Passive tool).
        \end{itemize}
    \item \textbf{Interest on Reserves (IOR):} Rate Fed pays banks on reserves held at Fed.
        \begin{itemize}
            \item \textit{Lowering} IOR $\rightarrow$ Reduces incentive to hold reserves $\rightarrow$ Increases lending $\rightarrow$ Increases money supply $\rightarrow$ Lowers FFR (Expansionary).
            \item \textit{Raising} IOR $\rightarrow$ Increases incentive to hold reserves $\rightarrow$ Decreases lending $\rightarrow$ Decreases money supply $\rightarrow$ Raises FFR (Contractionary).
        \end{itemize}
\end{enumerate}

\subsection*{Federal Funds Rate Targeting}
Fed uses OMOs daily to keep FFR near its target. Other rates tend to follow FFR.

\subsection*{Expansionary ("Easy") Monetary Policy}
Used during recession/slow growth. Goal: Increase AD. Fed \textit{buys} securities, lowers R, lowers discount rate, lowers IOR $\rightarrow$ Increases money supply $\rightarrow$ Lowers interest rates $\rightarrow$ Increases $I_g$ (and C) $\rightarrow$ Increases AD $\rightarrow$ Increases Real GDP (closes recessionary gap).

\subsection*{Contractionary ("Tight") Monetary Policy}
Used during demand-pull inflation. Goal: Decrease AD. Fed \textit{sells} securities, raises R, raises discount rate, raises IOR $\rightarrow$ Decreases money supply $\rightarrow$ Raises interest rates $\rightarrow$ Decreases $I_g$ (and C) $\rightarrow$ Decreases AD $\rightarrow$ Reduces inflation (closes inflationary gap).

\subsection*{Taylor Rule}
Guideline for Fed interest rate targeting based on current inflation and output gaps.

\subsection*{Evaluation}
Advantages (speed, flexibility, isolation from political pressure). Complications (lags, cyclical asymmetry - less reliable in recession, liquidity trap - zero lower bound problem). Quantitative Easing (QE) used when rates near zero.

\newpage % Start problems on new page

\section*{Example Practice Problems (Covering Specified Chapters)}
\begin{enumerate}[label=\arabic*.]
    \item \textbf{Opportunity Cost:} Define opportunity cost and explain its relevance using a Production Possibilities Curve.
    \item \textbf{Budget Line:} A consumer has an income of \$60. The price of good X is \$5 and the price of good Y is \$10. Draw the budget line. What happens if the price of Y falls to \$5?
    \item \textbf{Supply \& Demand:} The market for widgets has Demand: $P = 50 - Q_d$ and Supply: $P = 10 + Q_s$. Find the equilibrium price and quantity. If a technological improvement shifts the supply curve to $P = 5 + Q_s$, what is the new equilibrium?
    \item \textbf{Externalities:} Explain the difference between a positive and negative externality, providing an example of each. Illustrate the deadweight loss associated with a negative externality.
    \item \textbf{Elasticity:} If a 5\% increase in the price of a product leads to a 10\% decrease in quantity demanded, calculate the price elasticity of demand. Is demand elastic, inelastic, or unit elastic? Will total revenue increase or decrease?
    \item \textbf{Utility Maximization:} A consumer spends their income on pizza ($P_p=\$2$) and soda ($P_s=\$1$). They are currently consuming a bundle where $MU_{\text{pizza}} = 20$ and $MU_{\text{soda}} = 12$. Are they maximizing utility? Explain why or why not and what they should do.
    \item \textbf{Costs of Production:} What is the difference between accounting profit and economic profit? Why does the marginal cost curve intersect the average variable cost curve at its minimum point?
    \item \textbf{Perfect Competition:} Why is the demand curve facing a purely competitive firm perfectly elastic? Draw a graph showing a purely competitive firm making a short-run economic profit and label the profit area. What will happen in the long run?
    \item \textbf{Monopoly:} How does a pure monopolist determine its profit-maximizing output and price? Illustrate this on a graph, including MR, D, MC, and ATC curves, assuming the monopolist is making a profit. Why does a monopoly result in allocative inefficiency?
    \item \textbf{GDP Calculation:} Given the following data (in billions): Consumption=\$500, Net Investment=\$50, Government Purchases=\$100, Exports=\$30, Imports=\$40, Depreciation=\$20. Calculate Gross Investment ($I_g$) and GDP.
    \item \textbf{Unemployment/Inflation:} Define frictional, structural, and cyclical unemployment. What constitutes the Natural Rate of Unemployment (NRU)? If nominal GDP is \$10 trillion and the GDP price index is 200, what is real GDP?
    \item \textbf{Aggregate Expenditures Model:} If the MPS is 0.2, what is the spending multiplier? If equilibrium GDP is \$800 billion and full-employment GDP is \$900 billion, what is the size of the recessionary expenditure gap? How large an increase in autonomous spending is needed to close this gap?
    \item \textbf{AD/AS:} Explain the three reasons why the Aggregate Demand (AD) curve slopes downward. Use the AD-AS model to illustrate cost-push inflation (stagflation). What happens to the price level and real output?
    \item \textbf{Money Creation:} Assume the required reserve ratio is 20\%. If the Fed buys \$10 million worth of securities from the public, who deposit the proceeds into checking accounts, what is the maximum potential increase in the money supply?
    \item \textbf{Monetary Policy:} What are the four main tools of monetary policy? If the economy is experiencing high unemployment and slow growth, what type of monetary policy would the Fed likely implement? Explain which tool is most often used and how it would be employed in this situation. Show the effects in the money market and on AD/AS.
\end{enumerate}

\end{document}