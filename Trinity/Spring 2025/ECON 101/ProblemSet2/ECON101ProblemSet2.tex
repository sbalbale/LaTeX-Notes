\documentclass[12pt]{article}
\usepackage{amsmath, amssymb}
\usepackage{enumitem}
\usepackage{tikz}
\usepackage{geometry}
\geometry{margin=1in}

\title{Economics 101: Basic Economic Principles\\Problem Set \#2 Solutions}
\author{Sean Balbale}
\date{\today}

\begin{document}
\maketitle

\section*{Part I: Multiple Choice}

\subsection*{Problem 1: Utility and Diminishing Marginal Utility}
The utility schedule for movies is given as:
\[
    \begin{array}{c|cccccc}
        \text{Movies Consumed} & 0 & 1  & 2  & 3  & 4  & 5  \\\hline
        \text{Total Utility}   & 0 & 10 & 22 & 40 & 65 & 85
    \end{array}
\]
\textbf{Step 1.} Compute the marginal utility (MU) for each additional movie:
\begin{itemize}[noitemsep]
    \item 1st movie: \(MU = 10 - 0 = 10\)
    \item 2nd movie: \(MU = 22 - 10 = 12\)
    \item 3rd movie: \(MU = 40 - 22 = 18\)
    \item 4th movie: \(MU = 65 - 40 = 25\)
    \item 5th movie: \(MU = 85 - 65 = 20\)
\end{itemize}
\textbf{Step 2.} The consumer experiences a drop in marginal utility when the MU falls from 25 (4th movie) to 20 (5th movie). Thus, diminishing marginal utility begins at the \textbf{fifth movie}.

\textbf{Answer:} \textbf{C) fifth movie}

\subsection*{Problem 2: Cross Elasticity of Demand}
Digital cameras and memory cards are typically used together (complementary goods). When one increases in price, the demand for the other decreases. Therefore, the cross elasticity of demand is \textbf{negative}.

\textbf{Answer:} \textbf{B) A negative number}

\subsection*{Problem 3: Supply of Windows in a Multi-Product Firm}
A firm that produces both house windows and windows for other products can shift resources toward the more profitable alternative if prices change. This makes the supply of house windows more responsive (more elastic) compared to a firm that produces only house windows.

\textbf{Answer:} \textbf{C) Relatively more elastic than those of firms which only make house windows}

\subsection*{Problem 4: Price Decrease and Revenue}
A price decrease will increase total revenue if demand is \textbf{elastic} (i.e., the percentage increase in quantity demanded is larger than the percentage decrease in price).

\textbf{Answer:} \textbf{A) demand for that particular ice cream is elastic}

\subsection*{Problem 5: Price Elasticity of Demand for Tomatoes}
Given the market demand schedule:
\[
    \begin{array}{c|cccc}
        \text{Point} & \text{Price }(\$/\text{bushel}) & \text{Quantity (million bushels/year)} \\\hline
        A            & 20                              & 1                                      \\
        B            & 18                              & 3                                      \\
        C            & 16                              & 5                                      \\
        D            & 14                              & 7                                      \\
    \end{array}
\]
We are to compute the absolute value of the average (midpoint) elasticity between points B and D.

\textbf{Step 1.} Calculate the changes and averages:
\begin{align*}
    \Delta Q & = 7 - 3 = 4, \quad \text{Average } Q = \frac{7 + 3}{2} = 5,       \\
    \Delta P & = 14 - 18 = -4, \quad \text{Average } P = \frac{18 + 14}{2} = 16.
\end{align*}

\textbf{Step 2.} Use the midpoint formula for elasticity:
\[
    \text{Elasticity} = \frac{\frac{\Delta Q}{\text{Average } Q}}{\frac{\Delta P}{\text{Average } P}} = \frac{\frac{4}{5}}{\frac{4}{16}} = \frac{0.8}{0.25} = 3.2.
\]
Thus, the absolute value of the elasticity is \textbf{3.2}.

\subsection*{Problem 6: Pure Public Goods}
For a pure public good:
\begin{itemize}[noitemsep]
    \item (A) Nonexcludability is true.
    \item (C) Nonrivalry is true.
    \item (D) The free-rider problem leads to underproduction by the private sector.
\end{itemize}
However, (B) states that \textbf{all benefits} are received by the government, which is false because the benefits accrue to all members of society regardless of payment.

\textbf{Answer:} \textbf{B) All benefits associated with the production and use of a public good are received by the government}

\newpage
\section*{Part II: Short Essay}

\subsection*{Problem 1: Income and Food Expenditure}
\begin{itemize}[noitemsep]
    \item \textbf{Initial Income:} \$2,000.
    \item \textbf{Initial Food Spending:} 20\% of \$2,000 = \$400.
    \item \textbf{After Doubling Income:} New income = \$4,000.
\end{itemize}
Since the income elasticity of demand for food is 1, food expenditure will double:
\[
    \text{New Food Expenditure} = \$400 \times 2 = \$800.
\]

\subsection*{Problem 2: Mr. Ida H. O. Potato’s Choices}
\textbf{Given:}
\begin{itemize}[noitemsep]
    \item Price of bread = \$1.00 per loaf.
    \item Price of potatoes = \$1.50 per pound.
    \item Weekly allowance = \$30.
    \item Chosen bundle: 15 loaves of bread and 10 pounds of potatoes.
\end{itemize}

\subsubsection*{A. Marginal Utility Comparison}
At the optimum, the consumer equates the marginal utility per dollar:
\[
    \frac{MU_{\text{bread}}}{1.00} = \frac{MU_{\text{potatoes}}}{1.50} \quad \Longrightarrow \quad MU_{\text{potatoes}} = 1.50 \times MU_{\text{bread}}.
\]
Thus, the marginal utility of the 10th pound of potatoes is 1.5 times that of the 15th loaf of bread.

\subsubsection*{B. Impact of a Drought on the Potato Market}
A drought reduces the supply of potatoes, shifting the supply curve leftward.
\begin{itemize}[noitemsep]
    \item \textbf{Diagram Explanation:} The leftward shift in the supply curve raises the equilibrium price and reduces the equilibrium quantity.
    \item If demand is relatively elastic, the percentage decrease in quantity will be large compared to the percentage increase in price, leading to a decrease in total income for farmers.
\end{itemize}

\begin{center}
    \begin{tikzpicture}[scale=1.2]
        % Axes
        \draw[->] (0,0) -- (6,0) node[right] {Quantity};
        \draw[->] (0,0) -- (0,6) node[above] {Price};

        % Demand curve D: from (1,4.5) to (5,0.5)
        \draw[thick,blue] (1,4.5) -- (5,0.5) node[above, xshift=-0.4cm, yshift=0.5cm] {$D$};

        % Original supply curve S1: from (1,0.5) to (5,4.5)
        \draw[thick,red] (1,0.5) -- (5,4.5) node[above, xshift=0.4cm, yshift=-0.3cm] {$S_1$};

        % New supply curve S2: shifted left (from (1,1) to (4,4))
        \draw[thick,red,dashed] (1,1) -- (4,4) node[above, xshift=0.3cm, yshift=-0.3cm] {$S_2$};

        % Equilibrium E1: Intersection of D and S1 (approx. (3,2.5))
        \filldraw[black] (3,2.5) circle (2pt) node[above right, xshift=0.2cm, yshift=0.2cm] {$E_1(P_1,Q_1)$};

        % Equilibrium E2: Intersection of D and S2 (approx. (2.75,2.75))
        \filldraw[black] (2.75,2.75) circle (2pt) node[below left, xshift=-0.2cm, yshift=-0.2cm] {$E_2(P_2,Q_2)$};

        % Arrow indicating the shift from E1 to E2
        \draw[->, very thick] (3,2.5) -- (2.75,2.75) node[midway, above, xshift=-0.2cm, yshift=0.3cm] {Shift};
    \end{tikzpicture}
\end{center}


\subsubsection*{C. Consumer Response to the Price Change}
After the drought, the higher price of potatoes makes Ida’s original bundle unaffordable within her \$30 budget. As a rational consumer, Ida will substitute away from the now more expensive potatoes in favor of relatively cheaper bread.

\subsection*{Problem 3: Prospect Theory and Cereal Box Size}
According to prospect theory, consumers evaluate outcomes relative to a reference point and perceive losses (such as a direct price increase) more acutely than gains. By reducing the size of the cereal box instead of increasing its price, the cereal maker ``hides'' the cost increase as a reduction in quantity rather than a higher price. This tactic tends to be less noticeable and less painful for consumers.

\subsection*{Problem 4: Restaurant Menu Pricing Decision}
\textbf{Data:}
\begin{itemize}[noitemsep]
    \item \textbf{Steak:} Price drop of \$2.00 from \$15 results in sales increasing from 75 to 100 meals (an increase of 25 meals).
    \item \textbf{Salmon:} Price drop of \$2.50 from \$17 results in sales increasing from 40 to 75 meals (an increase of 35 meals).
\end{itemize}

\textbf{Step 1.} Compute approximate percentage changes:
\begin{itemize}[noitemsep]
    \item \textbf{Steak:}
          \[
              \text{Percentage change in price} \approx \frac{2}{15} \approx 13.3\% \quad \text{and} \quad \text{Percentage change in quantity} \approx \frac{25}{75} \approx 33.3\%.
          \]
          Elasticity for steak: \(\approx \frac{33.3}{13.3} \approx 2.5\).

    \item \textbf{Salmon:}
          \[
              \text{Percentage change in price} \approx \frac{2.5}{17} \approx 14.7\% \quad \text{and} \quad \text{Percentage change in quantity} \approx \frac{35}{40} \approx 87.5\%.
          \]
          Elasticity for salmon: \(\approx \frac{87.5}{14.7} \approx 6.0\).
\end{itemize}
Since the salmon entree has a much higher elasticity, the proportional increase in sales is greater when its price is reduced.

\textbf{Answer:} The restaurant should put the salmon entree on sale.

\subsection*{Problem 5: Ms. Sanchez’s Utility Maximization and Demand for Coke}
Ms. Sanchez has a weekly income of \$10 and consumes only two goods:
\begin{itemize}[noitemsep]
    \item \textbf{Coke:} Price = \$1 per bottle.
    \item \textbf{Pizza:} Price = \$2 per slice.
\end{itemize}

\textbf{Marginal Utility Schedules:}

\noindent\textbf{Coke:}
\[
    \begin{array}{c|cccccc}
        \text{Quantity} & 1  & 2  & 3  & 4  & 5  & 6  \\\hline
        MU              & 32 & 28 & 24 & 20 & 16 & 12
    \end{array}
\]

\noindent\textbf{Pizza:}
\[
    \begin{array}{c|cccccc}
        \text{Quantity} & 1  & 2  & 3  & 4  & 5  & 6 \\\hline
        MU              & 48 & 40 & 32 & 24 & 16 & 8
    \end{array}
\]

\subsubsection*{Part A: Optimal Consumption at Initial Prices}
\textbf{Step 1.} Compute the marginal utility per dollar:
\begin{itemize}[noitemsep]
    \item For Coke (price = \$1): MU per dollar is the same as the MU.
    \item For Pizza (price = \$2): MU per dollar is \( \frac{MU}{2} \).
\end{itemize}
The values are:
\begin{itemize}[noitemsep]
    \item \textbf{Coke:} 1st: 32, 2nd: 28, 3rd: 24, 4th: 20, 5th: 16, 6th: 12.
    \item \textbf{Pizza:} 1st: \(48/2 = 24\), 2nd: \(40/2 = 20\), 3rd: \(32/2 = 16\), 4th: \(24/2 = 12\), 5th: \(16/2 = 8\), 6th: \(8/2 = 4\).
\end{itemize}
\textbf{Step 2.} Purchase units in order of highest marginal utility per dollar until the \$10 budget is exhausted:
\begin{enumerate}[noitemsep]
    \item Buy 1st Coke: 32 (cost \$1; remaining \$9).
    \item Buy 2nd Coke: 28 (cost \$1; remaining \$8).
    \item Buy 1st Pizza: 24 (cost \$2; remaining \$6).
    \item Buy 3rd Coke: 24 (cost \$1; remaining \$5).
    \item Buy 2nd Pizza: 20 (cost \$2; remaining \$3).
    \item Buy 4th Coke: 20 (cost \$1; remaining \$2).
    \item Tie between 5th Coke (16 per dollar) and 3rd Pizza (16 per dollar); choose 5th Coke (cost \$1; remaining \$1).
    \item Buy 6th Coke: 12 (cost \$1; remaining \$0).
\end{enumerate}
\textbf{Final Bundle:} 6 bottles of Coke and 2 slices of Pizza \\
\(\text{Check: } 6 \times \$1 + 2 \times \$2 = \$6 + \$4 = \$10.\)

\subsubsection*{Part B: New Bundle When Coke Price Increases to \$2}
New prices:
\begin{itemize}[noitemsep]
    \item Coke: \$2 per bottle.
    \item Pizza: \$2 per slice.
\end{itemize}

\textbf{Recalculate MU per dollar:}
\begin{itemize}[noitemsep]
    \item \textbf{Coke:}
          \begin{itemize}[noitemsep]
              \item 1st: \(32/2 = 16\)
              \item 2nd: \(28/2 = 14\)
              \item 3rd: \(24/2 = 12\)
              \item 4th: \(20/2 = 10\)
              \item 5th: \(16/2 = 8\)
              \item 6th: \(12/2 = 6\)
          \end{itemize}
    \item \textbf{Pizza:}
          \begin{itemize}[noitemsep]
              \item 1st: \(48/2 = 24\)
              \item 2nd: \(40/2 = 20\)
              \item 3rd: \(32/2 = 16\)
              \item 4th: \(24/2 = 12\)
              \item 5th: \(16/2 = 8\)
              \item 6th: \(8/2 = 4\)
          \end{itemize}
\end{itemize}
\textbf{Purchasing Order:}
\begin{enumerate}[noitemsep]
    \item Buy 1st Pizza: 24 (cost \$2; remaining \$8).
    \item Buy 2nd Pizza: 20 (cost \$2; remaining \$6).
    \item Buy 3rd Pizza: 16 (cost \$2; remaining \$4).
    \item Buy 1st Coke: 16 (cost \$2; remaining \$2).
    \item Buy 2nd Coke: 14 (cost \$2; remaining \$0).
\end{enumerate}
\textbf{Final Bundle:} 3 slices of Pizza and 2 bottles of Coke \\
\(\text{Check: } 3 \times \$2 + 2 \times \$2 = \$6 + \$4 = \$10.\)

\subsubsection*{Part C: Constructing the Demand Curve for Coke}
From the two bundles:
\begin{itemize}[noitemsep]
    \item When \(P = \$1\), quantity of Coke = 6 bottles.
    \item When \(P = \$2\), quantity of Coke = 2 bottles.
\end{itemize}
Assuming a linear demand curve of the form:
\[
    Q = a - bP,
\]
we use the two points:
\[
    6 = a - b(1) \quad \text{and} \quad 2 = a - b(2).
\]
Subtracting the second from the first:
\[
    6 - 2 = [a - b] - [a - 2b] \quad \Longrightarrow \quad 4 = b.
\]
Substitute back to find \(a\):
\[
    6 = a - 4 \quad \Longrightarrow \quad a = 10.
\]
Thus, the demand curve for Coke is:
\[
    Q = 10 - 4P.
\]

\begin{center}
    \begin{tikzpicture}[scale=1]
        % Axes
        \draw[->] (0,0) -- (11,0) node[right] {Quantity of Coke (bottles)};
        \draw[->] (0,0) -- (0,3.5) node[above] {Price (\$)};

        % Demand curve: Q = 10 - 4P, or equivalently, P = (10-Q)/4.
        % Draw the line from (10,0) to (0,2.5)
        \draw[thick,blue] (10,0) -- (0,2.5) node[midway, above left] {Demand Curve};

        % Mark the known consumption points:
        % Original bundle: (Q=6, P=1)
        \filldraw (6,1) circle (2pt) node[below right, xshift=0.1cm, yshift=-0.1cm] {$(6,1)$};
        % New bundle: (Q=2, P=2)
        \filldraw (2,2) circle (2pt) node[above right, xshift=0.1cm, yshift=0.1cm] {$(2,2)$};

        % Optional: label intercepts
        \node at (10,0) [below] {10};
        \node at (0,2.5) [left] {2.5};
    \end{tikzpicture}
\end{center}

\section*{Part III: Long Essay}

\subsection*{Problem 1: Market for Surfboards in Southern California}
Let the initial equilibrium be \((P_1, Q_1)\).
\subsection*{(A) High Temperatures Increase Beach Attendance}
% In this case, an increase in temperature raises beach attendance. The demand curve shifts right.
\begin{center}
    \begin{tikzpicture}[scale=1.2]
        % Axes
        \draw[->] (0,0) -- (6,0) node[right]{Quantity};
        \draw[->] (0,0) -- (0,6) node[above]{Price};

        % Original demand curve D1 and supply S (unchanged)
        \draw[thick,blue] (1,4) node[left] {\(D_1\)} -- (5,1);
        \draw[thick,red] (1,1) node[left] {\(S\)} -- (5,4);

        % New demand curve D2 (shifted right)
        \draw[thick,blue,dashed] (1.5,4.5) node[left] {\(D_2\)} -- (5.5,1.5);

        % Equilibrium points
        \filldraw (3,2.5) circle (2pt) node[above left,xshift=0.2cm] {\(E_1(P_1,Q_1)\)};
        \filldraw (3.6,2.95) circle (2pt) node[below right,xshift=0.2cm] {\(E_2(P_2,Q_2)\)};
    \end{tikzpicture}
\end{center}
\textbf{Explanation:} Higher temperatures increase beach attendance and thus shift the demand curve rightward, leading to a higher equilibrium price and quantity.

\subsection*{(B) Increase in Epoxy Paint Price}
% Here, higher input costs shift the supply curve left.
\begin{center}
    \begin{tikzpicture}[scale=1.2]
        % Axes
        \draw[->] (0,0) -- (6,0) node[right]{Quantity};
        \draw[->] (0,0) -- (0,6) node[above]{Price};

        % Demand remains unchanged.
        \draw[thick,blue] (1,4) -- (5,1) node[midway,above left] {\(D\)};

        % Original supply curve S1 and new supply curve S2 shifted left.
        \draw[thick,red] (1,1) -- (5,4) node[below right] {\(S_1\)};
        \draw[thick,red,dashed] (1,2) -- (4,4) node[below right] {\(S_2\)};

        % Equilibrium points
        \filldraw (3,2.5) circle (2pt) node[below left,xshift=0.2cm] {\(E_1(P_1,Q_1)\)};
        \filldraw (2.4,2.9) circle (2pt) node[above left,xshift=-0.2cm] {\(E_2(P_2,Q_2)\)};
    \end{tikzpicture}
\end{center}
\textbf{Explanation:} An increase in the price of epoxy paint (a key input) raises production costs, shifting the supply curve leftward. The new equilibrium has a higher price and a lower quantity.

\subsection*{(C) Beach Admission Fees}
% Charging fees reduces beach attendance, shifting demand left.
\begin{center}
    \begin{tikzpicture}[scale=1.2]
        % Axes
        \draw[->] (0,0) -- (6,0) node[right]{Quantity};
        \draw[->] (0,0) -- (0,6) node[above]{Price};

        % Demand remains unchanged.
        \draw[thick,blue] (1,4) -- (5,1) node[midway,above left] {\(D\)};

        % Original supply curve S1 and new supply curve S2 shifted left.
        \draw[thick,red] (1,1) -- (5,4) node[below right] {\(S_1\)};
        \draw[thick,red,dashed] (1,2) -- (4,4) node[below right] {\(S_2\)};

        % Equilibrium points
        \filldraw (3,2.5) circle (2pt) node[below left,xshift=0.2cm] {\(E_1(P_1,Q_1)\)};
        \filldraw (2.4,2.9) circle (2pt) node[above left,xshift=-0.2cm] {\(E_2(P_2,Q_2)\)};
    \end{tikzpicture}
\end{center}
\textbf{Explanation:} Imposing admission fees increase the cost of supply, shifting the supply curve leftward. This results in a lower equilibrium price and quantity in the surfboard market.

\subsection*{(D) Expected Bad Weather}
% Expected bad weather makes surfing less attractive, shifting demand left.
\begin{center}
    \begin{tikzpicture}[scale=1.2]
        % Axes
        \draw[->] (0,0) -- (6,0) node[right]{Quantity};
        \draw[->] (0,0) -- (0,6) node[above]{Price};

        % Original demand curve D1 and supply S (unchanged)
        \draw[thick,blue,dashed] (1,4) node[left] {\(D_2\)} -- (5,1);
        \draw[thick,red] (1,1) node[left] {\(S\)} -- (5,4);

        % New demand curve D2 (shifted right)
        \draw[thick,blue] (1.5,4.5) node[left] {\(D_1\)} -- (5.5,1.5);

        % Equilibrium points
        \filldraw (3,2.5) circle (2pt) node[above left,xshift=0.2cm] {\(E_2(P_1,Q_1)\)};
        \filldraw (3.6,2.95) circle (2pt) node[below right,xshift=0.2cm] {\(E_(P_2,Q_2)\)};
    \end{tikzpicture}
\end{center}
\textbf{Explanation:} Anticipated bad weather reduces the attractiveness of surfing, causing the demand curve to shift leftward. The resulting new equilibrium has both a lower price and a lower quantity.

\subsection*{(E) Positive Externalities from Surfing}
% In this diagram, the private demand is lower than the social marginal benefit.
\begin{center}
    \begin{tikzpicture}[scale=1.2]
        % Axes
        \draw[->] (0,0) -- (6,0) node[right]{Quantity};
        \draw[->] (0,0) -- (0,6) node[above]{Price};

        % Original demand curve D1 and supply S (unchanged)
        \draw[thick,blue] (1,4) node[left] {\(D_p\)} -- (5,1);
        \draw[thick,red] (1,1) node[left] {\(S\)} -- (5,4);

        % New demand curve D2 (shifted right)
        \draw[thick,blue,dashed] (1.5,4.5) node[left] {\(D_s\)} -- (5.5,1.5);

        % Equilibrium points
        \filldraw (3,2.5) circle (2pt) node[above left,xshift=0.2cm] {\(E_1(P_1,Q_1)\)};
        \filldraw (3.6,2.95) circle (2pt) node[below right,xshift=0.2cm] {\(E_2(P_*,Q_*)\)};
    \end{tikzpicture}
\end{center}
\textbf{Explanation:} Research shows that surfing provides positive externalities, meaning the social benefits exceed the private benefits. The social demand curve (\(D_s\)) is shifted right relative to the private demand curve (\(D_p\)), indicating that the socially optimal level of surfboards (\(Q^*\)) is higher than the market equilibrium quantity \(Q_1\). This implies that the private market underproduces surfboards relative to the allocatively efficient level.

\subsection*{Problem 2: Extra Credit --- Gasoline Prices and Demand Over Time}
\subsection*{Short-Run Demand Curve}
\begin{tikzpicture}[scale=1.2]
    % Axes
    \draw[->] (0,0) -- (6,0) node[right]{Quantity};
    \draw[->] (0,0) -- (0,5) node[above]{Price};

    % Short-run demand curve (steep: inelastic demand)
    \draw[thick,blue] (1,4.5) -- (5,0.5) node[midway, above, sloped, xshift=-0.2cm] {\small Short-Run Demand};

    % Mark endpoints (optional)
    \node at (1,4.5) [left] {A};
    \node at (5,0.5) [right] {B};
\end{tikzpicture}

\bigskip

\subsection*{Long-Run Demand Curve}
\begin{tikzpicture}[scale=1.2]
    % Axes
    \draw[->] (0,0) -- (6,0) node[right]{Quantity};
    \draw[->] (0,0) -- (0,5) node[above]{Price};

    % Long-run demand curve (flatter: more elastic demand)
    \draw[thick,red] (1,3.5) -- (5,0.5) node[midway, above, sloped, xshift=0.2cm] {\small Long-Run Demand};

    % Mark endpoints (optional)
    \node at (1,3.5) [left] {C};
    \node at (5,0.5) [right] {D};
\end{tikzpicture}

\bigskip

\textbf{Explanation:}
In the short run, consumers have few substitutes and are less responsive to price changes—reflected in a steep demand curve (Diagram 1). Over time, as consumers adjust their behavior (by using alternatives such as public transportation, carpooling, or switching to fuel-efficient vehicles), they become more responsive to changes in gasoline prices. This behavior is illustrated by a flatter demand curve in the long run (Diagram 2). Consequently, even amid supply shortages, the long-run adjustment in demand can lead to a lower equilibrium price as consumers reduce their gasoline consumption more significantly when prices rise.


\end{document}
