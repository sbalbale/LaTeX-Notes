\documentclass[reqno, 12pt]{amsart}

\usepackage{amssymb, amsmath, amsthm, enumerate, mathtools, graphicx, enumitem, tikz, color, soul, bbm, verbatim, parskip, multicol}
\usepackage[margin=1 in]{geometry}
\usepackage[mathscr]{euscript}
\usepackage[urlcolor=blue,colorlinks=true]{hyperref}
\setlist[itemize]{noitemsep, topsep=0pt, parsep=0pt, partopsep=0pt}
\allowdisplaybreaks

\newcommand{\R}{\mathbb R}
\newcommand{\proj}{\operatorname{proj}}

\pagestyle{plain}  


\begin{document}

\begin{center}
    {\bf MATH 231-01: Homework Assignment 2}\\~\\
    15 September 2025\\~\\~\\~\\~\\~\\
\end{center}

{\bf Due:} 22 September 2025 by 10:00pm Eastern time, submitted on Moodle as a single PDF.~\\


{\bf Instructions:} Write your solutions on the following pages. If you need more space, you may add pages, but make sure they are in order and label the problem number(s) clearly. You should attempt each problem on scrap paper first, before writing your solution here. Excessively messy or illegible work will not be graded. You must show your work/reasoning to receive credit. You do not need to include every minute detail; however the process by which you reached your answer should be evident. You may work with other students, but please write your solutions in your own words.

~\\~\\~\\~\\~\\
{\bf Name:} Sean Balbale

~\\
{\bf Score:}

\newpage
\begin{itemize}
    \item[1.] Use the cross product to find the area of the triangle with vertices $A = (3,0,0)$, $B = (3,3,0)$, and $C = (0,1,3)$.
          \newline


          \noindent\fbox{%
              \parbox{\dimexpr\linewidth-2\fboxsep-2\fboxrule}{
                  Let $\vec{AB} = B - A = \langle 0,3,0 \rangle$ and $\vec{AC} = C - A = \langle -3,1,3 \rangle$. Then the area of the triangle is given by
                  \[
                      \text{Area} = \frac{1}{2} \| \vec{AB} \times \vec{AC} \|.
                  \]
                  Calculating the cross product, we have
                  \begin{align*}
                      \vec{AB} \times \vec{AC} = \begin{vmatrix}
                                                     \vec{i} & \vec{j} & \vec{k} \\
                                                     0       & 3       & 0       \\
                                                     -3      & 1       & 3
                                                 \end{vmatrix} & = (3 \cdot 3 - 0 \cdot 1)\vec{i} - (0 \cdot 3 - 0 \cdot -3)\vec{j} + (0 \cdot 1 - 3 \cdot -3)\vec{k} \\
                                                                 & = (9\vec{i} + 0\vec{j} + 9\vec{k})                                                                 \\
                                                                 & = \langle 9,0,9 \rangle.
                  \end{align*}

                  Thus,
                  \[
                      \text{Area} = \frac{1}{2} \sqrt{9^2 + 0^2 + 9^2} = \frac{1}{2} \sqrt{162} = \frac{9\sqrt{2}}{2}.
                  \]
              }%
          }

          \vspace{0.5 in}

    \item[2.] Suppose ${\bf u}$, ${\bf v}$, and ${\bf w}$ are unit vectors in $\R^3$ such that ${\bf v}$ and ${\bf w}$ make an angle of $\pi/6$, and ${\bf u}$ and ${\bf v}\times{\bf w}$ make an angle of $2\pi/3$ (both in radians). Find the volume of the parallelepiped generated by ${\bf u}$, ${\bf v}$, and ${\bf w}$.
          \newline

          \noindent\fbox{%
              \parbox{\dimexpr\linewidth-2\fboxsep-2\fboxrule}{

                  The volume of the parallelepiped generated by vectors $\vec{u}$, $\vec{v}$, and $\vec{w}$ is given by the scalar triple product:
                  \[
                      \text{Volume} = |\vec{u} \cdot (\vec{v} \times \vec{w})|.
                  \]
                  We know that $\|\vec{v}\| = 1$ and $\|\vec{w}\| = 1$, and the angle between them is $\pi/6$. Thus, we can find the magnitude of their cross product:
                  \[
                      \|\vec{v} \times \vec{w}\| = \|\vec{v}\| \|\vec{w}\| \sin\left(\frac{\pi}{6}\right) = 1 \cdot 1 \cdot \frac{1}{2} = \frac{1}{2}.
                  \]
                  Next, we need to find the angle between $\vec{u}$ and $\vec{v} \times \vec{w}$. We know that $\|\vec{u}\| = 1$ and the angle between them is $2\pi/3$. Thus,
                  \[
                      |\vec{u} \cdot (\vec{v} \times \vec{w})| = \|\vec{u}\| \|\vec{v} \times \vec{w}\| \cos\left(\frac{2\pi}{3}\right) = 1 \cdot \frac{1}{2} \cdot \left(-\frac{1}{2}\right) = -\frac{1}{4}.
                  \]
                  Therefore, the volume of the parallelepiped is
                  \[
                      \text{Volume} = \left| -\frac{1}{4} \right| = \frac{1}{4}.
                  \]
              }%
          }
          \vspace{0.5 in}

          \newpage
    \item[3.] Solve Problem 60 on p.~825 of the textbook (Section 10.4).\\
          Turning a bolt with a wrench produces a \textbf{torque} vector that drives the bolt forward. The magnitude of the torque vector is $\|\mathbf{r}\| \|\mathbf{F}\| \sin\theta$, where $\mathbf{r}$ is the vector along the handle of the wrench, $\mathbf{F}$ is the force vector applied to the handle of the wrench, and $\theta$ is the angle between these two vectors. Therefore, the magnitude of the torque is $\|\mathbf{r} \times \mathbf{F}\|$. Find the magnitude of the torque. Express each answer in foot-pounds.\\
          \textbf{60.} A force of 20 lb is applied to a wrench with a 6-inch handle at an angle of $60^\circ$.
          \newline

          \noindent\fbox{%
              \parbox{\dimexpr\linewidth-2\fboxsep-2\fboxrule}{
                  The torque is given by
                  \[
                      \|\mathbf{r} \times \mathbf{F}\| = \|\mathbf{r}\| \|\mathbf{F}\| \sin\theta.
                  \]
                  Here, $\|\mathbf{r}\| = 6$ inches $= 0.5$ feet, $\|\mathbf{F}\| = 20$ lb, and $\theta = 60^\circ$. Thus,
                  \[
                      \|\mathbf{r} \times \mathbf{F}\| = 0.5 \cdot 20 \cdot \sin(60^\circ) = 10 \cdot \frac{\sqrt{3}}{2} = 5\sqrt{3} \text{ ft-lb}.
                  \]
              }%
          }

          \vspace{0.5 in}
    \item[4.] Prove that the cross product is anticommutative. In other words, given arbitrary vectors ${\bf u} = \langle u_1,u_2,u_3\rangle$ and ${\bf v} = \langle v_1,v_2,v_3\rangle$, use the definition of the cross product to show that ${\bf v} \times {\bf u} = -({\bf u} \times {\bf v})$.
          \newline

          \noindent\fbox{%
              \parbox{\dimexpr\linewidth-2\fboxsep-2\fboxrule}{
                  Starting with the definition of the cross product:
                  \[
                      \vec{u} \times \vec{v} = \begin{vmatrix}
                          \vec{i} & \vec{j} & \vec{k} \\
                          u_1     & u_2     & u_3     \\
                          v_1     & v_2     & v_3
                      \end{vmatrix}.
                  \]
                  Calculating this determinant,
                  \begin{align*}
                      \vec{u} \times \vec{v} & = (u_2 v_3 - u_3 v_2)\vec{i} - (u_1 v_3 - u_3 v_1)\vec{j} + (u_1 v_2 - u_2 v_1)\vec{k} \\
                                             & = (u_2 v_3 - u_3 v_2, -(u_1 v_3 - u_3 v_1), u_1 v_2 - u_2 v_1).
                  \end{align*}
                  Now, computing $\vec{v} \times \vec{u}$:
                  \[
                      \vec{v} \times \vec{u} = \begin{vmatrix}
                          \vec{i} & \vec{j} & \vec{k} \\
                          v_1     & v_2     & v_3     \\
                          u_1     & u_2     & u_3
                      \end{vmatrix}.
                  \]
                  Calculating this determinant,
                  \begin{align*}
                      \vec{v} \times \vec{u} & = (v_2 u_3 - v_3 u_2)\vec{i} - (v_1 u_3 - v_3 u_1)\vec{j} + (v_1 u_2 - v_2 u_1)\vec{k} \\
                                             & = (v_2 u_3 - v_3 u_2, -(v_1 u_3 - v_3 u_1), v_1 u_2 - v_2 u_1).
                  \end{align*}
                  Notice that each component of $\vec{v} \times \vec{u}$ is the negative of the corresponding component of $\vec{u} \times \vec{v}$. Specifically,
                  \[
                      v_2 u_3 - v_3 u_2 = -(u_2 v_3 - u_3 v_2),
                  \]
                  \[
                      -(v_1 u_3 - v_3 u_1) = -(-(u_1 v_3 - u_3 v_1)) = u_1 v_3 - u_3 v_1,
                  \]
                  \[            v_1 u_2 - v_2 u_1 = -(u_1 v_2 - u_2 v_1).
                  \]
                  Therefore,
                  \[
                      \vec{v} \times \vec{u} = -(\vec{u} \times \vec{v}).
                  \]

              }%
          }
          \vspace{0.5 in}
          \newpage
    \item[5.] For many years, I've been trying to prove that $0 = 1$. Here's my latest attempt:\\

          Let ${\bf i}$, ${\bf j}$, and ${\bf k}$ denote the standard basis vectors in $\R^3$. Then
          \begin{align*}
              {\bf 0} = {\bf i} \times {\bf i} = ({\bf i} \times {\bf i}) \times {\bf j} = {\bf i} \times ({\bf i} \times {\bf j}) = {\bf i} \times {\bf k} = -{\bf j}.
          \end{align*}
          Since $\|{\bf 0}\| = 0$ and $\|-{\bf j}\| = 1$, it follows that $0 = 1$. \qed\\

          Where did I go wrong? Explain.
          \newline

          \noindent\fbox{%
              \parbox{\dimexpr\linewidth-2\fboxsep-2\fboxrule}{
                  The mistake was made during step 3, This step incorrectly assumes that the cross product is associative.
                  The cross product is not associative; in general, $(\vec{a} \times \vec{b}) \times \vec{c} \neq \vec{a} \times (\vec{b} \times \vec{c})$. \\
                  Breaking down the steps:
                  \begin{enumerate}
                      \item $\vec{0} = \vec{i} \times \vec{i}$ is correct.
                      \item $(\vec{i} \times \vec{i}) \times \vec{j} = \vec{0} \times \vec{j} = \vec{0}$ is also correct.
                      \item The error occurs here: $\vec{i} \times (\vec{i} \times \vec{j})$ is not equal to $(\vec{i} \times \vec{i}) \times \vec{j}$. Instead, we need to compute $\vec{i} \times (\vec{i} \times \vec{j})$ directly.
                      \item Continuing from step 3, we find $\vec{i} \times (\vec{i} \times \vec{j}) = \vec{i} \times \vec{k} = -\vec{j}$ is correct, but it does not follow from the previous steps.
                  \end{enumerate}
                  Therefore, the conclusion that $\vec{0} = -\vec{j}$ is incorrect, and the proof that $0 = 1$ is invalid.

              }%
          }
          \vspace{0.5 in}
                  
    \item[6.] Let $L_1$ be the line with vector equation
          \begin{align*}
              \langle x,y,z\rangle = \langle 1,0,1\rangle + t\langle 2,-1,0\rangle.
          \end{align*}
          Let $L_2$ be the line with symmetric equations
          \begin{align*}
              \frac{x+1}{3} = \frac{1-y}{2} = z-1.
          \end{align*}
          \begin{enumerate}
              \item[(a)]{Show that $L_1$ and $L_2$ intersect.}
                    \newline

                    \noindent\fbox{%
                        \parbox{\dimexpr\linewidth-2\fboxsep-2\fboxrule}{
                            To find the intersection of $L_1$ and $L_2$, first express $L_2$ in parametric form. Letting the common ratio be $s$,
                            \[
                                x = 3s - 1, \quad y = 1 - 2s, \quad z = s + 1.
                            \]
                            Now, set the parametric equations of $L_1$ equal to those of $L_2$:
                            \[
                                1 + 2t = 3s - 1, \quad -t = 1 - 2s, \quad 1 = s + 1.
                            \]
                            From the third equation, $s = 0$. Substituting $s = 0$ into the second equation gives:
                            \[
                                -t = 1 - 0 \implies t = -1.
                            \]
                            Substituting $t = -1$ into the first equation gives:
                            \[
                                1 + 2(-1) = 3(0) - 1 \implies -1 = -1,
                            \]
                            which is true. Thus, the lines intersect at the point obtained by substituting $t = -1$ into $L_1$:
                            \[
                                (x,y,z) = (1 + 2(-1), 0 - 1(-1), 1 + 0) = (-1, 1, 1).
                            \]
                            Therefore, $L_1$ and $L_2$ intersect at the point $(-1, 1, 1)$.
                        }%
                    }
                    \vspace{0.5 in}
              \item[(b)]{Find an equation for the plane that contains $L_1$ and $L_2$.}
                    \newline

                    \noindent\fbox{%
                        \parbox{\dimexpr\linewidth-2\fboxsep-2\fboxrule}{
                            To find the equation of the plane containing $L_1$ and $L_2$, a point on the plane and a normal vector to the plane are needed. The point of intersection is $P = (-1, 1, 1)$.

                            Next, find direction vectors for both lines. The direction vector for $L_1$ is $\vec{d_1} = \langle 2, -1, 0 \rangle$. For $L_2$, use the parametric form to find its direction vector $\vec{d_2} = \langle 3, -2, 1 \rangle$.

                            The normal vector $\vec{n}$ to the plane is found by taking the cross product of $\vec{d_1}$ and $\vec{d_2}$:
                            \[
                                \vec{n} = \vec{d_1} \times \vec{d_2} = \begin{vmatrix}
                                    \vec{i} & \vec{j} & \vec{k} \\
                                    2       & -1      & 0       \\
                                    3       & -2      & 1
                                \end{vmatrix}.
                            \]
                            Calculating this determinant:
                            \begin{align*}
                                \vec{n} & = (-1)(1) - (0)(-2)\vec{i} - (2)(1) - (0)(3)\vec{j} + (2)(-2) - (-1)(3)\vec{k} \\
                                        & = (-1)\vec{i} - (2)\vec{j} + (-4 + 3)\vec{k}                                   \\
                                        & = \langle -1, -2, -1 \rangle.
                            \end{align*}

                            With a normal vector $\vec{n} = \langle -1, -2, -1 \rangle$ and a point $P = (-1, 1, 1)$, the equation of the plane can be written as:
                            \[
                                -1(x + 1) - 2(y - 1) - 1(z - 1) = 0.
                            \]
                            After simplifying:
                            \[
                                -x - 1 - 2y + 2 - z + 1 = 0 \implies x + 2y + z = 2.
                            \]

                        }%
                    }
                    \vspace{0.5 in}
          \end{enumerate}

          \vspace{4 in}
    \item[7.] Find an equation for the line that contains the point $P = (2,0,1)$ and is parallel to to the planes given by the equations $x+2y+3z = 6$ and $2x-y+z = 4$.
          \newline

          \noindent\fbox{%
              \parbox{\dimexpr\linewidth-2\fboxsep-2\fboxrule}{
                  To find the equation of the line that contains the point $P = (2,0,1)$ and is parallel to the planes given by the equations $x + 2y + 3z = 6$ and $2x - y + z = 4$, we first need to determine the normal vectors of these planes.

                  The normal vector of the first plane $x + 2y + 3z = 6$ is $\vec{n_1} = \langle 1, 2, 3 \rangle$. The normal vector of the second plane $2x - y + z = 4$ is $\vec{n_2} = \langle 2, -1, 1 \rangle$.

                  The direction vector $\vec{d}$ of the line we are looking for can be found by taking the cross product of the two normal vectors:
                  \[
                      \vec{d} = \vec{n_1} \times \vec{n_2} = \begin{vmatrix}
                          \vec{i} & \vec{j} & \vec{k} \\
                          1       & 2       & 3       \\
                          2       & -1      & 1
                      \end{vmatrix}.
                  \]
                  Calculating this determinant:
                  \begin{align*}
                      \vec{d} & = (2)(1) - (3)(-1)\vec{i} - (1)(1) - (3)(2)\vec{j} + (1)(-1) - (2)(2)\vec{k} \\
                              & = (2 + 3)\vec{i} - (1 - 6)\vec{j} + (-1 - 4)\vec{k}                          \\
                              & = \langle 5, 5, -5 \rangle.
                  \end{align*}

                  Now that we have the direction vector $\vec{d} = \langle 5, 5, -5 \rangle$ and a point $P = (2,0,1)$ on the line, we can write the parametric equations of the line:
                  \[
                      x = 2 + 5t, \quad y = 0 + 5t, \quad z = 1 - 5t.
                  \]
                  Therefore, the vector equation of the line is
                  \[
                      \vec{r}(t) = \langle 2, 0, 1 \rangle + t \langle 5, 5, -5 \rangle.
                  \]
              }%
          }
          \vspace{0.5 in}
          \newpage
    \item[8.] Let $A = (-1,0,2)$ and $B = (0,3,0)$. The collection of all points $P = (x,y,z)$ with equal distance to $A$ and $B$ forms a plane. Find an equation of the form
          \begin{align*}
              a(x-x_0)+b(y-y_0) + c(z-z_0) = 0
          \end{align*}
          that represents this plane.
          \newline

          \noindent\fbox{%
              \parbox{\dimexpr\linewidth-2\fboxsep-2\fboxrule}{
                  To find the equation of the plane that is equidistant from points $A$ and $B$, we can use the fact that the set of points equidistant from two points is the perpendicular bisector of the segment connecting the two points.

                  The midpoint $M$ of the segment $AB$ is given by:
                  \[
                      M = \left( \frac{-1 + 0}{2}, \frac{0 + 3}{2}, \frac{2 + 0}{2} \right) = \left( -\frac{1}{2}, \frac{3}{2}, 1 \right).
                  \]

                  The direction vector $\vec{d}$ from $A$ to $B$ is:
                  \[
                      \vec{d} = B - A = (0 - (-1), 3 - 0, 0 - 2) = \langle 1, 3, -2 \rangle.
                  \]

                  A normal vector to the plane is given by the direction vector $\vec{d}$. Therefore, the equation of the plane can be written as:
                  \[
                      1\left(x + \frac{1}{2}\right) + 3\left(y - \frac{3}{2}\right) - 2(z - 1) = 0.
                  \]
              }%
          }


\end{itemize}

\end{document}







