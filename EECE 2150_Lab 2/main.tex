\documentclass{article}
\usepackage{graphicx}
\usepackage{float}
\usepackage{booktabs}

\title{Lab 2: Getting started with Ohm's Law, KVL, KCL, and Multi-Meter
Measurements}
\author{Sean Balbale}
\date{September 9th, 2023}

\begin{document}

\begin{titlepage}
    \begin{center}
        \vspace*{1in}
            
        \Huge
        \textbf{Lab 2}
            
        \LARGE
        Getting started with Ohm’s Law, KVL, KCL, and Multi-Meter Measurements
            
        \vspace{3 in}
            
        \textbf{Student Name:} Sean Balbale
        \\ \textbf{Instructor:} Dr. Iman Salama
        \\ \textbf{Lab Partner Name:} Krish Gupta
        \\ \textbf{Date:} September 9th, 2024

        \vfill
            
            
    \end{center}
\end{titlepage}

\newpage 


\section{Introduction}
The lab had a few primary purposes: practicing Ohm’s law, 
KVL (Kirchhoff's voltage law), and KCL (Kirchhoff's current law), 
and learning how to use a Multimeter. The Keysight power supply 
and digital multimeter were used in this lab, which taught a 
necessary skillset for all electrical engineers.

\section{Results}

\subsection{From Circuit Diagrams to Protoboards}



\subsection{Building a Simple LED Circuit}


\section{Discussions and Conclusions}


\section{References}
[1] Dr. Iman Salama. “Lab 2 – Getting started with Ohm’s Law, KVL, KCL, 
and Multi-Meter Measurements” Northeastern University. 9 September 2024.

\end{document}
