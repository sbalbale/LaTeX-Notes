\documentclass{article}
\usepackage{amsmath, amssymb, graphicx}

\title{Economics 101: Problem Set 2 Solutions}
\author{Sean Balbale}
\date{\today}

\begin{document}
\maketitle

\section*{Part I: Multiple Choice}

\begin{enumerate}
    \item \textbf{Answer: B) fourth movie}

          \textbf{Explanation:}

          Diminishing marginal utility occurs when the additional satisfaction from consuming one more unit of a good decreases. In the table, the marginal utility (MU) for each movie is as follows:

          \begin{itemize}
              \item MU1 = 10
              \item MU2 = 12
              \item MU3 = 18
              \item MU4 = 25
              \item MU5 = 20
          \end{itemize}

          As you can see, the marginal utility starts decreasing when the consumer watches the fourth movie.

    \item \textbf{Answer: B) A negative number}
          \textbf{Explanation:}

          Cross elasticity of demand measures the responsiveness of the quantity demanded of one good to a change in the price of another good. Digital cameras and memory cards are complementary goods, meaning they are used together. When the price of digital cameras increases, the demand for memory cards is likely to decrease, and vice versa. Therefore, the cross elasticity of demand between these two goods is negative.

    \item \textbf{Answer: C) Relatively more elastic than those of firms which only make house windows}

          \textbf{Explanation:}

          A firm that produces windows for various purposes has more flexibility in allocating its resources. If the demand for house windows decreases, it can shift its production towards other types of windows. This makes its supply curve for house windows more elastic compared to a firm that solely focuses on house windows.

    \item \textbf{Answer: A) demand for that particular ice cream is elastic}

          \textbf{Explanation:}

          When demand is elastic, a decrease in price leads to a more than proportionate increase in quantity demanded, resulting in higher total revenue. On the other hand, when demand is inelastic, a decrease in price leads to a less than proportionate increase in quantity demanded, resulting in lower total revenue.

    \item \textbf{Answer:}

          The absolute value of the average price elasticity of demand between points B and D is \textbf{1}.

          \textbf{Explanation:}

          The midpoint formula for calculating price elasticity of demand is:

          \begin{equation*}
              |Price Elasticity| = \frac{|Q2 - Q1| / [(Q1 + Q2) / 2]}{|P2 - P1| / [(P1 + P2) / 2]}
          \end{equation*}

          Using the values from the table:

          \begin{align*}
              |Price Elasticity| & = \frac{|5 - 3| / [(3 + 5) / 2]}{|16 - 18| / [(18 + 16) / 2]} \\
                                 & = \frac{2 / 4}{2 / 17}                                        \\
                                 & = 1
          \end{align*}

    \item \textbf{Answer: B) All benefits associated with the production and use of a public good are received by the government.}

          \textbf{Explanation:}

          Public goods are non-excludable and non-rivalrous. This means that it is impossible to prevent anyone from benefiting from them, and one person's consumption does not diminish the amount available to others. While the government often provides public goods, the benefits are enjoyed by the entire society, not just the government.
\end{enumerate}

\section*{Part II: Short Essay}

\begin{enumerate}
    \item \textbf{Answer:}

          If the consumer's income doubles to \$4,000, she will spend \textbf{\$800} on food.

          \textbf{Explanation:}

          The income elasticity of demand for food is 1, which means that a 1\% increase in income leads to a 1\% increase in the quantity demanded of food. Since the consumer initially spends 20\% of her income on food, when her income doubles, her spending on food will also double. Therefore, she will spend 20\% of \$4,000, which is \$800, on food.

    \item
          \begin{enumerate}
              \item \textbf{Answer:}

                    Ida's marginal utility associated with the 10th pound of potatoes is equal to his marginal utility associated with the 15th loaf of bread.

                    \textbf{Explanation:}

                    Consumers maximize their utility by allocating their budget in a way that the marginal utility per dollar spent is equal across all goods. In Ida's case, he has allocated his budget so that the marginal utility per dollar spent on potatoes is equal to the marginal utility per dollar spent on bread.

              \item \textbf{Answer:}

                    The drought will cause the supply curve for potatoes to shift to the left, resulting in a higher equilibrium price and a lower equilibrium quantity. If the demand for potatoes is relatively elastic, the farmers' total income will decrease.

                    \textbf{Explanation:}

                    A drought will reduce the supply of potatoes, leading to a higher price. If the demand for potatoes is elastic, the increase in price will cause a more than proportionate decrease in quantity demanded, resulting in a decrease in total revenue for the farmers.

              \item \textbf{Answer:}

                    No, Ida's original bundle will not be affordable after the drought. He will likely consume less of the good that has become relatively more expensive (potatoes) and more of the good that has become relatively cheaper (bread).

                    \textbf{Explanation:}

                    The drought has increased the price of potatoes, making Ida's original bundle unaffordable. As a rational consumer, Ida will adjust his consumption to maximize his utility within his budget constraint. He will likely consume fewer potatoes and more bread, as bread has become relatively cheaper compared to potatoes.
          \end{enumerate}

    \item \textbf{Answer:}

          Prospect theory suggests that people are more sensitive to losses than to gains. Reducing the size of the cereal box is perceived as a less significant loss compared to increasing the price.

          \textbf{Explanation:}

          Prospect theory suggests that consumers frame decisions in terms of gains and losses relative to a reference point. Reducing the size of the cereal box is perceived as a smaller loss compared to increasing the price, even if the effective change in value is the same. This is because consumers are more likely to notice and react negatively to a price increase, which is framed as a loss, than to a reduction in quantity, which is framed as a smaller loss or even a non-event.

    \item \textbf{Answer:}

          The restaurant owner should choose to put the \textbf{salmon entr\'ee} on sale.

          \textbf{Explanation:}

          To determine which entr\'ee would generate more revenue with a price decrease, we need to compare the price elasticity of demand for each entr\'ee. Using the midpoint formula, we can calculate the price elasticity of demand for each entr\'ee:

          \begin{itemize}
              \item \textbf{Steak:}
                    \begin{equation*}
                        |Price Elasticity| = \frac{|100 - 75| / [(75 + 100) / 2]}{|13 - 15| / [(15 + 13) / 2]} = 1.54
                    \end{equation*}

              \item \textbf{Salmon:}
                    \begin{equation*}
                        |Price Elasticity| = \frac{|75 - 40| / [(40 + 75) / 2]}{|14.5 - 17| / [(17 + 14.5) / 2]} = 2.33
                    \end{equation*}
          \end{itemize}

          The salmon entr\'ee has a higher price elasticity of demand, meaning that a price decrease will lead to a more than proportionate increase in quantity demanded, resulting in higher total revenue. Therefore, the restaurant owner should choose to put the salmon entr\'ee on sale.

    \item
          \begin{enumerate}
              \item \textbf{Answer:}

                    Ms. Sanchez will buy \textbf{2 bottles of coke} and \textbf{4 slices of pizza} each week to maximize her utility.

                    \textbf{Explanation:}

                    To maximize utility, Ms. Sanchez should allocate her budget so that the marginal utility per dollar spent is equal across both goods. We can calculate the marginal utility per dollar spent for each good by dividing the marginal utility by the price:

                    \begin{itemize}
                        \item \textbf{Coke:}
                              \begin{align*}
                                  MU per dollar (bottle 1) & = 32 / 1 = 32 \\
                                  MU per dollar (bottle 2) & = 28 / 1 = 28 \\
                                  MU per dollar (bottle 3) & = 24 / 1 = 24 \\
                                  MU per dollar (bottle 4) & = 20 / 1 = 20 \\
                                  MU per dollar (bottle 5) & = 16 / 1 = 16 \\
                                  MU per dollar (bottle 6) & = 12 / 1 = 12
                              \end{align*}

                        \item \textbf{Pizza:}
                              \begin{align*}
                                  MU per dollar (slice 1) & = 48 / 2 = 24 \\
                                  MU per dollar (slice 2) & = 40 / 2 = 20 \\
                                  MU per dollar (slice 3) & = 32 / 2 = 16 \\
                                  MU per dollar (slice 4) & = 24 / 2 = 12 \\
                                  MU per dollar (slice 5) & = 16 / 2 = 8  \\
                                  MU per dollar (slice 6) & = 8 / 2 = 4
                              \end{align*}
                    \end{itemize}

                    Ms. Sanchez will continue to consume coke and pizza until the marginal utility per dollar spent is equal for both goods. This occurs when she consumes 2 bottles of coke and 4 slices of pizza. At this point, the MU per dollar for both coke and pizza is 28.

              \item \textbf{Answer:}

                    If the price of coke increases to \$2, Ms. Sanchez will buy \textbf{1 bottle of coke} and \textbf{4 slices of pizza} to maximize her utility.

                    \textbf{Explanation:}

                    With the new price of coke, we need to recalculate the MU per dollar for coke:

                    \begin{align*}
                        MU per dollar (bottle 1) & = 32 / 2 = 16 \\
                        MU per dollar (bottle 2) & = 28 / 2 = 14 \\
                        MU per dollar (bottle 3) & = 24 / 2 = 12 \\
                        MU per dollar (bottle 4) & = 20 / 2 = 10 \\
                        MU per dollar (bottle 5) & = 16 / 2 = 8  \\
                        MU per dollar (bottle 6) & = 12 / 2 = 6
                    \end{align*}

                    Now, Ms. Sanchez will maximize her utility by consuming 1 bottle of coke and 4 slices of pizza. At this point, the MU per dollar for both coke and pizza is 16.

              \item \textbf{Answer:}

                    Based on the information from parts A and B, we can construct Ms. Sanchez's demand curve for coke as follows:

                    \begin{center}
                        \begin{tabular}{|c|c|}
                            \hline
                            Price of Coke & Quantity Demanded \\
                            \hline
                            \$1           & 2                 \\
                            \$2           & 1                 \\
                            \hline
                        \end{tabular}
                    \end{center}

                    This demand curve shows an inverse relationship between the price of coke and the quantity demanded, which is consistent with the law of demand.
          \end{enumerate}
\end{enumerate}

\section*{Part III: Long Essay}

\begin{enumerate}
    \item
          \begin{enumerate}
              \item \textbf{Answer:}

                    The unusually high temperatures will increase the demand for surfboards, leading to a higher equilibrium price (P2) and a higher equilibrium quantity (Q2).

                    \textbf{Explanation:}

                    The high temperatures will attract more people to the beaches, increasing the demand for surfboards. This will shift the demand curve to the right, resulting in a higher equilibrium price and quantity.

              \item \textbf{Answer:}

                    The increase in the price of epoxy paint will decrease the supply of surfboards, leading to a higher equilibrium price (P2) and a lower equilibrium quantity (Q2).

                    \textbf{Explanation:}

                    Epoxy paint is an input in the production of surfboards. An increase in the price of this input will increase the cost of production for surfboard manufacturers, leading to a decrease in supply. This will shift the supply curve to the left, resulting in a higher equilibrium price and a lower equilibrium quantity.

              \item \textbf{Answer:}

                    The admission fees at beaches will decrease the demand for surfboards, leading to a lower equilibrium price (P2) and a lower equilibrium quantity (Q2).

                    \textbf{Explanation:}

                    The admission fees will make it more expensive for people to go to the beaches, reducing the demand for surfboards. This will shift the demand curve to the left, resulting in a lower equilibrium price and quantity.

              \item \textbf{Answer:}

                    The announcement of bad weather will decrease the demand for surfboards, leading to a lower equilibrium price (P2) and a lower equilibrium quantity (Q2).

                    \textbf{Explanation:}

                    The expectation of bad weather will discourage people from going to the beaches, reducing the demand for surfboards. This will shift the demand curve to the left, resulting in a lower equilibrium price and quantity.

              \item \textbf{Answer:}

                    No, the private market for surfboards would not produce the socially optimal number of surfboards. The positive externalities associated with surfing would result in underproduction.

                    \textbf{Explanation:}

                    The private market only considers the private benefits and costs of surfing. If there are positive externalities, such as increased happiness and well-being for others, the social benefit of surfing is greater than the private benefit. This means that the private market will underproduce surfboards compared to the socially optimal level. To achieve the socially optimal level, government intervention, such as subsidies, might be necessary.
          \end{enumerate}

    \item \textbf{Extra Credit: Answer:}

          The short-run demand curve for gasoline is less elastic (steeper) than the long-run demand curve. This means that in the short run, consumers are less responsive to price changes, while in the long run, they have more time to adjust their behavior and find substitutes.

          \textbf{Explanation:}

          The article argues that the fall in gasoline prices can be attributed to changes in consumer behavior over time. In the short run, when gas prices increased, consumers had limited options to adjust their behavior, such as finding alternative modes of transportation or reducing their driving. However, in the long run, consumers had more time to find substitutes, such as carpooling, using public transport, or buying more fuel-efficient vehicles. This increased responsiveness to price changes in the long run is reflected in the more elastic long-run demand curve.

          The combination of supply shortages due to the Ukrainian War and the less elastic short-run demand curve led to a sharp increase in gas prices. However, as time passed and consumers adjusted their behavior, the demand for gasoline decreased, leading to a fall in prices despite the continuing supply shortages. This highlights the importance of considering the time horizon when analyzing demand and price changes.
\end{enumerate}

\end{document}
