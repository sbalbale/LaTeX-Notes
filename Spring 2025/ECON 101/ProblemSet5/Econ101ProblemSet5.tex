\documentclass{article}
\usepackage{amsmath} % For math environments if needed
\usepackage{geometry} % For adjusting margins
\geometry{a4paper, margin=1in}
\usepackage{enumitem} % For better lists

\title{Economics 101: Problem Set \#5 Solutions}
\author{Sean Balbale} % Add your name(s)
\date{\today} % Or specify the due date: April 4th

\begin{document}
\maketitle

\section*{Part I: Multiple Choice}
\textit{Instructions: For each of the multiple choice questions, please include a short explanation with your answer and show your work.}

\begin{enumerate}[label=\arabic*.]
  \item \textbf{Question:} Alejandro Scobertini owns a store specializing in soccer jerseys. In 2008, he purchased \$150,000 worth of jerseys from clothing manufacturers, employed one worker for \$40,000, purchased \$20,000 worth of supplies from an office supply store, and sold jerseys for \$280,000. Based on this information, what was the value added at Alejandro's store in 2008?
    \begin{itemize}
      \item \textbf{Answer:} D. \$110,000
      \item \textbf{Work:} Value added is the market value of a firm's output minus the value of the inputs the firm has bought from others.
        \begin{itemize}
          \item Value of Output = Sales = \$280,000
          \item Value of Intermediate Inputs = Purchases from manufacturers + Purchases of supplies = \$150,000 + \$20,000 = \$170,000
          \item Value Added = Value of Output - Value of Intermediate Inputs
          \item Value Added = \$280,000 - \$170,000 = \$110,000
        \end{itemize}
      \item \textbf{Explanation:} The value added is the difference between the sales revenue and the cost of intermediate goods purchased from other firms. Employee wages (\$40,000) are part of the value added, not an intermediate input cost deducted from sales to calculate it.
    \end{itemize}

  \item \textbf{Question:} Official unemployment rate statistics may:
    \begin{itemize}
      \item \textbf{Answer:} D. Understate the amount of unemployment because of the presence of "discouraged" workers who are not actively seeking employment.
      \item \textbf{Explanation:} The official unemployment rate only counts individuals who are actively looking for work. Discouraged workers, those who want a job but have given up searching, are not counted as unemployed or part of the labor force. This leads to an understatement of the true extent of unemployment.
    \end{itemize}

  \item \textbf{Question:} The best example of a "frictionally unemployed" worker is one who:
    \begin{itemize}
      \item \textbf{Answer:} C. Is in the process of voluntarily switching jobs.
      \item \textbf{Explanation:} Frictional unemployment refers to the temporary unemployment experienced by individuals who are between jobs, including those who have voluntarily quit to search for a better one or those who are new entrants/re-entrants to the labor force. It's a normal part of a dynamic economy.
    \end{itemize}

  \item \textbf{Question:} The consumer price index (CPI) for a market basket of goods in year 1 in the country of Wonderland was 105. In year 2, the price index for the same market basket of goods increased to 111. What was Wonderland's rate of inflation during year 2?
    \begin{itemize}
      \item \textbf{Answer:} D. 5.7 percent.
      \item \textbf{Work:} Inflation Rate = [(CPI in Year 2 - CPI in Year 1) / CPI in Year 1] * 100
        \begin{align*}
          \text{Inflation Rate} &= \frac{111 - 105}{105} \times 100 \\
          &= \frac{6}{105} \times 100 \\
          &\approx 0.05714 \times 100 \\
          &\approx 5.714\%
        \end{align*}
      \item \textbf{Explanation:} The inflation rate measures the percentage change in the price index from one period to the next. Calculating this percentage change gives approximately 5.7\%.
    \end{itemize}

  \item \textbf{Question:} Some economists prefer to use the term business fluctuations rather than business cycles to describe the historical growth record in the United States because:
    \begin{itemize}
      \item \textbf{Answer:} B. Cycles imply regularity while fluctuations do not.
      \item \textbf{Explanation:} The term "cycle" suggests a regular, predictable pattern of ups and downs. However, economic expansions and contractions (fluctuations) vary significantly in duration and intensity, making "fluctuations" a potentially more accurate, less deterministic term.
    \end{itemize}

  \item \textbf{Question:} Most economists agree that the immediate cause of the majority of cyclical changes in the level of real output is unexpected changes in the:
    \begin{itemize}
      \item \textbf{Answer:} A. Level of total spending.
      \item \textbf{Explanation:} Keynesian and many mainstream economists argue that unexpected changes in aggregate demand (total spending) are the primary drivers of short-run fluctuations in output and employment (business cycles).
    \end{itemize}

  \item \textbf{Question:} The unemployment rate in an economy is 10\%. The total population of the economy is 350 million, and the size of the civilian labor force is 200 million. The number of employed workers in this economy is:
    \begin{itemize}
      \item \textbf{Answer:} D. 180 million.
      \item \textbf{Work:}
        \begin{itemize}
          \item Unemployment Rate = (Number of Unemployed / Labor Force) * 100
          \item 10\% = (Number of Unemployed / 200 million) * 100
          \item 0.10 = Number of Unemployed / 200 million
          \item Number of Unemployed = 0.10 * 200 million = 20 million
          \item Labor Force = Number of Employed + Number of Unemployed
          \item 200 million = Number of Employed + 20 million
          \item Number of Employed = 200 million - 20 million = 180 million
        \end{itemize}
      \item \textbf{Explanation:} The labor force consists of both employed and unemployed individuals. Given the labor force size and the unemployment rate, we can calculate the number of unemployed people and subtract that from the labor force to find the number of employed people.
    \end{itemize}

  \item \textbf{Question:} Consider the following three economies, A, B, and C: Economy A: gross private domestic investment expenditures equal depreciation. Economy B: depreciation exceeds gross private domestic investment expenditures. Economy C: gross private domestic investment expenditures exceed depreciation. Referring to the above information. Positive net private domestic investment expenditures occurs in:
    \begin{itemize}
      \item \textbf{Answer:} C. economy C only.
      \item \textbf{Explanation:} Net Private Domestic Investment = Gross Private Domestic Investment - Depreciation. Positive net investment occurs only when gross investment exceeds depreciation (Economy C).
    \end{itemize}
\end{enumerate}

\section*{Part II: Short Essay}

\begin{enumerate}[label=\arabic*.]
  \item \textbf{Problem (from Chapter 29, p. 577, Problem 2):} Assume the following data for a country: total population, 500; population under 16 years of age or institutionalized, 120; not in the labor force, 150; unemployed, 23; part-time workers looking for full-time jobs, 10. What is the size of the labor force? What is the official unemployment rate?
    \begin{itemize}
      \item \textbf{Answer:} Size of Labor Force = 230; Official Unemployment Rate = 10\%
      \item \textbf{Work and Explanation:}
        \begin{itemize}
          \item Potential Labor Force = Total Population - Population under 16 or institutionalized = 500 - 120 = 380
          \item Labor Force = Potential Labor Force - Not in the labor force = 380 - 150 = 230
          \item Unemployment Rate = (Number of Unemployed / Labor Force) * 100 = (23 / 230) * 100 = 10\%
        \end{itemize}
        Part-time workers seeking full-time work are counted as employed and are part of the labor force figure calculated above.
    \end{itemize}

\item \textbf{Problem:} Consider the following three scenarios. For each scenario, please i) identify the type of unemployment and ii) provide a one-two sentence explanation for your choice.
\begin{enumerate}[label=\Alph*.]
  \item Tamara works in an automotive assembly plant. She was laid off six months ago as the economy weakened. She expects to return to work in several months when national economic conditions improve.
    \begin{itemize}
    \item i) \textbf{Type:} Cyclical Unemployment
  \item ii) \textbf{Explanation:} Tamara's job loss is directly tied to a downturn in the overall economy ("economy weakened") and she expects to be rehired upon recovery, characteristic of cyclical unemployment.
\end{itemize}
\item A worker loses her job at a petroleum refinery because consumers and business firms switch from the use of oil to the burning of coal.
\begin{itemize}
\item i) \textbf{Type:} Structural Unemployment
\item ii) \textbf{Explanation:} This unemployment results from a long-term shift in demand/technology (oil to coal), creating a mismatch between the worker's skills and available jobs.
\end{itemize}
\end{enumerate}

\item \textbf{Problem (from Chapter 27, p. 535, Problem 4a):} Using the provided data (all figures in billions), determine GDP by both the expenditures approach and the income approach. Then determine NDP. (Data provided in problem set image).
\begin{itemize}
\item \textbf{Answers:} GDP (Expenditures Approach) = \$388 billion; GDP (Income Approach) = \$388 billion; NDP = \$361 billion
\item \textbf{Work and Explanation:}
\begin{itemize}
\item \textbf{Expenditures Approach (GDP = C + Ig + G + NX):}
\begin{itemize}
  \item Gross Investment (Ig) = Net Investment + Depreciation = \$33 + \$27 = \$60 billion
  \item GDP = \$245 (C) + \$60 (Ig) + \$72 (G) + \$11 (NX) = \$388 billion
\end{itemize}
\item \textbf{Income Approach (GDP = NI - Net Foreign Factor Income + Statistical Discrepancy + Depreciation):}
\begin{itemize}
  \item National Income (NI) = Compensation of employees (\$223) + Rents (\$14) + Interest (\$13) + Proprietors' income (\$33) + Corporate profits (\$56) + Taxes on production and imports (\$18) = \$357 billion
  \item GDP = \$357 (NI) - \$4 (Net Foreign Factor Income) + \$8 (Statistical Discrepancy) + \$27 (Depreciation) = \$388 billion
\end{itemize}
\item \textbf{Net Domestic Product (NDP):}
\begin{itemize}
  \item NDP = GDP - Depreciation = \$388 - \$27 = \$361 billion
\end{itemize}
\end{itemize}
\end{itemize}

\item \textbf{Problem:} A friend asks: Why might the official unemployment rate actually increase as an economic recovery begins (as we move from the "trough" to the early expansionary phase of the business cycle)? Provide a simple, clear answer (max 150 words).
\begin{itemize}
\item \textbf{Answer:} It seems counterintuitive, but the official unemployment rate can sometimes rise just as the economy starts getting better. Think about people who were so discouraged by the bad economy that they stopped looking for jobs altogether. Because they weren't actively searching, they weren't counted as "unemployed". When the economy shows signs of life and new jobs start appearing, these discouraged workers become hopeful again and start actively looking for work. As soon as they start searching, they re-enter the labor force and are officially counted as unemployed until they find a job. If enough people start looking again before hiring really picks up speed, this influx can temporarily push the official unemployment rate higher, even though the economy is improving.
\end{itemize}
\end{enumerate}

\end{document}
