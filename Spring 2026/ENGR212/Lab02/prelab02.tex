\documentclass[11pt, letterpaper]{article}
\usepackage[utf8]{inputenc}
\usepackage[margin=1in]{geometry}
\usepackage{amsmath}
\usepackage{amssymb}
\usepackage{booktabs} % For professional tables
\usepackage{siunitx}  % For consistent unit formatting
\usepackage{graphicx}
\usepackage{float}

% Document Info
\title{\textbf{ENGR 212: Laboratory Experiment 2}\\Answer Guide \& TA Notes}
\author{Teaching Assistant Reference}
\date{}

\begin{document}

\maketitle

\section*{Overview}
This guide outlines the theoretical values and expected behaviors for \textbf{Lab Experiment 2}. Use this to verify student results and explain discrepancies during the lab session.

\section{Part 1: Voltage Divider Circuit Calculations}

\subsection*{Problem Statement}
Students must calculate $v_{out}$, $i$, and $P_{R_2}$ for the circuit in Figure 1(a) with a fixed source of $10\text{V}$ and $R_1 = 10\text{k}\Omega$. The value of $R_2$ varies across five specific resistance values.

\subsection*{Formulas Used}
The circuit is a standard unloaded voltage divider.

\begin{enumerate}
  \item \textbf{Current ($i$):}
    \begin{equation*}
      i = \frac{V_{source}}{R_{eq}} = \frac{10\text{V}}{R_1 + R_2}
    \end{equation*}

  \item \textbf{Output Voltage ($v_{out}$):}
    \begin{equation*}
      v_{out} = i \times R_2 = 10\text{V} \times \left( \frac{R_2}{R_1 + R_2} \right)
    \end{equation*}

  \item \textbf{Power Dissipated by $R_2$ ($P_{R_2}$):}
    \begin{equation*}
      P_{R_2} = i^2 \times R_2 = \frac{(v_{out})^2}{R_2}
    \end{equation*}
\end{enumerate}

\subsection*{Answer Key (Theoretical Values)}
Use the table below to check the ``Calculation'' columns in student reports. Note that power is calculated in milliwatts (mW).

\begin{table}[H]
  \centering
  \renewcommand{\arraystretch}{1.3}
  \begin{tabular}{@{}c c c c c@{}}
    \toprule
    \textbf{$\mathbf{R_2}$ Value} & \textbf{Total R ($\mathbf{R_1 + R_2}$)} & \textbf{Current $\mathbf{i}$} & \textbf{Voltage $\mathbf{v_{out}}$} & \textbf{Power $\mathbf{P_{R_2}}$} \\
    \midrule
    $2.2\,\text{k}\Omega$ & $12.2\,\text{k}\Omega$ & \textbf{0.820 mA} & \textbf{1.80 V} & \textbf{1.48 mW} \\
    $4.7\,\text{k}\Omega$ & $14.7\,\text{k}\Omega$ & \textbf{0.680 mA} & \textbf{3.20 V} & \textbf{2.18 mW} \\
    $10\,\text{k}\Omega$  & $20.0\,\text{k}\Omega$ & \textbf{0.500 mA} & \textbf{5.00 V} & \textbf{2.50 mW} \\
    $20\,\text{k}\Omega$  & $30.0\,\text{k}\Omega$ & \textbf{0.333 mA} & \textbf{6.67 V} & \textbf{2.22 mW} \\
    $33\,\text{k}\Omega$  & $43.0\,\text{k}\Omega$ & \textbf{0.233 mA} & \textbf{7.67 V} & \textbf{1.78 mW} \\
    \bottomrule
  \end{tabular}
  \caption{Theoretical values for Table 1. Measured values should be within resistor tolerance ($\approx 5\%$).}
\end{table}

\section{Part 2: LED Circuit Analysis}

\subsection*{Theoretical Expectations}

\subsubsection*{1. Forward Bias Behavior (Steps a--c)}
\begin{itemize}
  \item \textbf{Threshold:} An LED is a non-linear device. Current will remain near zero until the voltage across the LED reaches its \textbf{turn-on voltage} ($V_{turn-on}$).
  \item \textbf{Typical $V_{turn-on}$:} For a standard red LED, this is typically \textbf{1.8V -- 2.0V}.
  \item \textbf{Behavior:}
    \begin{itemize}
      \item \textbf{Input 1V:} Source $<$ Turn-on. LED is OFF. $I \approx 0\text{mA}$. $V_{LED} \approx 1\text{V}$ (Open circuit behavior).
      \item \textbf{Input 2V $\sim$ 5V:} Source $>$ Turn-on. LED turns ON. $V_{LED}$ clamps near the forward voltage ($\approx 2\text{V}$) and rises very slowly. Current increases linearly with the remaining voltage drop across the resistors.
    \end{itemize}
\end{itemize}

\subsubsection*{2. Effect of Varying $R_1$ (Step d)}
Increasing $R_1$ increases the total series resistance ($R_{total} = R_1 + R_2$).
\begin{itemize}
  \item \textbf{Impact:} For the same input voltage, a higher $R_1$ results in \textbf{lower current} flowing through the LED, making it dimmer.
  \item The voltage across the LED ($V_{LED}$) will remain relatively stable (characteristic of a diode), but the voltage drop across the resistors will change significantly.
\end{itemize}

\subsubsection*{3. Estimating Turn-on Voltage (Step f)}
When students plot $I$ ($y$-axis) vs $V_{LED}$ ($x$-axis), they should observe a ``knee'' in the graph.
\begin{itemize}
  \item The curve should be flat (at $I=0$) until roughly \textbf{1.8V} (for Red LED), after which current shoots up steeply.
\end{itemize}

\subsection*{Sample Calculation}
\textit{Assuming a Red LED ($V_{turn-on} \approx 1.9\text{V}$), Input = 5V, and $R_1 = 0\Omega$:}

\begin{align*}
  V_{R} &= 5\text{V} - 1.9\text{V} = 3.1\text{V} \\
  R_{tot} &= 3.9\text{k}\Omega \\
  I &= \frac{3.1\text{V}}{3.9\text{k}\Omega} \approx 0.79\text{mA}
\end{align*}

\section*{Supplementary Note: Loading Effect}
Although ``loading effect'' is listed in the objectives, it is not explicitly calculated in the numbered steps.
\begin{itemize}
  \item \textbf{Concept:} The loading effect occurs when a load resistor $R_L$ is attached in parallel to $R_2$ (Figure 1b). This reduces the effective resistance of the bottom branch to $R_{eq} = R_2 \parallel R_L$.
  \item \textbf{Result:} The measured $v_{out}$ will be \textbf{lower} than the calculated unloaded $v_{out}$ from Table 1.
\end{itemize}

\end{document}
