\documentclass[11pt, letterpaper]{article}
\usepackage[utf8]{inputenc}
\usepackage[margin=1in]{geometry}
\usepackage{amsmath}
\usepackage{amssymb}
\usepackage{booktabs} % For professional tables
\usepackage{siunitx}  % For consistent unit formatting
\usepackage{graphicx}
\usepackage{float}
\usepackage{multirow} % For grouping rows in the big table

% Document Info
\title{\textbf{ENGR 212: Laboratory Experiment 2}\\Answer Guide \& TA Notes}
\author{Teaching Assistant Reference}
\date{}

\begin{document}

\maketitle

\section*{Overview}
This guide outlines the theoretical values and expected behaviors for \textbf{Lab Experiment 2}. Use this to verify student results and explain discrepancies during the lab session.

\section{Part 1: Voltage Divider Circuit Calculations}

\subsection*{Problem Statement}
Students must calculate $v_{out}$, $i$, and $P_{R_2}$ for the circuit in Figure 1(a) with a fixed source of $10\text{V}$ and $R_1 = 10\text{k}\Omega$. The value of $R_2$ varies across five specific resistance values.

\subsection*{Formulas Used}
The circuit is a standard unloaded voltage divider.

\begin{enumerate}
    \item \textbf{Current ($i$):}
    \begin{equation*}
        i = \frac{V_{source}}{R_{eq}} = \frac{10\text{V}}{R_1 + R_2}
    \end{equation*}
    
    \item \textbf{Output Voltage ($v_{out}$):}
    \begin{equation*}
        v_{out} = i \times R_2 = 10\text{V} \times \left( \frac{R_2}{R_1 + R_2} \right)
    \end{equation*}
    
    \item \textbf{Power Dissipated by $R_2$ ($P_{R_2}$):}
    \begin{equation*}
        P_{R_2} = i^2 \times R_2 = \frac{(v_{out})^2}{R_2}
    \end{equation*}
\end{enumerate}

\subsection*{Answer Key (Theoretical Values)}
Use the table below to check the ``Calculation'' columns in student reports. Note that power is calculated in milliwatts (mW).

\begin{table}[H]
    \centering
    \renewcommand{\arraystretch}{1.3}
    \begin{tabular}{@{}c c c c c@{}}
        \toprule
        \textbf{$\mathbf{R_2}$ Value} & \textbf{Total R ($\mathbf{R_1 + R_2}$)} & \textbf{Current $\mathbf{i}$} & \textbf{Voltage $\mathbf{v_{out}}$} & \textbf{Power $\mathbf{P_{R_2}}$} \\
        \midrule
        $2.2\,\text{k}\Omega$ & $12.2\,\text{k}\Omega$ & \textbf{0.820 mA} & \textbf{1.80 V} & \textbf{1.48 mW} \\
        $4.7\,\text{k}\Omega$ & $14.7\,\text{k}\Omega$ & \textbf{0.680 mA} & \textbf{3.20 V} & \textbf{2.18 mW} \\
        $10\,\text{k}\Omega$  & $20.0\,\text{k}\Omega$ & \textbf{0.500 mA} & \textbf{5.00 V} & \textbf{2.50 mW} \\
        $20\,\text{k}\Omega$  & $30.0\,\text{k}\Omega$ & \textbf{0.333 mA} & \textbf{6.67 V} & \textbf{2.22 mW} \\
        $33\,\text{k}\Omega$  & $43.0\,\text{k}\Omega$ & \textbf{0.233 mA} & \textbf{7.67 V} & \textbf{1.78 mW} \\
        \bottomrule
    \end{tabular}
    \caption{Theoretical values for Table 1. Measured values should be within resistor tolerance ($\approx 5\%$).}
\end{table}

\section{Part 2: LED Circuit Analysis}

\subsection*{Theoretical Expectations}

\subsubsection*{1. Forward Bias Behavior}
\begin{itemize}
    \item \textbf{Threshold:} An LED is a non-linear device. Current will remain near zero until the voltage across the LED reaches its \textbf{turn-on voltage} ($V_{turn-on}$).
    \item \textbf{Typical $V_{turn-on}$:} For a standard red LED, this is typically \textbf{1.8V -- 2.0V}.
    \item \textbf{Behavior:}
    \begin{itemize}
        \item \textbf{Input 1V:} Source $<$ Turn-on. LED is OFF. $I \approx 0\text{mA}$. $V_{LED} \approx 1\text{V}$ (Open circuit behavior).
        \item \textbf{Input 2V $\sim$ 5V:} Source $>$ Turn-on. LED turns ON. $V_{LED}$ clamps near the forward voltage ($\approx 2\text{V}$) and rises very slowly.
    \end{itemize}
\end{itemize}

\subsection*{Master Reference Table: Part 2 Measurements}
The table below provides theoretical data for all permutations of $R_1$ and Input Voltage ($V_{in}$).
\textbf{Note:} Calculations assume a generic Red LED with $V_f \approx 1.8\text{V} - 2.0\text{V}$.

\begin{table}[H]
    \centering
    \small % Slightly smaller font to fit the large table
    \renewcommand{\arraystretch}{1.2}
    \begin{tabular}{@{}c c c c c c@{}}
        \toprule
        \textbf{$\mathbf{R_1}$ Value} & \textbf{Input ($V_{in}$)} & \textbf{LED State} & \textbf{$V_{LED}$ (approx)} & \textbf{$V_{R1}$ (calc)} & \textbf{Current $i$ (mA)} \\
        \midrule
        \multirow{5}{*}{$\mathbf{0\,\Omega}$} 
          & 1.0 V & OFF & 1.00 V & 0.00 V & 0.00 \\
          & 2.0 V & Dim ON & 1.80 V & 0.00 V & 0.05 \\
          & 3.0 V & ON & 1.88 V & 0.00 V & 0.29 \\
          & 4.0 V & Bright & 1.95 V & 0.00 V & 0.53 \\
          & 5.0 V & Brightest & 2.00 V & 0.00 V & 0.77 \\
        \midrule
        \multirow{5}{*}{$\mathbf{100\,\Omega}$} 
          & 1.0 V & OFF & 1.00 V & 0.00 V & 0.00 \\
          & 2.0 V & Dim ON & 1.80 V & 0.01 V & 0.05 \\
          & 3.0 V & ON & 1.88 V & 0.03 V & 0.28 \\
          & 4.0 V & Bright & 1.95 V & 0.05 V & 0.51 \\
          & 5.0 V & Brightest & 2.00 V & 0.08 V & 0.75 \\
        \midrule
        \multirow{5}{*}{$\mathbf{1\,\text{k}\Omega}$} 
          & 1.0 V & OFF & 1.00 V & 0.00 V & 0.00 \\
          & 2.0 V & Faint & 1.78 V & 0.05 V & 0.04 \\
          & 3.0 V & ON & 1.85 V & 0.23 V & 0.23 \\
          & 4.0 V & ON & 1.90 V & 0.43 V & 0.43 \\
          & 5.0 V & Bright & 1.95 V & 0.62 V & 0.62 \\
        \midrule
        \multirow{5}{*}{$\mathbf{100\,\text{k}\Omega}$} 
          & 1.0 V & OFF & 1.00 V & 0.00 V & 0.00 \\
          & 2.0 V & OFF/Dim & 1.60 V & 0.38 V & 0.004 \\
          & 3.0 V & Very Dim & 1.65 V & 1.30 V & 0.013 \\
          & 4.0 V & Dim & 1.70 V & 2.21 V & 0.022 \\
          & 5.0 V & Dim & 1.75 V & 3.12 V & 0.031 \\
        \bottomrule
    \end{tabular}
    \caption{Theoretical Measurements for LED Circuit (Figure 3).}
\end{table}

\subsection*{TA Grading Notes}
\begin{itemize}
    \item \textbf{$\mathbf{100\,\text{k}\Omega}$ Behavior:} At this high resistance, current drops to $\approx 30\,\mu\text{A}$. The LED may appear OFF or extremely faint. Most of the voltage drop occurs across $R_1$, not the LED.
    \item \textbf{Low Voltage ($<1.8\text{V}$):} If $V_{in}$ is below the turn-on threshold, the circuit is open. $V_{LED} = V_{in}$ and $I = 0$.
    \item \textbf{Measurement Tolerance:} Student DMMs may not be sensitive enough to capture $\mu\text{A}$ changes. Zero current readings at 100k$\Omega$ are acceptable for grading purposes.
\end{itemize}

\end{document}