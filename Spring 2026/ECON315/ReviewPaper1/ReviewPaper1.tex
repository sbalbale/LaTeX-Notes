\documentclass[12pt, letterpaper]{article}
\usepackage[utf8]{inputenc}
\usepackage[margin=1in]{geometry}
\usepackage{setspace}
\usepackage{titlesec}
\usepackage{indentfirst}

% Formatting Section Titles to be bold but not huge
\titleformat{\section}{\normalsize\bfseries}{}{0em}{}

% Double Spacing (Required by General Objectives)
\doublespacing

\begin{document}

% Header Information
\noindent
Sean Balbale \\
Theories of International Trade \\
Professor Ramirez \\
February 6, 2026

\begin{center}
    \textbf{A Review of Rodrik’s ``What's So Special About China's Exports?''}
\end{center}

\section{I. Introduction}

Standard economic theory tells a simple story: you export what you have. If a nation is labor-abundant, the Heckscher-Ohlin model predicts it will export labor-intensive goods like textiles. If it is capital-rich, it exports machinery. In ``What’s So Special About China’s Exports?,'' Dani Rodrik argues that China has fundamentally broken this rule. His central thesis is that China is an economic outlier—a country that managed to construct a high-tech export profile long before its natural endowments should have allowed it. While skeptics like Gene Grossman argue that governments lack the necessary information to ``pick winners,'' Rodrik provides compelling evidence that China successfully defied its comparative advantage through aggressive industrial policy. This review argues that while Rodrik’s ``EXPY'' metric quantifies the success of this strategy, the paper underestimates the difficulty of sustaining this growth once the ``catch-up'' phase ends and the real challenge of innovation begins.

\section{II. Summary of Arguments}

Rodrik builds his case around a central paradox: China exports the goods of a wealthy nation while its citizens remain relatively poor. To prove this, he introduces a quantitative metric called EXPY, which measures the implied productivity of a country’s export basket. By analyzing the income levels of other countries that export similar goods, Rodrik assigns a ``sophistication score'' to what China sells.

The data is striking. Rodrik shows that China’s export basket is significantly more sophisticated than its income level predicts—resembling that of a country three times richer. He argues this wasn't an accident of the free market, but the result of a deliberate government strategy. Through policies like domestic content requirements and ``forced'' technology transfers from foreign multinationals, Beijing pushed its industries into complex electronics and machinery sectors that would not have emerged naturally. Rodrik concludes that \textit{what} you export matters; by latching onto these high-productivity goods early, China bootstrapped its way to rapid convergence.

\section{III. Analysis: Validating the ``Infant Industry''}

Rodrik’s argument effectively validates the ``Infant Industry'' theory we discussed in class, specifically Ha-Joon Chang’s ``Six-Year-Old Son'' analogy. Chang argues that exposing a developing industry to full global competition is like forcing a child to get a job; they need protection to build capabilities first. Rodrik provides the hard data to back this up. By shielding its tech sector from the full brunt of global competition, China allowed its firms to mature.

This directly challenges the skepticism found in Gene Grossman’s \textit{Strategic Export Promotion}. Grossman warns that governments usually fail when they try to target specific industries because they lack the information to know which ones will succeed. Rodrik counters this by suggesting the goal isn't to be clairvoyant, but to encourage ``cost discovery.'' The Chinese government didn't need to predict the iPad; they just needed to create an ecosystem where high-tech manufacturing was profitable enough to survive the initial risks. In this sense, Rodrik makes a strong case that comparative advantage is not a static destiny, but a dynamic variable that policy can alter.

However, the argument has limitations. Rodrik focuses heavily on the \textit{outcome} (the high EXPY score) but glosses over the \textit{cost}. Paul Krugman warns in \textit{Is Free Trade Passé?} that even if strategic trade is theoretically sound, the political economy of it is messy. ``Political capture'' often leads to resources being wasted on politically connected firms rather than efficient ones. Rodrik assumes a level of government competence that is unique to China and difficult for other developing nations to replicate.

\section{IV. Evaluation}

The paper’s greatest strength is its methodological innovation. The EXPY index is a useful tool for visualizing development quality beyond just GDP. It proves that not all exports are created equal—exporting computer chips has different spillover effects than exporting rice.

However, Rodrik’s optimism might be overstated regarding the future. His model explains how China got from poverty to middle income, but it doesn't necessarily explain how it will get to high income. The strategy of copying and ``latching on'' to existing technologies has diminishing returns. As China closes the technological gap, it can no longer just manufacture what others invented; it has to innovate. Rodrik’s analysis doesn't fully address whether a state-led system that is good at \textit{copying} is also good at \textit{creating}. Furthermore, there is a risk that the high EXPY score is inflated by assembly trade—if China is just assembling high-tech components imported from Japan and Korea, the ``sophistication'' is imported, not indigenous.

\section{V. Conclusion}

Ultimately, Rodrik provides a necessary corrective to the idea that free markets alone drive development. He successfully demonstrates that China’s rise was an engineered anomaly, not a natural occurrence. The paper is valuable because it forces us to rethink the role of the state: not just as a regulator, but as a venture capitalist that absorbs the initial risks of development. While the strategy carries risks of inefficiency, Rodrik’s evidence suggests that for a country of China’s scale, the government’s ``heavy hand'' was the essential catalyst for escaping the ``low-income trap.''

\newpage
\begin{center}
    \textbf{References}
\end{center}
\singlespacing

\noindent Chang, H-J. (2008). \textit{Bad Samaritans: The Myth of Free Trade and the Secret History of Capitalism}. Bloomsbury Press.

\vspace{1em}

\noindent Grossman, G. M. (1986). ``Strategic Export Promotion: A Critique.'' In \textit{Strategic Trade Policy and the New International Economics}. MIT Press.

\vspace{1em}

\noindent Krugman, P. R. (1987). ``Is Free Trade Passé?'' \textit{The Journal of Economic Perspectives}, 1(2), 131–144.

\vspace{1em}

\noindent Rodrik, D. (2006). ``What's So Special About China's Exports?'' \textit{NBER Working Paper Series}.

\end{document}