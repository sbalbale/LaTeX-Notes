\documentclass[10pt]{article}
\usepackage{amsmath}
\usepackage{amssymb}
\usepackage{fancyhdr}
\usepackage{nicematrix}
\usepackage{tikz}  % Use tikz package for node-based graphs
\usetikzlibrary{positioning}  % Add positioning library for 'of' syntax

\title{CS1800 Homework 7 Solutions}
\author{}
\date{}

% Header for every page
\pagestyle{fancy}
\fancyhf{}
\fancyhead[L]{Name: Sean Balbale}
\fancyhead[R]{HW Group: None}

\setlength\parindent{0pt}

% Use Roman numerals for subsections
\renewcommand{\thesubsection}{\Roman{subsection}}
\renewcommand{\thesubsubsection}{\Roman{subsection}.\Roman{subsubsection}}

\begin{document}

\maketitle
\newpage

\section{Upper Triangular Matrix}

\textbf{Problem Statement:} \\
Use induction to show that an \( n \times n \) matrix has \( \frac{n(n+1)}{2} \) upper-diagonal entries.

\subsection{Solution}

\subsubsection{Base Case}

For \( n = 1 \):
\[
\frac{1 \times (1 + 1)}{2} = \frac{2}{2} = 1
\]
which is correct because a \( 1 \times 1 \) matrix has only one entry, which is on the diagonal.

\subsubsection{Inductive Hypothesis}

Assume for some integer \( k \) that a \( k \times k \) matrix has \( \frac{k(k+1)}{2} \) upper-diagonal entries.

\subsubsection{Inductive Step}

Consider a \( (k+1) \times (k+1) \) matrix. This matrix contains all the entries of the \( k \times k \) matrix plus an additional row and column, which contribute \( k + 1 \) new upper-diagonal entries.

So, the total number of upper-diagonal entries in a \( (k+1) \times (k+1) \) matrix is:
\[
\frac{k(k+1)}{2} + (k + 1) = \frac{k(k+1) + 2(k+1)}{2} = \frac{(k+1)(k+2)}{2}
\]
Replacing \( k+1 = n \) in the formula, we get:
\[
\frac{n(n+1)}{2}
\]
Thus, by induction, the formula holds for all \( n \geq 1 \).

\newpage

\section{A Partial Series \(\frac{1}{k(k+1)}\)}

\textbf{Problem Statement:} \\
Show that:
\[
\sum_{k=1}^n \frac{1}{k(k+1)} = 1 - \frac{1}{n+1}
\]

\subsection{Solution}

\subsubsection{Base Case}

For \( n = 1 \):
\[
\sum_{k=1}^1 \frac{1}{k(k+1)} = \frac{1}{1 \cdot 2} = \frac{1}{2}
\]
and the formula gives \( 1 - \frac{1}{2} = \frac{1}{2} \), which is correct.

\subsubsection{Inductive Hypothesis}

Assume that for some integer \( k \),
\[
\sum_{j=1}^k \frac{1}{j(j+1)} = 1 - \frac{1}{k+1}
\]

\subsubsection{Inductive Step}

For \( n = k+1 \):
\[
\sum_{j=1}^{k+1} \frac{1}{j(j+1)} = \left(1 - \frac{1}{k+1}\right) + \frac{1}{(k+1)(k+2)}
\]
Simplifying, we get:
\[
= 1 - \frac{1}{k+1} + \frac{1}{(k+1)(k+2)} = 1 - \frac{(k+2) - 1}{(k+1)(k+2)} = 1 - \frac{1}{k+2}
\]
Replacing \( k+1 = n\) in the formula, we get:
\[
1 - \frac{1}{n+1}
\]
Thus, by induction, the formula holds for all \( n \geq 1 \).

\newpage

\section{A Partial Series \(k^3\)}

\textbf{Problem Statement:} \\
Prove that:
\[
\sum_{k=0}^n k^3 = \left(\frac{n(n+1)}{2}\right)^2
\]

\subsection{Solution}

\subsubsection{Base Case}

For \( n = 0 \):
\[
\sum_{k=0}^0 k^3 = 0
\]
The formula gives \( \left(\frac{0 \cdot (0+1)}{2}\right)^2 = 0 \), which is correct.

\subsubsection{Inductive Hypothesis}

Assume that for some \( k \),
\[
\sum_{j=0}^k j^3 = \left(\frac{k(k+1)}{2}\right)^2
\]

\subsubsection{Inductive Step}

For \( n = k+1 \):
\[
\sum_{j=0}^{k+1} j^3 = \sum_{j=0}^k j^3 + (k+1)^3
\]
Using the inductive hypothesis:
\[
= \left(\frac{k(k+1)}{2}\right)^2 + (k+1)^3
\]
Simplifying both sides confirms that:
\[
\sum_{j=0}^{k+1} j^3 = \left(\frac{(k+1)(k+2)}{2}\right)^2
\]
Replacing \( k+1 = n \) in the formula, we get:
\[
\left(\frac{n(n+1)}{2}\right)^2
\]
Hence, by induction, the formula holds for all \( n \geq 0 \).

\newpage

\section{Function Growth \((n < 2^n)\)}

\textbf{Problem Statement:} \\
Prove that \( n < 2^n \) for all \( n \in \mathbb{Z}^+ \).

\subsection{Solution}

\subsubsection{Base Case}

For \( n = 1 \):
\[
1 < 2^1
\]
which is true.

\subsubsection{Inductive Hypothesis}

Assume that \( k < 2^k \) for some \( k \geq 1 \).

\subsubsection{Inductive Step}

We need to show that \( k+1 < 2^{k+1} \). Since \( k < 2^k \) by the hypothesis, add \( 1 < 2^k \):
\[
k+1 < 2^k + 2^k = 2^{k+1}
\]
Thus, by induction, \( n < 2^n \) holds for all \( n \in \mathbb{Z}^+ \).

\end{document}
