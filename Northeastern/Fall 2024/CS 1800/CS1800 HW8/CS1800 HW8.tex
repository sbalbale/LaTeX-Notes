\documentclass[10pt]{article}
\usepackage{amsmath}
\usepackage{amssymb}
\usepackage{fancyhdr}
\usepackage{nicematrix}
\usepackage{tikz}
\usetikzlibrary{positioning}

\title{HW8: Function Growth, Sequences \& Series}
\author{}
\date{Due: Nov 26, 2024 @ 11:59 PM}

% Header
\pagestyle{fancy}
\fancyhf{}
\fancyhead[L]{Name: Sean Balbale}
\fancyhead[R]{HW Group: None}

\setlength\parindent{0pt}

% Roman numerals for subsections
\renewcommand{\thesubsection}{\Roman{subsection}}
\renewcommand{\thesubsubsection}{\Roman{subsection}.\Roman{subsubsection}}

\begin{document}

\maketitle

\newpage

\section*{Problem 1 [36 pts]: Sequences \& Series}

\subsection*{(i) Sequence: \( 18, 72, 288, 1152, 4608, 18432, \dots \)}

\textbf{Type:} Geometric sequence (constant ratio \( r = 4 \)).

\textbf{Expression for \( a_k \):}
\[
a_k = 18 \cdot 4^k
\]

\textbf{Sum of the first 11 terms:}
\[
S_{11} = 18 (\frac{1 - 4^{11}}{1 - 4}) = 25165818
\]

\subsection*{(ii) Sequence: \( -1, 1, 7, 17, 31, 49, \dots \)}

\textbf{Type:} Quadratic sequence (constant second difference \( \Delta^2 = 4 \)).

\textbf{Expression for \( a_k \):}
\[
a_k = 2k^2 - 4k + 1
\]
\textbf{Sum of the first 11 terms:}
\[
S_{11}=\sum_{k=1}^{11} (2k^2 - 4k + 1) = 759
\]

\subsection*{(iii) Sequence: \( 0, -1, -2, -3, -4, -5, \dots \)}

\textbf{Type:} Arithmetic sequence (constant difference \( d = -1 \)).

\textbf{Expression for \( a_k \):}
\[
a_k = -k
\]

\textbf{Sum of the first 11 terms:}
\[
S_{11} = \sum_{k=1}^{11} (-k) = -66
\]

\newpage

\section*{Problem 2 [18 pts (3 each)]: Function Growth True/False}

\begin{enumerate}
    \item \( 5n^3 + 2n = O(n^4) \): \textbf{False} \\
    The growth of \( 5n^3 + 2n \) is \( O(n^3) \), not \( O(n^4) \).

    \item \( n^5 = O(7n + 1) \): \textbf{False} \\
    \( n^5 \) grows faster than \( O(7n + 1) \).

    \item If \( 2n^3 + 6n = O(h(n)) \), then \( 2n^3 + 6n > h(n) \): \textbf{False} \\
    The statement misinterprets Big-O; \( O(h(n)) \) does not imply \( f(n) > h(n) \) for all \( n \).

    \item \( 4n + 1 = \Omega(n^2) \): \textbf{False} \\
    The growth of \( 4n + 1 \) is \( O(n) \), so it cannot be \( \Omega(n^2) \).

    \item \( 6n^2 + 4 = O(6n^2 + 4) \): \textbf{True} \\
    A function is always \( O \) of itself.

    \item \( 5 \log_2 n = \Theta(7 \log_{10} n + 1)^2 \): \textbf{False} \\
    The growth of \( \log n \) is much slower than \( (\log n)^2 \), so this cannot hold.
\end{enumerate}


\newpage

\section*{Problem 3 [22 pts (14, 8 pts)]: Function Growth}

\subsection*{(i) Function Columns}
Group the following functions based on asymptotic growth rates (\( O \)):

\begin{align*}
\text{Column 1:} & \quad 10000 \\
\text{Column 2:} & \quad \log_2 n, \; 3^{\log_3 n} \\
\text{Column 3:} & \quad n \\
\text{Column 4:} & \quad n \log_2 n, \; n \log_3 n \\
\text{Column 5:} & \quad n^2, \; 100n \\
\text{Column 6:} & \quad 2^n, \; 3^n \\
\text{Column 7:} & \quad n! \\
\end{align*}

\subsection*{(ii) Simplest \( f(n) \) for \( 3n + 4n^2 + 3n! = O(f(n)) \):}
\[
f(n) = n!
\]

\newpage

\section*{Problem 4 [24 pts (6 each)]: Demonstrating Function Growth}

For each statement, find \( c, x_0 \) such that \( 0 \leq f(x) \leq c g(x) \) for \( x \geq x_0 \), or provide a justification if the statement is false.

\begin{enumerate}
    \item \( 2^x = O(3^x) \): \textbf{True}. \\
    Exponential functions \( 2^x \) and \( 3^x \) satisfy \( 2^x \leq c \cdot 3^x \) for \( c = 1 \) and \( x_0 = 1 \).

    \item \( 5x^3 + x = O(x^3) \): \textbf{True}. \\
    The cubic term dominates, so \( 5x^3 + x \leq 6x^3 \) for \( c = 6 \) and \( x_0 = 1 \).

    \item \( x^4 = O(\ln x) \): \textbf{False}. \\
    Polynomial growth \( x^4 \) is faster than logarithmic growth \( \ln x \), so no constants \( c, x_0 \) can satisfy the inequality.

    \item \( 4x + 7 = O(x^2) \): \textbf{True}. \\
    For \( x \geq 8 \), \( 4x + 7 \leq x^2 \) can be satisfied with \( c = 1 \) and \( x_0 = 8 \).
\end{enumerate}


\end{document}
