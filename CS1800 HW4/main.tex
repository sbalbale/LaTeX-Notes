\documentclass{article}
\usepackage{amsmath}
\usepackage{amssymb}
\usepackage{fancyhdr}

\title{CS1800 Homework 4 Solutions}
\author{}
\date{}

% Header for every page
\pagestyle{fancy}
\fancyhf{}
\fancyhead[L]{Name: Sean Balbale}
\fancyhead[R]{HW Group: None}

\setlength\parindent{0pt}

\begin{document}

\maketitle
\newpage

\section*{Problem 1: Survey}

Given 20 multiple-choice questions, each with four possible answers (A, B, C, or D):

\subsection*{i. Unique ways to complete the entire survey}

Each question has 4 possible answers. Hence, the total number of ways to complete the survey is:

\[
4^{20} = 1,099,511,627,776
\]

\subsection*{ii. Unique ways to complete the survey if at least one question is unanswered}

Each question now has 5 options (A, B, C, D, or no answer). Thus, the total number of ways is:

\[
5^{20-1} = 5^{19} = 1.90734863* 10^{13}
\]

\subsection*{iii. How many ways can a participant respond if they select option A for exactly 8 questions and option B for the remaining 12?}

This is a combination problem, where we choose 8 questions from 20 for option A:

\[
\binom{20}{8} = \frac{20!}{8!12!} = 125,970
\]

\newpage

\section*{Problem 2: Soccer Team}

\subsection*{i. In how many ways can the coach choose 11 players to play?}

The number of ways to choose 11 players from 20 is:

\[
\binom{20}{11} = \frac{20!}{11!9!} = 167,960
\]

\subsection*{ii. How many ways can the coach choose a lineup?}

We need to choose:
\begin{itemize}
    \item 4 midfielders from 7: \( \binom{7}{4} \)
    \item 3 defenders from 6: \( \binom{6}{3} \)
    \item 3 attackers from 5: \( \binom{5}{3} \)
    \item 1 goalkeeper from 2: \( \binom{2}{1} \)
\end{itemize}

\noindent The total number of ways is:

\[
\binom{7}{4} \times \binom{6}{3} \times \binom{5}{3} \times \binom{2}{1} = 35 \times 20 \times 10 \times 2 = 14,000
\]

\subsection*{iii. How many ways can the coach choose a lineup if one defender can also play attack?}

In this case, there are 6 potential attackers and 6 defenders. The number of ways is:

\[
\binom{7}{4} \times \binom{6}{3} \times \binom{6}{3} \times \binom{2}{1} = 35 \times 20 \times 20 \times 2 = 28,000
\]

\newpage

\section*{Problem 3: Sharing Books}

\subsection*{i. How many ways can we distribute 15 unique books among 5 friends?}

The number of ways to distribute 15 unique books to 5 friends is:

\[
5^{15} = 30,517,578,125
\]

\subsection*{ii. How many ways to distribute 15 identical books among 5 friends?}

Using the stars and bars formula:

\noindent Let $k=15$ and $n=5$.

\[
\binom{n+k-1}{k-1} = \binom{15+5-1}{4} = \binom{19}{4} = 3,876
\]

\subsection*{iii. How many ways if one friend does not receive exactly 4 books?}

We subtract the number of ways where one friend gets exactly 4 books from the total:

\[
\text{Total ways} = \binom{19}{4}, \quad \text{Ways where one friend gets 4 books} = \binom{14}{3}
\]

Thus:

\[
\binom{19}{4} - \binom{14}{3} = 3,876 - 336 = 3,512
\]

\newpage

\section*{Problem 4: Principle of Inclusion-Exclusion Factors}

\subsection*{i. How many integers from 1 to 1000 are multiples of 7?}

\[
\left\lfloor \frac{1000}{7} \right\rfloor = 142
\]

\subsection*{ii. How many integers from 1 to 1000 are multiples of 11?}

\[
\left\lfloor \frac{1000}{11} \right\rfloor = 90
\]

\subsection*{iii. How many integers from 1 to 1000 are multiples of both 7 and 11?}

Multiples of both 7 and 11 are multiples of the LCM (77):

\[
\left\lfloor \frac{1000}{7 \cdot 11} \right\rfloor = 12
\]

\subsection*{iv. How many integers from 1 to 1000 are multiples of 7 or 11?}

Using inclusion-exclusion:

\[
142 + 90 - 12 = 220
\]

\subsection*{v. How many integers from 1 to 1000 are multiples of 7 but not 11?}

\[
142 - 12 = 130
\]

\subsection*{vi. How many integers from 1 to 1000 are multiples of neither 7 nor 11?}

\[
1000 - (142+90) = 768
\]

\newpage

\section*{Problem 5: Pigeonhole Principle}

\subsection*{i. How many pigeons, at least, are guaranteed to share the first nest?}

\[
\left\lceil \frac{14}{3} \right\rceil = \left\lceil 4.667 \right\rceil = 5
\]

\subsection*{ii. How many pigeons, at least, are guaranteed to be in the nest with the most pigeons?}

\[
\left\lceil \frac{14}{3} \right\rceil = 5
\]

\subsection*{iii. Minimum number of students to guarantee a table with at least 3 students}

By the pigeonhole principle:

\[
n = 29 \quad \text{(at least one table will have 3 students)}
\]

If you place 2 students at each of the 14 tables, that uses up 28 students.
However, when you add a 29th student, they will need to sit at one of the
tables that already has 2 students, meaning that table will now have 3 students.
Therefore, with 29 students, it's guaranteed that at least one table will have 3
students.

\subsection*{iv. What does the pigeonhole principle imply about 123 students taking an exam with 102 questions?}

Since there are 123 students but only 103 possible different scores, by the
pigeonhole principle, there must be at least two students with the same score.

\newpage

\section*{Problem 6: Rectangle Extra}

\subsection*{i. How many integer rectangles have an area of 720?}

The prime factorization of 720 is:

\[
720 = 2^4 \times 3^2 \times 5
\]

The total number of divisors (and thus possible rectangles) is:

\[
(4+1)(2+1)(1+1) = 30
\]

Thus, there are 30 distinct integer rectangles.

\end{document}
