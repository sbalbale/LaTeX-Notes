\documentclass{article}
\usepackage{amsmath}
\usepackage{amssymb}
\usepackage{fancyhdr}
\usepackage{lipsum}

\title{CS1800 Homework 3 Solutions}
\author{}
\date{}

% Header for every page
\pagestyle{fancy}
\fancyhf{}
\fancyhead[L]{Name: Your Name}
\fancyhead[R]{HW Group: None}

\begin{document}

\maketitle
\newpage

\section*{Problem 1: Beatles Set Representation}

Given:
\[
A = \{ \text{paul, george} \}, \quad B = \{ \text{ringo, george} \}, \quad U = \{ \text{john, paul, ringo, george} \}
\]
The bit string representations:
\[
A = 0110, \quad B = 0011
\]
\begin{enumerate}
    \item[i.] \( A \cup B \)
    \[
    A \cup B = 0110 \ \text{OR} \ 0011 = 0111
    \]
    \item[ii.] \( A \cap B \)
    \[
    A \cap B = 0110 \ \text{AND} \ 0011 = 0010
    \]
    \item[iii.] \( A^C \)
    \[
    A^C = 1001
    \]
\end{enumerate}

\newpage
\section*{Problem 2: Set Operations (Listing)}

Given:
\[
A = \{ 2, 4, 6, 8 \}, \quad B = \{ 1, 3, 5 \}, \quad U = \{ 1, 2, 3, 4, 5, 6, 7, 8, 9 \}
\]
\begin{enumerate}
    \item[i.] \( \{ x - 1 \in U \mid x \in A \} \)
    \[
    \{ x - 1 \mid x \in A \} = \{ 1, 3, 5, 7 \}
    \]
    \item[ii.] \( \{ x \in B \mid x \ \text{is even} \} \)
    \[
    \emptyset
    \]
    \item[iii.] \( \{ x \in A \mid x + 3 \in U \} \)
    \[
    \{ 6 \}
    \]
    \item[iv.] \( A \cap B \)
    \[
    \emptyset
    \]
    \item[v.] \( A \cup B \)
    \[
    A \cup B = \{ 1, 2, 3, 4, 5, 6, 8 \}
    \]
    \item[vi.] \( B - A \)
    \[
    B - A = \{ 1, 3, 5 \}
    \]
    \item[vii.] \( (A \cap B^C)^C \)
    \[
    (A \cap B^C)^C = \{ 1, 3, 5, 7, 9 \}
    \]
    \item[viii.] \( A \triangle B \)
    \[
    A \triangle B = \{ 1, 2, 3, 4, 5, 6, 8 \}
    \]
\end{enumerate}

\newpage
\section*{Problem 3: Set Operations (Shading)}
\begin{enumerate}
    \item[i.] \( A \cup (B - C) \)
    \item[ii.] \( (A \cup B) - C \)
    \item[iii.] \( A^C \cap B^C \)
    \item[iv.] \( ((A^C \cup B^C) \cup C^C)^C \)
    \item[v.] \( (B^C \cap B) \cup (C \cap A) \)
    \item[vi.] \( (C - A) \cup (A - B) \cup (B - C) \)
\end{enumerate}

\newpage
\section*{Problem 4: Set Algebra}
\begin{enumerate}
    \item[i.] \( A \cap A \)
    \[
    A \cap A = A
    \]
    \item[ii.] \( (A^C \cap B^C)^C \cap U \)
    \[
    (A^C \cap B^C)^C \cap U = (A \cup B)
    \]
    \item[iii.] \( (A \cup A) \cap (B \cup A^C) \)
    \[
    (A \cup A) \cap (B \cup A^C) = A \cap B
    \]
\end{enumerate}

\newpage
\section*{Problem 5: Set Builder Notation}
\begin{enumerate}
    \item[i.] Express the set:
    \[
    S = \{ n \in \mathbb{Z} \mid n \in \mathbb{N}, -5 \leq n < 7 \}
    \]
    The list:
    \[
    S = \{ 0, 1, 2, 3, 4, 5, 6 \}
    \]
    \item[ii.] Express the set \( B \) of integers whose fourth power is either 16 or 81:
    \[
    B = \{ x \in \mathbb{Z} \mid x^4 = 16 \text{ or } x^4 = 81 \}
    \]
    \item[iii.] The list for \( B \):
    \[
    B = \{ -3, -2, 2, 3 \}
    \]
\end{enumerate}

\newpage
\section*{Problem 6: Digital Circuit}
\begin{enumerate}
    \item[i.] Express \( Y \) in terms of \( A, B, C \) using logical operators.
    \item[ii.] Simplify \( Y \) using logic identities.
    \item[iii.] Draw the simplified logic circuit.
\end{enumerate}

\end{document}
